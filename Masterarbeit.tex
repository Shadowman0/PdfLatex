% Vorlage für eine Bachelorarbeit
% Siehe auch LaTeX-Kurs von Mathematik-Online
% www.mathematik-online.org/kurse
% Anpassungen für die Fakultät für Mathematik
% am KIT durch Klaus Spitzmüller und Roland Schnaubelt Dezember 2011

\documentclass[12pt,a4paper]{scrartcl}
% scrartcl ist eine abgeleitete Artik  el-Klasse im Koma-Skript
% zur Kontrolle des Umbruchs Klassenoption draft verwenden


% die folgenden Packete erlauben den Gebrauch von Umlauten und ß
% in der Latex Datei
\usepackage[utf8]{inputenc}
% \usepackage[latin1]{inputenc} %  Alternativ unter Windows
\usepackage[T1]{fontenc}
\usepackage[ngerman]{babel}
\usepackage{hyperref}

\usepackage[pdftex]{graphicx}
\usepackage{latexsym}
\usepackage{amsmath,amssymb,amsthm}
\usepackage{lmodern}
\usepackage{enumerate}   
\usepackage{todonotes}

% Abstand obere Blattkante zur Kopfzeile ist 2.54cm - 15mm
\setlength{\topmargin}{-15mm}


% Umgebungen für Definitionen, Sätze, usw.
% Es werden Sätze, Definitionen etc innerhalb einer Section mit
% 1.1, 1.2 etc durchnummeriert, ebenso die Gleichungen mit (1.1), (1.2) ..
\newtheorem{Satz}{Satz}[section]
\newtheorem{Def}[Satz]{Definition} 
\newtheorem{DefSatz}[Satz]{Definition und Satz} 
\newtheorem{Lemma}[Satz]{Lemma}	
\newtheorem{Bemerkung}[Satz]{Bemerkung}
\newtheorem{Kor}[Satz]{Korollar}
\newtheorem{Beispiel}{Beispiel}[section]
\newcommand{\dd}{\mathrm{d}} 
                  
\numberwithin{equation}{section} 

% einige Abkuerzungen
\newcommand{\C}{\mathbb{C}} % komplexe
\newcommand{\K}{\mathbb{K}} % komplexe
\newcommand{\R}{\mathbb{R}} % reelle
\newcommand{\Q}{\mathbb{Q}} % rationale
\newcommand{\Z}{\mathbb{Z}} % ganze
\newcommand{\N}{\mathbb{N}} % natuerliche
\newcommand{\F}{\mathcal{F}} %Fourier
\newcommand{\Sc}{\mathcal{S}} %SchwartzRaum
\newcommand{\BC}{\operatorname{BC}}
\newcommand{\laplace}{\bigtriangleup} 
\newcommand{\grad}{\bigtriangledown} 
\newcommand{\supp}{\operatorname{supp}} 

\newcommand{\per}{\operatorname{per}}
\newcommand{\DL}{\operatorname{DL}}
\newcommand{\erf}{\operatorname{erf}}
\newcommand{\erfc}{\operatorname{erfc}}
\newcommand{\fa}{\text{\quad für alle }}
\newcommand{\calA}{\mathcal{A}} 


\begin{document}
  % Keine Seitenzahlen im Vorspann
  \pagestyle{empty}

  % Titelblatt der Arbeit
  \begin{titlepage}

    \includegraphics[scale=0.45]{kit-logo.jpg} 
    \vspace*{2cm} 

 \begin{center} \large 
    
    Masterarbeit
    \vspace*{2cm}

    {\huge Approximation von schwach singulären
Randintegraloperatoren durch Glättung der Singularität}
    \vspace*{2.5cm}

    Peter Wegner
    \vspace*{1.5cm}

    Datum der Abgabe \\
    20. Januar 2017
    \vspace*{3.5cm}


    Betreuung: PD Dr. Tilo Arens \\[1cm]
    Fakultät für Mathematik \\[1cm]
		Karlsruher Institut für Technologie
  \end{center}
\end{titlepage}
\vspace*{1cm}
\newpage


  % Inhaltsverzeichnis
  \tableofcontents
%  \listoftodos
\newpage

 


  % Ab sofort Seitenzahlen in der Kopfzeile anzeigen
  \pagestyle{headings}
\section{Einführung}
\subsection{Motivation und Gliederung}
Die Theorie der Integralgleichungen stellt eine ebenso elegante wie anwendungsbezogene Mischung aus den Bereichen Analysis, Funktionalanalysis und Numerik dar. Gerade in den letzten Jahrzehnten haben sich dabei zahlreiche Anwendungsmöglichkeiten eröffnet. Ein prominentes Beispiel dafür ist die Computertomographie (CT) als wichtiges Werkzeug medizinischer Bildgebung. Die damaligen Lösungsmethoden der CT-Pioniere Anfang der 70er Jahre wären für die dabei auftretenden Integralgleichungen so aufwändig, dass ein Computer etwa 70 Stunden pro Einzelbild rechnen müsste.\footnote{vgl. Christof Schütte, "Was ein CT mit Algorithmen zu tun", in: welt.de (10.03.2008) unter: https://www.welt.de/wissenschaft/article1781601/Was-ein-CT-mit-Algorithmen-zu-tun.html (aufgerufen am 18.01.2017) } \\
Eine andere, mathematisch relevante Anwendung ist die Integralgleichungsmethode, bei der partielle Differentialgleichungen in äquivalente Integralgleichungen überführt werden. Dies eröffnet nicht nur einen funktionalanalytischen Zugang für eine Lösungstheorie, wie er in den Abschnitten \ref{Intmethode} und \ref{chaIntgl} angedeutet ist, sondern darüber hinaus auch die Möglichkeit konzeptionell einfache Verfahren zur numerischen Lösung der Differentialgleichungen zu verwenden. \\

Diese Arbeit setzt sich mit letzterer Möglichkeit auseinander und beruht auf den Arbeiten von \emph{J. Thomas Beale}. Dabei geht es um das numerische Lösen von Integralgleichungen mit schwach singulärer Kernfunktion. Aufgrund der unbeschränkten Kernfunktion erfordern diese Probleme üblicherweise aufwändigere Schritte um zu konvergenten Diskretisierungsverfahren zu gelangen.
\emph{Beale} löst dieses Problem, indem er den schwach singulären Integranden approximiert.
Eine Präsentation des Verfahrens findet sich in \cite{beale} und in ausführlicherer Bearbeitung in \cite{Collet}. 

In dem von uns betrachteten Fall lassen sich die Resultate von \emph{Beale} weiter verschärfen. Dazu wird in Kapitel \ref{chaKernapprox} am Beispiel der im Abschnitt \ref{Intmethode} eingeführten Laplacegleichung im $\R^3$ gezeigt, wie durch Glättung eine Approximation des auftretenden schwach singulären Integraloperators gewonnen werden kann. Die Fehlerordnung dieser Regularisierung werden wir in Kapitel \ref{chaRegErr} nachweisen. Auf das regularisierte Problem wenden wir im Anschluss das Nystrom-Verfahren an und erhalten ein vollständig diskretisiertes Problem, das sich mit üblichen Lösern für lineare Gleichungssysteme lösen lässt. 

Ein wesentlicher Schritt ist dabei die genauere Bestimmung der Konvergenzordnung mithilfe der Fouriertransformation im Abschnitt \ref{chaFourierKonstanen}.
Dabei werden wir zeigen, dass der diskretisierte regularisierte Integraloperator $A_{2,\delta,N}$ für die Wahl des Regularisierungsparameters $\delta=h^q$ für $q<1$, wobei $h$ der Gitterabstand der Diskretisierung und $2 N^2$ die Anzahl der Gitterpunkte ist, mit Ordnung $5q$ gegen den ursprünglichen Integraloperator $A$ konvergiert. Konkret bedeutet das, dass $||A-A_{2,\delta,N}|| \leq C h^{5q}$ für jedes $q<1$ mit einer von $\delta$ und $h$ unabhängigen Konstanten $C>0$.  

\subsection{Notation}
   \begin{table}[!ht]
    
     \begin{tabular}{ll}
       $\partial M$          & Rand der Menge $M \subset R^d (d \in \N)$          \\
       $\overline{M}$       & Abschluss der Menge $M \subset R^d (d \in \N)$          \\
       $C_{\per}^k(\R^2,\R^d)$        & Menge der $k$-mal stetig differenzierbaren Funktionen von $\R^2 \to \R^d, d \in \N$, \\& die in allen Variablen $2 \pi$-periodisch sind              \\
       $C$ &             generische Konstante, die von Gleichung zu Gleichung variieren kann. \\
       $\lceil \cdot \rceil$ & Aufrunde-Funktion \\
       $|\cdot|$   & Betragsfunktion/Euklidische Norm/Summe Komponenten für Vektoren aus $\N_0^n$ \\
       $||\cdot||_{k,\infty}$ & Supremumsnorm in $C^{k}(M)$, d.h. $||f||_{k,\infty}=\max_{|\alpha|\leq k}||D^\alpha f||_{\infty} $\\
       $O(f(x))$ & Landau-Symbol \\
       $D^\alpha$ & Differentialoperator. Für $\alpha \in \N_0^d$ ist $D^\alpha := \frac{\partial^{|\alpha|}}{\partial x_1^{\alpha_1} \dots \partial x_d^{\alpha_d}}$
     \end{tabular}
\end{table}
  
 \thispagestyle{empty}


\vspace*{5cm}


\section*{Erklärung}

Hiermit versichere ich, dass ich diese Arbeit selbständig verfasst und keine anderen, als die angegebenen Quellen und Hilfsmittel benutzt, die wörtlich oder inhaltlich übernommenen Stellen als solche kenntlich gemacht und die Satzung des Karlsruher Instituts für Technologie zur Sicherung guter wissenschaftlicher Praxis in der jeweils gültigen Fassung beachtet habe. \\[2ex] 

\noindent
Karlsruhe, den 20.01.2017\\[5ex]

% Unterschrift (handgeschrieben)


\newpage

\section{Grundlagen}
% \subsection{Motivation und Gliederung}
Wir wollen uns zunächst einen Überblick über die Grundkenntnisse, die zur Einordnung und dem Verständnis dieser Arbeit nötig sind, machen. Dabei wird der Abschnitt zur Fouriertransformation gerade bei der Bestimmung des Diskretisierungsfehlers in Kapitel \ref{chaDiskErr} von großer Bedeutung sein. Da uns die Fouriertransformation lediglich als Werkzeug dient, ist dieser Abschnitt über eine so umfangreiche Theorie bewusst so kurz wie möglich gehalten. Gerade Satz \ref{Feig} ist vor allem als Sammlung aller für Kapitel \ref{chaDiskErr} notwendigen Eigenschaften der Fouriertransformation anzusehen. Deshalb sind die einzelnen Aussagen in der Form, in der wir sie später verwenden werden, formuliert. 

\subsection{Integralgleichungsmethode} \label{Intmethode}
Zur numerischen Lösung partieller Differentialgleichungen sind diverse Verfahren praktikabel. Eine Klasse von Verfahren lässt sich bei Differentialgleichungen anwenden die sich auf Integralgleichungen zurückführen lassen. Durch das numerische Lösen der entsprechenden Integralgleichung lässt sich damit eine Lösung der Differentialgleichung bestimmen. Im Folgenden soll dies am Beispiel der Laplacegleichung vorgeführt werden. 
Sei von nun an $D \subset \R^3$ ein beschränktes Gebiet, d.h. offen und zusammenhängend.
\begin{Def}
	Sei $u\in C^2(D)$. Dann heißt die Funktion $u$ \emph{harmonisch}, wenn sie die Laplacegleichung erfüllt, d.h.
	\[
		\laplace u = 0 \text{ in } D,
	\]
	\\
	wobei wir mit
	\[
		\laplace u := \sum_{j=1}^3 \frac{\partial^2 u}{\partial x_j^2}
	\]
	den Laplace-Operator bezeichnen.
	
\end{Def}


Wir untersuchen im Folgenden die Laplacegleichung mit Dirichlet-Randbedingungen. Zudem gehen wir davon aus, dass der Rand $\partial D$ des Gebiets eine 2-mal stetig differenzierbare Parametrisierung  besitzt, d.h. $\partial D \in C^2$ (siehe Definition \ref{Randpar}). Damit suchen wir zu gegebenen Randdaten $g \in C^2(\partial D, \R)$ eine Funktion $u : \overline{D} \to \R$ mit
\begin{equation} \label{LGl}
 \begin{cases}
   \laplace u = 0 \text{ in } D \\
   u = -g \text{ auf } \partial D
 \end{cases}
\end{equation}  

Viele wichtige Eigenschaften harmonischer Funktionen lassen sich mit der Fundamentallösung der Laplacegleichung herleiten, die auch eine wesentliche Rolle bei der Integralgleichungsmethode spielt. 
\begin{DefSatz} \label{DFund}
  Die Funktion 
  \[
    \Phi(x,y) := \frac{1}{4 \pi} \frac{1}{|x-y|} \fa x,y \in \R^3, x\neq y
  \]
  heißt Fundamentallösung der Laplacegleichung. Für alle $y \in \R^3$ ist \[\Phi( \cdot, y) : \R^3 \setminus\{y\}  \to \R\] eine harmonische Funktion.

\end{DefSatz}   
\begin{proof}
  Direktes Nachrechnen.
\end{proof}
Mittels der Fundamentallösung lässt sich das betrachtete Dirichletproblem auf eine Integralgleichung zurückführen.
\begin{Def}
  Für $\varphi \in C(\partial D)$ heißt die Funktion
  \[    
    u(x) := \DL \varphi(x) := \int_{\partial D} \frac{\partial \Phi}{\partial n (y)}(x,y) \varphi(y) \dd s (y) \fa x \in \R^3 \setminus \partial D
  \] 
  \emph{Doppelschichtpotential der Laplacegleichung}, wobei $n: \partial D \to  \R^3$ die nach außen gerichtete Einheitsnormale bezeichnet. Dabei ist
\begin{align*}
\frac{\partial \Phi}{\partial n (y)}(x,y) &= n(y) \cdot \grad_y \Phi (x,y) \\
&=n(y) \cdot \frac{x-y}{4 \pi |x-y|^3}
\end{align*}
für alle $x \neq y$.
  
\end{Def}
\begin{Satz} \label{Intpart}
  Sei $\partial D \in C^2$ und $g \in C(\partial D, \R)$ eine Funktion. Dann ist $\DL \varphi$ genau dann eine Lösung der Dirichletgleichung \eqref{LGl}, wenn das Potential $\varphi : \partial D \to \R$ die Integralgleichung
  \begin{equation} \label{IntGl}
  \frac{1}{2}\varphi(x) - \int_{\partial D}\frac{\partial \Phi}{\partial n(y)}(x,y)\varphi(y)\dd s(y) = g(x) \fa x \in \partial D
  \end{equation}
  erfüllt. In diesem Fall ist $\varphi \in C(\partial D,\R)$ und eindeutig bestimmt.
\end{Satz} 
\begin{proof}
  Siehe \cite{kress} Theorem 6.22 und Theorem 6.23.
\end{proof}

Dieses Resultat ermöglicht einen Ansatz zur Lösung des Dirichletproblems über die Integralgleichung \eqref{IntGl}. Da es sich bei der Gleichung um eine Fredholmsche Integralgleichung zweiter Art handelt, lässt sich dafür mit der Riesz/Fredholmtheorie eine Existenztheorie aufbauen. Dazu soll der nächste Abschnitt die wichtigsten Aspekte für unser Modellproblem skizzieren.
Im Anschluss gehen wir auf das in dieser Arbeit verwendete numerische Verfahren ein.

\subsection{Integralgleichungen} \label{chaIntgl}
Da eine vollständige Existenztheorie den Umfang dieser Arbeit überschreiten und das Ziel verfehlen würde, soll an dieser Stelle lediglich eine kurze Erinnerung an die für unsere Zwecke nötigen Resultate gegeben werden. 
\begin{DefSatz} \label{SbeschrIntop}
Sei $ K : \overline D \times \overline D \to \C$ eine stetige Funktion. Dann heißt der  lineare Operator $A:C( \overline G) \to C( \overline G)$ definiert durch
\[
	(A \varphi)(x):= \int_{\overline D} K(x,y) \varphi (y) \dd y \fa x \in \overline{D}
\]
\emph{Integraloperator mit stetigem Kern K}. Der Operator ist beschränkt mit
\[
	||A||_\infty = \max_{x\in \overline D} \int_{\overline D} |K(x,y)| \dd y
\]
Darüber hinaus ist $A$ sogar kompakt.
\end{DefSatz}
\begin{proof}
Siehe Theorem 2.13 und Theorem 2.27 in \cite{kress}.


\end{proof}
Bei vielen Problemen ist der Kern auftretender Integraloperatoren wie bei unserem Modellproblem nicht stetig auf $G \times G$, aber immerhin \emph{schwach singulär} (siehe Gleichung \ref{IntGl}).
\begin{Def}
	Sei $K \in C(\{(t,s) \in \overline D \times \overline D \,|\, t \neq s\})$ und für ein $\alpha \in (0,n]$ gelte
\[
	|K(x,y)| \leq M |x - y|^{\alpha-n}. 
\]	
Dann heißen $K$ und der zugehörige Integraloperator \emph{schwach singulär}.
\end{Def}
\begin{Satz}
Schwach singuläre Integraloperatoren sind kompakt auf $C(\overline D)$.
\end{Satz}
\begin{proof}
Siehe Theorem 2.22 in \cite{kress}.
\end{proof}

Typischerweise sind die bei Randwertproblemen auftretenden Integraloperatoren durch Integrale über Oberflächen im $\R^n$ definiert, weshalb wir zunächst noch genauer darauf eingehen wollen was wir unter \emph{$C^m$-glatten Rändern} verstehen. 
\begin{Def} \label{Randpar}
Sei $D \subset \R^n$ ein beschränktes offenes Gebiet mit Rand $\partial D$ und $m \in \N$. Dann hat $D$ einen \emph{m-mal stetig differenzierbaren Rand}, schreibe $\partial D \in C^m$, wenn der Abschluss $\overline{D}$ eine endliche offene Überdeckung besitzt
\[
	\overline{D} \subset \cup_{q=1}^p V_q,
\]
so dass für alle offenen Mengen $V_q$ die den Rand $\partial D$ schneiden folgende Eigenschaften gelten:
\begin{itemize}
	\item Der Schnitt $V_q \cap \overline{D}$ kann bijektiv auf die Halbkugel $H:=\{x\in \R^n:|x|<1,x_n=0\}$ abgebildet werden.
	\item Diese Abbildung und ihre Inverse sind m-mal stetig differenzierbar.
	\item Der Schnitt $V_q \cap \partial D$ wird auf die Scheibe $H \cap \{x \in \R^n :x_n=0\}$ abgebildet.
\end{itemize}
Dies bedeutet insbesondere, dass sich der Rand $\partial D$ lokal parametrisieren lässt, d.h.
\[
	\partial D \ni x = x(u) = (x_1(u),\dots , x_m(u)),
\]
 wobei $u \in U \subset \R^{n-1}$, und $x:U \to S \subset \partial D$ bijektiv. Den ganzen Rand $\partial D$ erhält man durch endlich viele solcher Oberflächenstücke $S$.
\end{Def} 
Damit können wir uns Randintegraloperatoren wie den aus Gleichung \eqref{IntGl} anschauen. 

%\begin{Def}
%Die Funktion $K:\partial D \times \partial D \to \partial D$ heißt \emph{schwach singulär}, wenn sie für $x \neq y$ definiert und stetig ist und es $M>0$ und $\alpha \in (0,n-1]$ gibt so, dass
%\[
%	|K(x,y)| \leq M |x-y|^{\alpha-n+1} \fa x,y\in \partial D, x\neq y.
%\]  
%\end{Def}

\begin{DefSatz}
Sei $\partial D \in C^1$ und der Operator $A:C(\partial D) \to C( \partial D)$ definiert durch 
\[
	(A \varphi)(x) := \int_{\partial D} K(x,y) \varphi (y) \dd s(y) \fa x\in \partial D,
\]
wobei $K$ stetig oder schwach singulär ist. Dann ist $A:C(\partial D) \to C( \partial D)$ kompakt, wobei $C(\partial D)$ mit der Maximumsnorm 
\[
	||\varphi ||_\infty := \max_{x\in \partial D} |\varphi (x) |
\]
versehen ist.
\end{DefSatz} 
Mit dieser Kompaktheitseigenschaft lässt sich für unser Modellproblem eine Existenztheorie aufbauen. Das dritte Riesz-Theorem liefert dafür ein zufriedenstellendes Resultat. Die für uns wichtigste Folgerung der Riesztheorie gibt der nächste Satz wieder.
\begin{Satz}
  Sei $A:X \to X$ ein kompakter linearer Operator auf einem linearen Vektorraum $X$. Dann ist $I-A$ injektiv, genau dann wenn es surjektiv ist. Wenn $I-A$ injektiv ist, dann ist die Inverse $(I-A)^{-1}:X \to X$ beschränkt.
\end{Satz}  
\begin{proof}
Siehe Theorem 3.4 in \cite{kress}.
\end{proof}

Für die von uns betrachtete Integralgleichung \eqref{IntGl} bedeutet das, dass sich die Existenz der Lösung für jede rechte Seite $g$ ergibt, wenn gezeigt werden kann, dass es höchstens eine Lösung gibt. Für die Laplacegleichung lässt sich die Eindeutigkeit der Lösung direkt mit der Greenschen Formel zeigen (siehe Kapitel 7 in \cite{kress}) und wegen Satz \ref{Intpart} damit auch für die Integralgleichung.
    


\subsection{Quadraturformeln}
Eine Variante zur Lösung Fredholmscher Integralgleichungen der zweiten Art besteht darin, den auftretenden Integraloperator mithilfe von Quadraturformeln zu diskretisieren. Dadurch wird aus der Integralgleichung ein lineares Gleichungssystem, das sich numerisch leicht lösen lässt. \\
Dazu beginnen wir zunächst mit einer kurzen Erinnerung zu Quadraturformeln.
\begin{Def}
Sei $G\subset \R^n$ eine kompakte Menge und $f,w: G \to \R$. Dann bezeichnen wir das Funktional
\[
  Q_n(f):= \sum_{j=1}^n \alpha_j^{(n)}f(x_j^{(n)})
\]
mit gewichten $\alpha_j^{(n)}\in \R$ und Quadraturpunkten $x_j^{(n)} \in G$ für $n\in \N$ als \emph{Quadraturformel}, wenn es 
\[
 Q(f) := \int_G w(x)f(x) \dd x
\]
approximiert, d.h. $Q_n(f) \approx Q(f)$.
\end{Def}
Die einfachsten Quadraturformeln erhält man durch Interpolation des Integranden $x \mapsto w(x)f(x)$, z.B durch Polynome, trigonometrische Polynome oder Splines. Der interpolierte Integrand lässt sich einfach analytisch integrieren und man erhält eine Quadraturformel.    
\begin{Def}
  Eine Folge $(Q_n)$ von Quadraturformeln heißt \emph{konvergent}, wenn \[Q_n(f) \to Q(f)\] für $n \to \infty$. 
\end{Def}
Eine einfache Variante konvergente Quadraturformeln zu erhalten, besteht darin, den Integrationsbereich zu zerlegen und auf den Teilstücken eine festgelegte Quadraturformel zu verwenden. Von diesen so genannten \emph{zusammengesetzten Quadraturformeln} sind die \emph{zusammengesetzte Trapezregel} und die \emph{zusammengesetzte Simpsonregel} am gebräuchlichsten. Der folgende Satz gibt eine Fehlerabschätzung für die in dieser Arbeit verwendeten zusammengesetzten Trapezregel im Falle eines Intervalls als Integrationsbereich.
\begin{Satz} \label{Trapez}
  Sei $[a,b]\subset \R$ ein Intervall und $g\in C^2[a,b]$. Dann ist die zusammengesetzte Trapezregel zu äquidistanten Quadraturpunkten $x_j=a+j h$ für $j=0,\dots,n$ mit $h=(b-a)/n$ gegeben durch
\[
  Q^T_n(g) := h \left[ \frac{1}{2}(g(x_0)+g(x_n))+\sum_{j=1}^{n-1} g(x_j) \right].
\] 
Der Fehler genügt der Abschätzung
\[
  \left|Q^T_n(g)-\int_a^b g(x) \dd x \right| \leq \frac{1}{12} h^2(b-a)||g''||_\infty .
\]
\end{Satz}  
\begin{proof}
  Siehe Theorem 12.1 in \cite{kress}
\end{proof}
Wir werden die Trapezregel im Zuge dieser Arbeit auf ein Integral über ein 2-dimensionales Integrationsgebiet anwenden. Die dafür verwendete Fehlerabschätzung aus Lemma \ref{Lepsdisk} ist stärker als die eben gesehene, nutzt allerdings die höhere Glattheit und das Abfallverhalten des Integranden. Allgemein lassen sich für periodische glatte Funktionen stärkere Fehlerabschätzung für die Trapezregel zeigen, da man diese als Quadratur einer trigonometrischen Interpolation auffassen kann.
Ausgeführt wird dies beispielsweise bei \cite{kress} im Abschnitt 12.1.
\subsection{Nystrom-Verfahren} \label{chaNystrom}
Sei $Q_n$ eine konvergente Folge von Quadraturformeln zur Berechnung von $Q(g) = \int_G g(x) \dd x$. Dann lässt sich der Integraloperator
\[
  A(\varphi)(x):= \int_G K(x,y) \varphi(y) \dd(y) \fa x\in G
\]
mit stetigem Kern $K$ durch eine Folge von Integraloperatoren
\[
  (A_n\varphi)(x):=\sum_{k=1}^n \alpha_k^{(n)}K(x,x_k^{n)})\varphi(x_k^{(n)}), \fa x\in G
\]
approximieren, d.h. die Lösung der Fredholmschen Integralgleichung zweiter Art
\[
  \varphi -A \varphi = f
\]
durch die Lösung von
\[
  \varphi_n -A_n \varphi_n = f
\]
approximieren. Der folgende Satz zeigt, dass sich dieses diskretisierte Problem auf ein lineares Gleichungssystem reduzieren lässt.
\begin{Satz}
  Sei $\varphi_n$ eine Lösung von
  \begin{equation} \label{NysDiskGL}
   \varphi_n(x)-\sum_{k=1}^n \alpha_k K(x,x_k)\varphi(x_k)  = f(x) \fa x\in G.
  \end{equation}
  Dann erfüllen die Funktionswerte $\varphi_j^{(n)}=\varphi_n(x_j), j=1,\dots,n$ an den Quadraturpunkten $x_j$ das lineare Gleichungssystem
  \begin{equation} \label{NysLGS}
    \varphi_j^{(n)}-\sum_{k=1}^n \alpha_k K(x_j,x_k)\varphi_k^{(n)}=f(x_j) \fa j=1,\dots,n.
  \end{equation}
  Sei andererseits $\varphi_j^{(n)}, j=1,\dots,n$ eine Lösung von \eqref{NysLGS}. Dann erfüllt die Funktion $\varphi_n$ definiert durch
  \begin{equation} \label{NysMethode}
    \varphi_n(x):=f(x)+\sum_{k=1}^n\alpha_k K(x,x_k)\varphi_k^{(n)} \fa x \in G,
  \end{equation}
  die Gleichung \eqref{NysDiskGL}.
\end{Satz}
\begin{proof}
  Siehe Theorem 12.7 in \cite{kress}
\end{proof}
Die Gleichungen \eqref{NysLGS} und \eqref{NysMethode} geben das Vorgehen beim Nyströmverfahren vor. Das lineare Gleichungssystem \eqref{NysLGS} wird numerisch berechnet und daraus im Anschluss mit \eqref{NysMethode} die Lösung des diskretisierten Problems bestimmt. Um die Konvergenz zu beweisen und die Fehlerordnung zu bestimmen bietet es sich an, die auftretenden Operatoren $A$ und $A_n$ in einem funktionalanalytischen Rahmen zu betrachten. Ausführlich wird dies beispielsweise in Kapitel 10 in \cite{kress} getan. Wir beschränken uns an dieser Stelle auf das für unsere Zwecke wesentliche Resultat. Dazu benötigen wir zunächst ein wichtiges Konzept.
\begin{Def}
  Seien $X$ und $Y$ normierte Vektorräume. Eine Menge $\calA=\{A:X\to Y\}$ linearer Operatoren heißt \emph{kollektiv kompakt}, wenn für jede beschränkte Menge $ U \subset X$ das Bild $\calA(U):=\{A\varphi : \varphi \in U, A \in \calA\}$ relativ kompakt ist, d.h  $\overline{\calA(U)}$ kompakt. 
\end{Def}

\begin{Satz} \label{ErrSatz}
Sei $A:X\to X$ ein kompakter linearer Operator in einem Banachraum $X$ und sei außerdem $I - A$ injektiv. Sei weiterhin $A_n : X \to X$ eine kollektiv kompakte und punktweise konvergente Folge von Operatoren $A_n \varphi \to A \varphi$, $n \to \infty$ für $\varphi \in X$. \\
Dann gilt für genügend großes $n$, genauer für alle $n$ mit
\[
  ||(I-A)^{-1}(A_n-A)A_n|| < 1,
\]   
dass der inverse Operator $(I-A_n)^{-1}:X \to X$ existiert und beschränkt ist mit
\begin{equation}
||(I-A_n)^{-1}||\leq \frac{1+||(I-A)^{-1}A_n||}{1-||(I-A)^{-1}(A_n-A)A_n||}.
\end{equation}
Für die Lösungen der Gleichungen 
\[
  \varphi - A\varphi =f \text{ und }\varphi_n-A_n\varphi_n=f_n
\]
gilt
\[
  ||\varphi_n-\varphi||\leq \frac{1+||(I-A)^{-1} A_n||}{1-||(I-A)^{-1}(A_n-A)A_n||}\{||(A_n-A)\varphi||+||f_n-f||\}
\]
\end{Satz}
\begin{proof}
Siehe Theorem 10.9 in \cite{kress}
\end{proof}
Damit hängt die Genauigkeit der approximierten Lösung im Wesentlichen davon ab, wie gut der Operator $A$ durch $A_n$ approximiert wird.
\begin{Kor} \label{ErrSatz2}
Unter den Vorraussetzungen von Satz \ref{ErrSatz} gilt für den Fehler der Lösungen
\begin{equation}
|| \varphi_n - \varphi || \leq C \{||(A_n -A) \varphi||+||f_n-f||\}
\end{equation}
\end{Kor}
  
\begin{Satz}
  Für eine eindeutig lösbare Integralgleichung  zweiter Art mit stetiger Kernfunktion und stetiger rechten Seite und einer konvergenten Folge von Quadraturformeln ist das Nyströmverfahren gleichmäßig konvergent.
\end{Satz}
\begin{proof}
  Siehe Korollar 12.9 in \cite{kress}.
\end{proof}
Damit ist einerseits die Konvergenz des Verfahrens bewiesen, andererseits haben wir mit Satz \ref{ErrSatz} bzw. dem Korollar \ref{ErrSatz2} ein Mittel zur Hand um die Fehlerordnung des Verfahrens zu bestimmen. Diese ist identisch mit der Ordnung der zugrundeliegenden Quadraturformel, wenn die nötige Glattheit des Integranden gegeben ist. Im Falle der zusammengesetzten Trapezregel ergibt sich für $\varphi \in C^2[a,b]$ und $K \in C^2([a,b]\times [a,b])$ mit Satz \ref{Trapez} die Abschätzung
\[
||(A-A_n)\varphi ||_\infty \leq \frac{1}{12}h^2(b-a) \operatorname{max}_{x,y\in G}\left| \frac{\partial^2}{\partial y^2}K(x,y) \varphi(y)\right|.
\]
\\
Leider lässt sich das Nyströmverfahren mit der zusammengesetzten Trapezregel nicht auf das Dirichletproblem aus dem ersten Abschnitt anwenden, da der Kern des Integraloperators aus Gleichung \eqref{IntGl} schwach singulär ist. Eine Variante dieses Problem zu umgehen, ist es, andere Quadraturformeln zu wählen, die für die Integration schwach singulärer Integranden geeignet sind. Dies wird beispielsweise in Abschnitt 12.3 bei \cite{kress} gemacht. 
\\ Der Lösungsansatz dieser Arbeit besteht darin, den singulären Kern zu glätten, wodurch das entstehende modifizierte Problem leichter zu lösen ist. Dabei muss für die Analyse des vollständigen Fehlers neben dem Fehler der durch die Diskretisierung beim Nyströmverfahren gemacht wird, auch der Fehler, der bei der Regularisierung entsteht, berücksichtigt werden.

\subsection{Fouriertransformation} 
Da wir zur Analyse des Fehlers, der bei der Diskretisierung gemacht wird die Fouriertransformation verwenden werden, soll an dieser Stelle ein kurzer Überblick der für uns notwendigen Eigenschaften der Fouriertransformation gegeben werden. Eine systematische Einführung in die Fourieranalysis findet sich beispielsweise in \cite{Grafakos}.

Eine wesentliche Rolle beim Studium der Fouriertransformation spielen die sogenannten \emph{Schwartzfunktionen}. Grob gesagt handelt es sich dabei um glatte Funktionen, bei denen jede Ableitung gegen unendlich schneller fällt als der Kehrwert jedes Polynoms. Genauer lässt sich dies folgendermaßen formalisieren.    
\begin{Def}
Eine Funktion $f: \R^n \to \C$ heißt \emph{Schwartzfunktion} oder \emph{schnell-fallend}, wenn sie beliebig oft differenzierbar ist und wenn für alle Multiindizes $\alpha, \beta \in \N_0^n$ die Funktion \[
x \mapsto x^\alpha D^\beta f(x), \qquad x \in \R^n
\] auf $\R^n$ beschränkt ist.
Offensichtlich bildet die Menge aller schnell-fallenden Funktionen $\Sc(\R^n)$ einen Vektorraum, den man als Schwartz-Raum bezeichnet. Somit ist
\begin{align*}
\Sc(\R^n) &= \left\lbrace  \varphi \in C^\infty (\R^n) | \forall \alpha, \beta \in \N_0^n : p_{\alpha,\beta}(\varphi) < \infty\right\rbrace \\
&=\left\lbrace  \varphi \in C^\infty (\R^n) | \forall \alpha, \beta \in \N_0^n , \exists C \geq 0, \forall x\in \R^n :|x^\alpha D^\beta \varphi(x)| \leq C \right\rbrace
\end{align*}
mit den Halbnormen $ p_{\alpha,\beta}(\varphi) := \sup_{x\in \R^n}|x^\alpha D^\beta \varphi(x)|$ für alle $\alpha, \beta \in \N_0^n$.
\end{Def}

Wie man leicht einsieht enthält der Schwartz-Raum insbesondere alle glatten Funktionen mit kompaktem Träger, also die Menge $C_0^\infty (\R^n)$.
 
Induktiv lässt sich zeigen, dass $x \mapsto e^{-|x|^2}$, bzw. $x \mapsto e^{-c|x|^2}$ für alle $c>0$ ebenfalls eine Schwartzfunktion ist (vergleiche Example 2.2.2 in \cite{Grafakos}).\footnote{Der Beweis von Lemma \ref{LErf2Ordnung} lässt sich zum Induktionsanfang für den Fall $n=1$ erweitern.} Damit ist $C_0^\infty (\R^n)$ also eine echte Teilmenge des Schwartz-Raums. 
Anhand folgender Äquivalenz sieht man auch leicht ein, dass der Schwartz-Raum eine Teilmenge von $L^1(\R^n)$ ist:
Eine Funktion $f $ ist genau dann in $\Sc(\R^n)$, wenn es für alle $N>0$ und Multiindizes $\alpha \in \N_0^n$ ein $C_{\alpha,N}>0$ gibt, so dass 
\[
|(D^\alpha f)(x)| \leq C_{\alpha,N} (1 + |x|)^{-N}
\] (siehe Remark 2.2.4 in \cite{Grafakos}).
%Bevor wir uns der Fouriertransformation widmen brauchen wir noch einen Satz über die Abgeschlossenheit des Schwartz-Raums unter Faltung.
%\begin{Satz}
%Sei $f,g \in \Sc(\R^n)$. Dann ist $f g \in \Sc(\R^n)$ und $f \ast g \in \Sc(\R^n)$ und es gilt für alle Multiindizes $\alpha \in \N_0^n$
%\[
%D^\alpha(f \ast g) = (D^\alpha f) \ast g = f \ast (D^\alpha g),
%\]wobei die Faltung $f \ast g$ gegeben ist durch 
%\[
%(f \ast g)(x) = \int_{\R^n} f(y) g(x-y) \dd y.
%\]
%\end{Satz}
%\begin{proof}
%Siehe Proposition 2.2.7 in \cite{Grafakos} .
%\end{proof}

\begin{Def}
Für $f \in L^1 (\R^n)$ ist die \emph{Fouriertransformation} $\F f : \R^n \to \C$ gegeben durch
\[
	\F f (\xi) := \frac{1}{(2 \pi)^{n/2}} \int_{\R^n} f(\tau) e^{-i \xi \cdot \tau} \dd \tau, \qquad \xi \in \R^n.
\]
\end{Def}
Für den Integranden in der obigen Definition $g(\tau,\xi)=  f(\tau) e^{-i \xi \cdot \tau}$ gilt offensichtlich $|g(\tau,\xi)|=|f(\tau)|$, woraus mit dem Satz von Lebesgue die Existenz des Integrals folgt. Weiterhin ist die Abbildung $\xi \mapsto g(\tau,\xi)$ stetig für fast alle $\tau \in \R^n$ und der Satz von Lebesgue liefert sogar 
\[
\F f \in \BC(\R^2)=\{f \in C(\R^n)| \quad ||f||_\infty < \infty \}.
\]

Auf dem Schwartz-Raum lassen sich nun diverse Abbildungseigenschaften der Fouriertransformation nachweisen. Insbesondere lässt sich zeigen, dass die Fouriertransformation $\F : \Sc(\R^n) \to \Sc(\R^n)$ eine Bijektion und sogar Isometrie bezüglich $||\cdot ||_2$ ist. Wegen der Dichtheit von $C_0^\infty(\R^n) \subset L^2(\R^n)$ lässt sich die Fouriertransformation damit als beschränkter linearer Operator von $L^2(\R^n)$ nach $L^2(\R^n)$ fortsetzen. \\
Wir beschränken uns an dieser Stelle auf die für unsere Zwecke notwendigen Eigenschaften und fassen diese in einem Satz zusammen. Wegen des Zusammenhangs der Fouriertransformation mit der Faltung zweier Funktionen erinnern wir kurz an die Definition.
Für Funktionen $f,g \in L^1(\R^n)$ ist die Faltung von $f$ und $g$ definiert durch
\[
(f \ast g)(x) = \int_{\R^n} f(y) g(x-y) \dd y \qquad \fa x \in \R^n.
\]
\begin{Satz} \label{Feig}
Für die Fouriertransformation im $\R^2$ bzw. im $\R^n$ gilt
\begin{enumerate}[(i)]  
\item \label{Feig6}$\F[f(h \cdot)](x) = (1/h^2) \F f (x/h)$ für alle $x \in \R^2$ und $h>0$, falls $f \in \Sc(\R^2)$
\item \label{FeigPoisson}$\sum_{\nu \in \Z^2}f(\nu) = 2 \pi \sum_{\mu \in \Z^2}\F f(2\pi \mu)$ für $f \in C_0^2(\R^2)$ (Poisson'sche Summenformel)
\item \label{Feig7}$\F ( f \ast g ) = 2 \pi \F f \F g$ für $f,g \in L^1(\R^2)$
\item \label{Feig1} Für $G\in BC(\R^2)$, $g \in \Sc(\R^2)$ und $\lambda \in \R ^2$ gilt: 
\[
 \lim_{h \to \infty} \frac{1}{h^2} \int_{\R^2} G(\tau) g \left(\frac{\lambda - \tau}{h} \right)\dd \tau = 2 \pi G(\lambda) \F g(0)
\]
\item \label{Feig2} $\lim_{x\to \infty} \frac{\F f (x)}{|x|^2} = 0$ für $f \in C^2(\R^2) \cap L^1(\R^2)$ mit $D^\alpha f \in L^1(\R^2)$ für jeden Multiindex $|\alpha|=2$.
\item $g(x) = \F \F g(-x)$ für $g \in \Sc(\R^2)$  \label{Feig3}
\item \label{Feig4} Für $f\in L^1(\R^n)$ mit $\F f \in L^1(\R^n)$ gilt die Formel
\[
\F^{-1} \F f(x) = f(x) = \frac{1}{(2\pi)^{n/2}}\int_{\R^n} \F f(\tau) e^{i \tau x} \dd \tau
\]
\item \label{Feig5} Sei $k \in \N$ und $\alpha\in \N_0^n$ mit $|\alpha| \leq k$. Dann gilt für $f \in L^1(\R^n) \cap C^k(\R^n)$ mit $D^\alpha f \in L^1(\R^n)$
\[
\F ( D^\alpha f)(x) = (i x)^\alpha \F f (x)
\] 
\end{enumerate}
\end{Satz}
\begin{proof}
% und Theorem 3.2.8
Für \eqref{Feig6} und \eqref{Feig7} vergleiche Proposition 2.2.11 in \cite{Grafakos}\footnote{Die unterschiedliche Skalierung in der Definition der Fouriertransformation führt zu anderen Konstanten, die Beweise lassen sich aber genauso führen.}.  \\
Die Aussage \eqref{FeigPoisson} folgt beispielsweise aus Korollar 2.6 in \cite{Stein}, wenn wir noch die Soboleveinbettung  $C_0^2 (\R^2) \subset H^2(\R^2)$ (vergleiche Theorem 8.5 in \cite{kress}) nutzen oder lässt sich wie in \cite{Dym} Abschnitt 2.7 direkt beweisen. \\
Für \eqref{Feig1} wenden wir den Transformationssatz mit der Substitution $\xi = \frac{\lambda-\tau}{h}$ an mit $|\det \frac{\partial \xi}{\partial \tau}| = 1/h^2 $ und erhalten
\begin{align*}
\frac{1}{h^2} \int_{\R^2} G(\tau) g \left(\frac{\lambda - \tau}{h} \right)\dd \tau &= \int_{\R^2} G(\lambda - h \xi) g (\xi)\dd \xi 
\end{align*}
Der Integrand besitzt mit $||G||_\infty g$ eine integrierbare Majorante für $h \to 0$, d.h. Grenzwert und Integration dürfen vertauscht werden und  es folgt
\begin{align*}
\lim_{h \to 0}\frac{1}{h^2} \int_{\R^2} G(\tau) g \left(\frac{\lambda - \tau}{h} \right)\dd \tau &= \lim_{h \to 0} \int_{\R^2} G(\lambda - h \xi) g (\xi)\dd \xi \\
& = G(\lambda) \int_{\R^2} g(\xi) \dd \xi \\
&= 2 \pi G(\lambda) \F g(0) .
\end{align*}


Die Aussage \eqref{Feig3},\eqref{Feig4} und \eqref{Feig5} sind direkte Folgen von Theorem 2.4, Korollar 1.21 und Theorem 1.8 in \cite{Stein}. 
Für \eqref{Feig2} lässt sich das Riemann-Lebesgue Lemma (Prop. 3.3.1 in \cite{Grafakos}) auf die Funktion $\partial^{(2,0)} f + \partial^{(0,2)}f$ anwenden und die Behauptung folgt aus der Eigenschaft \eqref{Feig5}.
\end{proof}
\newpage
  \newpage
\section{Kernapproximation} \label{chaKernapprox}
Die Grundidee dieser Arbeit ist es, die in der Integralgleichung \eqref{IntGl} auftretende schwach singuläre Kernfunktion $\frac{\partial \Phi}{\partial n (y)}(x,y)$ mit einer Glättungsfunktion $s$ zu multiplizieren, sodass die Singularität gehoben wird. Dazu bietet es sich zunächst an, die Singularität mithilfe des nächsten Satzes zu beschränken.
\begin{Satz} \label{MinSing}
Es gilt
\begin{equation}
\int_{\partial D} \frac{\partial \Phi(x,y)}{\partial n (y)} \dd s(y) = - \frac{1}{2}
\end{equation}
und damit ist \eqref{IntGl} äquivalent zu 
\begin{equation} \label{IntGl2}
\varphi(x) - \int_{\partial D} \frac{\partial \Phi(x,y)}{\partial n (y)}(\varphi(y)-\varphi(x)) \dd s(y) = - g(x) \fa x \in \partial D
\end{equation} 
\end{Satz}
\begin{proof}
Siehe Example 6.17 in \cite{kress}
\end{proof}
Der so erhaltene Integrand ist wegen der geforderten Stetigkeit von $\varphi$ und des asymptotischen Verhaltens der Kernfunktion bei $0$ beschränkt. Wir approximieren nun den singulären Anteil des Integranden mithilfe geeigneter Glättungsfunktionen.
\begin{Def}
Seien die \emph{Glättungsfunktionen} $s_1$,$s_2$ für $r>0$ definiert durch
\begin{align*}
s_1(r) &= \erf(r) -\frac{2}{\sqrt{\pi}}r e^{-r^2}, \\
s_2(r) &= \erf(r) -\frac{2}{\sqrt{\pi}}\left(r- \frac{2}{3}r^3\right) e^{-r^2},
\end{align*}
wobei die \emph{Gaußsche Fehlerfunktion} $\erf: \C \to \C$ durch 
\[
\erf(z) = \frac{2}{\sqrt{\pi}}\int_0^z e^{-t^2} \dd t \fa z\in\C
\]
gegeben ist. Die \emph{komplementäre Fehlerfunktion} $\erfc$ ist definiert durch 
\[
\erfc(z) := 1 - \erf(z) = \frac{2}{\sqrt{\pi}}\int_0^z e^{-t^2} \dd t \fa z\in\C.
\]
 Die \emph{Kernfunktion} sei definiert durch 
\[
K(x,y) :=\frac{\partial \Phi(x,y)}{\partial n (y)} \fa x,y\in \R^3,x\neq y.
\]
Für $\delta>0$ sei die \emph{approximierte Kernfunktion} definiert durch
\begin{equation} \label{defKd}
K_{j,\delta}(x,y) := \frac{\partial \Phi(x,y)}{\partial n (y)} s_j \left(\frac{|x-y|}{\delta}\right) \approx \frac{\partial \Phi(x,y)}{\partial n (y)} \fa x,y \in \R^3,x \neq y
\end{equation}
\end{Def}
Durch Multiplikation der Kernfunktion mit der Glättungsfunktion wird die Singularität gehoben. Dies lässt sich am einfachsten mithilfe der Potenzreihendarstellung der Fehlerfunktion einsehen.

\begin{Lemma} \label{erfseries}
Für alle $z \in \C$ gilt
\[
\erf(z) = \frac{2}{\sqrt{\pi}}\sum_{n=0}^\infty \frac{(-1)^n z^{2 n +1}}{n!(2 n +1 )}.
\]
Außerdem gilt $\lim_{z \to \infty}\erf(z)=1$.
\end{Lemma}
\begin{proof}
Siehe Kapitel 7 in \cite{Stegun}.
\end{proof} 
Zusammen mit der folgenden Identität erkennt man, dass es sich bei $K_{1,\delta}(x,y)$ um eine analytische Funktion in $x$ und $y$ handelt.  
\begin{Satz} \label{k1ddef}
Es gilt
\begin{equation}
    K_{1_,\delta}(x,y) = n(y) \cdot \grad_y \left[ \Phi (x,y) \erf \left(\frac{|x-y|}{\delta}\right) \right]. \label{k1ddefeq}
\end{equation}
und
\begin{equation}
    K_{2_,\delta}(x,y) = n(y) \cdot \grad_y \left[ \Phi (x,y) (\erf \left(\frac{|x-y|}{\delta}\right) +\frac{2}{3\sqrt{\pi}}\frac{|x-y|}{\delta}e^{-\frac{|x-y|^2}{\delta^2}}) \right]. \label{k2ddefeq}
\end{equation}
für alle $x,y \in \R^3, x \neq y$.
\end{Satz}
\begin{proof}
Wir definieren für $x\in \R^3, x \neq 0$ die Funktion 
\[
G(x) = \frac{1}{4 \pi |x|}
\]
also $\Phi(x,y) = G(x-y)$. Es ist $ \grad G(x) = - G(x) \frac{x}{|x|^3}$ für $x \neq 0$ und damit
\begin{align*}
\grad \left[ G(x) \erf \left(\frac{|x|}{\delta}\right) \right] &= \grad G(x) \erf \left(\frac{|x|}{\delta}\right)+ G(x) \frac{2x}{\sqrt{\pi}|x| \delta}e^{-\frac{|x|^2}{\delta^2}} \\
&= \grad G(x) \erf \left(\frac{|x|}{\delta} \right)+ G(x) \frac{2x |x|}{\sqrt{\pi}|x|^2 \delta}e^{-\frac{|x|^2}{\delta^2}}\\
&=\grad G(x) \erf \left(\frac{|x|}{\delta}\right) - \grad G(x) \frac{2 |x|}{\sqrt{\pi} \delta}e^{-\frac{|x|^2}{\delta^2}} \\
&= \grad G(x)s_1 \left( \frac{|x|}{\delta}  \right)
\end{align*}
und
\begin{align*}
&\grad \left[ G(x) \left(\erf \left(\frac{|x|}{\delta}\right) +\frac{2}{3\sqrt{\pi}}\frac{|x|}{\delta}e^{-\frac{|x|^2}{\delta^2}} \right) \right] \\
&= \grad G(x)s_1 \left( \frac{|x|}{\delta}  \right)+ \grad G(x) \frac{2 |x|}{3 \sqrt{\pi}\delta}e^{-\frac{|x|^2}{\delta^2}} +G(x)\frac{2 x}{3 \sqrt{\pi} |x|^2} \left( \left( \frac{|x|}{\delta} -\frac{2}{\delta^3} |x| x \right) e^{-\frac{|x|^2}{\delta^2}} \right) \\
&= \grad G(x) \left( s_1  \left(\frac{|x|}{\delta} \right) +\frac{4 |x|^3}{3 \sqrt{\pi} \delta^3}e^{-\frac{|x|^2}{\delta^2}} \right) \\
&= \grad G(x) s_2 \left(\frac{|x|}{\delta} \right)
\end{align*}
womit die Behauptung folgt.
\end{proof}
\begin{Bemerkung}
Zur Vereinfachung der Notation werden wir im Folgenden häufig die unsaubere Notation
\begin{align*}
&\quad \ \grad_y \left[\Phi (x_0,y_0) \erf \left(\frac{|x_0-y_0|}{\delta}\right) \right]\\
&:=\grad_y \left[(x,y) \mapsto \Phi (x,y) \erf \left(\frac{|x-y|}{\delta}\right) \right](x_0,y_0)
\end{align*}
für $x_0,y_0 \in \R^3, x_0\neq y_0$ verwenden.


\end{Bemerkung}
\begin{Satz} 
Die Funktion $K_{1,\delta}$ ist analytisch und besitzt die Reihendarstellung
\begin{equation}
K_{1,\delta}(x,y) =  -n(y) \cdot (x-y) \left[\frac{1}{\pi^{3/2} \delta^3} \sum_{n=1}^\infty \frac{(-1)^n(\frac{|x-y|}{\delta})^{2 (n-1)}}{(n-1)! (2n+1)} \right] \fa x,y \in \R^3
\end{equation}
\end{Satz}
\begin{proof}
Mit Lemma \ref{erfseries} und $r = |x-y|$ erhält man die Reihendarstellung
\begin{align*}
\Phi_L (x,y) \erf \left(\frac{|x-y|}{\delta}\right) & = \frac{\erf(\frac{r}{\delta})}{4 \pi r} \\
& =  \frac{1}{2 \pi^{3/2} \delta} \sum_{n=0}^\infty \frac{(-1)^n(\frac{r}{\delta})^{2 n}}{n! (2n+1)} \fa x,y \in \R^3.
\end{align*} 
Weiter gilt 
\[ 
\grad_y r^{2n} = 2n r^{2n-1} \frac{-(x-y)}{r}= -2n(x-y) r^{2(n-1)} \mathrm{\, für \,}n>0.
\] 
Somit folgt aus Satz \ref{k1ddef}
\begin{align*}
K_{1_,\delta}(x,y) &= n(y) \cdot \grad_y \left[ \Phi (x,y) \erf \left(\frac{|x-y|}{\delta}\right) \right] \\
&=n(y) \cdot  \grad_y \left[\frac{1}{2 \pi^{3/2} \delta} \sum_{n=0}^\infty \frac{(-1)^n(\frac{r}{\delta})^{2 n}}{n! (2n+1)} \right] \\
&=-n(y) \cdot(x-y) \left[\frac{1}{\pi^{3/2} \delta^3} \sum_{n=1}^\infty \frac{(-1)^n(\frac{r}{\delta})^{2 (n-1)}}{(n-1)! (2n+1)} \right]
\end{align*}
für alle $x,y \in \R^3$.

\end{proof}
Mithilfe der Definition aus Gleichung \eqref{defKd} lässt sich folgern, dass $K_{2,\delta}$ ebenfalls analytisch ist, denn es gilt 
\begin{align*}
K_{2,\delta}(x,y) &= \frac{\partial \Phi(x,y)}{\partial n (y)} s_2 \left(\frac{|x-y|}{\delta}\right) \\
  &= \frac{\partial \Phi(x,y)}{\partial n (y)} \left(s_1 \left(\frac{|x-y|}{\delta}\right) + \frac{2}{3}r^3 e^{-r^2} \right)\\
  &= K_{1,\delta}(x,y)  + \frac{\partial \Phi(x,y)}{\partial n (y)}\frac{2}{3}r^3 e^{-r^2} \\
  &=  K_{1,\delta}(x,y) + n(y) \cdot \frac{2(x-y)}{12\pi}e^{-r^2}
\end{align*}
für alle $x,y \in \R^3$, wobei $r=|x-y|$.
Nun können wir das regularisierte Problem formulieren.
\subsection{Regularisiertes Problem} 
Wir betrachten die Gleichung \eqref{IntGl2} und ersetzen den singulären Integranden durch den approximierten aus dem letzten Abschnitt, womit wir die Gleichung
\begin{equation} \label{IntGl3}
\varphi(x) - \int_{\partial D} K_{j,\delta}(x,y)(\varphi(y)-\varphi(x)) \dd s(y) = - g(x) \fa x \in \partial D
\end{equation}
erhalten. Da wir $\partial D$ als $C^2$-glatt vorausgesetzt haben, gibt es nach Definition \ref{Randpar} 2-mal stetig differenzierbare lokale Parametrisierungen von $\partial D$. Wir gehen im Zuge dieser Arbeit sogar davon aus, dass es eine globale reguläre Parametrisierung $\eta \in C^2(Q, \partial D)$ mit $Q = [-\pi, \pi]^2$ gibt.
Dadurch müssen wir den Integrationsbereich in Gleichung \eqref{IntGl3} nicht zerlegen um unsere Parametrisierungen einsetzen zu können. Insbesondere die Darstellung des Fehler in Kapitel \ref{chaDiskErr} wird dadurch eleganter und wir können Resultate für den Fehler der Trapezregel bei periodischen Funktionen direkt verwenden.
Der allgemeine Fall mit lokalen Parametrisierungen wird beispielsweise bei \cite{Collet} behandelt. \\
 
Wir setzen nun die Parametrisierung des Randes $\eta : Q \to D$  in \eqref{IntGl3} ein und transformieren das Oberflächenintegral zu einem Integral über den Parameterbereich, womit wir die Gleichung 
\begin{align*}
\varphi(\eta(t)) - \int_{Q} K_{j,\delta}(\eta(t),\eta(\tau))(\varphi(\eta(\tau))&-\varphi(\eta(t))) |\partial_1 \eta (\tau) \times \partial_2 \eta(\tau)| \dd \tau \\&= - g(\eta(t)) \fa t \in Q
\end{align*}
erhalten. Zur Vereinfachung der Notation definieren wir die Kernfunktionen
\begin{align*}
K(t,\tau) &= \frac{\partial \Phi(\eta(t),\eta(\tau))}{\partial n (\eta(\tau))}|\partial_1 \eta (\tau) \times \partial_2 \eta(\tau)|, &t \neq \tau \\
K_{j,\delta}(t,\tau)&= K(t,\tau) s_j \left(\frac{|\eta(t)-\eta(\tau)|}{\delta}\right)
\end{align*}
für $t,\tau \in Q$. Weiterhin definieren wir für $\varphi \in C(Q)$ die Operatoren 
\begin{align*}
A \varphi(t) &= \int_Q K(t,\tau) (\varphi(\tau)- \varphi(t)) \dd \tau \\
A_{j,\delta} \varphi(t) &= \int_Q K_{j,\delta}(t,\tau) (\varphi(\tau)- \varphi(t)) \dd \tau.
\end{align*}
Damit lassen sich \eqref{IntGl2} und \eqref{IntGl3} schreiben als 
\begin{align*}
\varphi - A \varphi = -g  \\
\varphi - A_{j,\delta} \varphi = -g
\end{align*}
Zur Analyse des Fehlers, der bei der Regularisierung des Kerns gemacht wird, untersuchen wir die Norm der Differenz der Operatoren $||A- A_{j,\delta}||$ in Kapitel \ref{chaRegErr}. 
\subsection{Diskretisiertes Problem} \label{DiskProblem}

%Zur einfacheren Analyse des Fehlers der bei der Diskretisierung entsteht, gehen wir davon aus, dass $Q=[-\pi,\pi]^2$ als Parameterbereich des Randes gewählt werden kann. 
Da wir $Q=[-\pi,\pi]^2$ als Parametergebiet gewählt haben, können $\varphi\in C_{\mathrm{per}}(\R^2)$ und $K$ und $K_{j,\delta}$ als in beiden Variablen $2 \pi$-periodische Funktionen betrachtet werden. Genauso können wir die Parametrisierung $\eta$ als $\eta \in C_{\per}^2(\R^2,\partial D)$ auffassen. Für das Nyströmverfahren wählen wir als Quadraturformel die zusammengesetzte Trapezregel zu den Gitterpunkten $t^{(\mu)}=h \mu$, wobei $\mu \in \Z_N^2=\{\nu =(\nu_1,\nu_2)^T \in \Z^2 : -N+1 \leq \nu_j \leq N, j= 1,2 \}$ mit $h=\pi/N$ für $N\in \N$.
\\ Unter Anwendung des Nyströmverfahrens auf die Integralgleichung mit regularisiertem Kern erhalten wir den diskreten Operator
\[
A_{j,\delta,N} \varphi(t) = \frac{\pi^2}{N^2}\sum_{\nu\in\Z_N^2}  K_{j,\delta}(t,t^{(\nu)}) \left(\varphi(t^{(\nu)})- \varphi(t)\right).
\] und die Gleichung
\[
\varphi - A_{j,\delta,N} \varphi = -g.
\] Mithilfe der Dreiecksungleichung brauchen wir wegen
\[
||A- A_{j,\delta,N}|| \leq ||A- A_{j,\delta}|| + || A_{j,\delta}-A_{j,\delta,N}||
\] 
für den Gesamtfehler noch eine Abschätzung für   $|| A_{j,\delta}-A_{j,\delta,N}||$. Diese stellt den größten Teil der Arbeit dar und findet sich in Kapitel \ref{chaDiskErr}.
%\subsection{Spezielles Problem}
%In dieser Arbeit sollen die obigen Fehler für ein konkretes Gebiet bestimmt werden. Wir wählen für $D \subset \R^3$ einen Torus dessen Oberfläche durch die Parametrisierung  
%\[
%\eta(t) = \begin{pmatrix} 1 \\ 2 \\ 3 
%\end{pmatrix}
%\]
\newpage
\section{Fehler der Regularisierung} \label{chaRegErr} 
In diesem Abschnitt soll eine Abschätzung für den Fehler bei der Regularisierung  \mbox{$||A- A_{j,\delta}||$} für $j=1,2$ gegeben werden. 
Dazu schauen wir uns den lokalen Fehler $\varepsilon = (A- A_{j,\delta})\varphi (t)$  für $\varphi \in C_{\per}^3(\R^2)$ und $t \in Q$ an und definieren $x=\eta(t)$ und $r=r(\tau)=|x-\eta(\tau)|$.

Wir werden dabei zunächst ein paar vereinfachende Annahmen an die Parametrisierung stellen, die dafür sorgen, dass die von uns verwendeten Taylorentwicklungen einfacher werden. Anschließend werden wir aber sehen, warum wir diese Annahmen ohne Einschränkungen machen können. \\
Im Anschluss zeigen wir mithilfe der schnellen Konvergenz der Fehlerfunktion $\erf(z) \to 1$ für $z \to \infty$ in Lemma \ref{LepsohneSing}, dass der lokale Fehler, wenn man beim Integrieren der Singularität fernbleibt, schnell fällt. 


\subsection{Vorüberlegungen zur Parametrisierung} \label{ParametrisierungVoraussetzungen}
Wir werden den Regularisierungsfehler im Wesentlichen mithilfe der Taylorentwicklung des Integranden abschätzen. Deshalb stellen wir einige Voraussetzungen an die Parametrisierung $\eta$, die die Entwicklungen vereinfachen. Zunächst sei an die Taylorentwicklung erinnert.

\begin{Satz} \label{STaylor}
Sei $F\in C^r(D)$ mit $r\in \N$ und $D \subset \R^n$, dann gilt für $z,z_0 \in D$
\begin{align}
F(z) = \sum_{k=0}^{r-1} \sum_{|\alpha|=k} \frac{(z-z_0)^\alpha}{\alpha!}D^\alpha F(z_0) + R_{F,r}(z)
\end{align}
mit 
\[
 R_{F,r}(z) = \sum_{|\alpha|=r}\frac{r}{\alpha!}\int_0^1 (1-\xi)^r D^\alpha F(z_0 + \xi(z-z_0)) \dd \xi(z-z_0)^\alpha
\]
\end{Satz}  
\begin{proof}
%Siehe $\S 7$ Satz 2 in \cite{Forster} \todo{Quelle mit Integraldarstellung angeben}
Siehe I.1 in \cite{Grafakos}.
\end{proof}
\paragraph{Zusatzvoraussetzungen:} 
Da wir die Taylorentwicklung aller von $\tau$ abhängigen Terme brauchen, nehmen wir vereinfachend an, dass $\eta(0)=0$ und $t=0$, wobei $n_0=n(\eta(0))$ die Einheitsnormale auf $\partial D$ bei $\eta(0)$ ist. 
%Für den allgemeinen Fall kann für beliebiges fest gewähltes $x \in \partial D$ eine Verschiebung des Koordinatensystems vorgenommen werden und eine passende Parametrisierung $\eta$ gewählt werden.
\\
Außerdem fordern wir, dass die Tangenten der Parametrisierung $T_j(\tau):=\partial_j \eta(\tau)$ in $0$ senkrecht zueinander stehen und normiert sind, d.h. der metrische Tensor $g_{ij}(\tau) := T_i(\tau) \cdot T_j(\tau),\, j=1,2$ ist bei $\tau=0$ die Identität. Weiterhin fordern wir, dass $\partial g_{ij}/\partial \tau_k (0)=0, \, i,j,k=1,2$ und $T_1$ und $T_2$ in Richtung der Hauptkrümmungen zeigen. Die Hauptkrümmungswerte bezeichnen wir im  Folgenden mit $\kappa_j$, $j=1,2$.\footnote{Für eine Einführung der verwendeten differentialgeometrischen Begriffe ist \cite{Borceux}, speziell Kapitel 5.10 zu empfehlen.}





\begin{Lemma} \label{Leta}
Unter den obigen Voraussetzungen und falls $\eta \in C_{\per}^3(\R^2,\partial D)$ existieren Funktionen $R_\eta, R_n \in C(\R^2,\R^3)$ und $R_{r^2} \in C(\R^2,\R)$ mit
\begin{align*}
\eta(\tau) &= \sum_{j=1}^2\left(T_j(0) \tau_j + \frac{\kappa_j}{2} n_0 \tau_j^2 \right) + R_{\eta}(\tau),\\
n(\tau) &= n_0 - \sum_{j=1}^2 \kappa_j T_j (0) \tau_j +R_n(\tau), \\
r^2 &=|\eta(\tau)|^2 = |\tau|^2 + R_{r^2}(\tau).
\end{align*} 
Desweiteren gibt es Konstanten $C,\rho>0$ mit 
\begin{align*}
|R_{\eta}(\tau)| &\leq C ||\eta||_{3,\infty} |\tau|^3, \\
|R_n(\tau)| &\leq C ||\eta||_{3,\infty} |\tau|^2, \\
|R_{r^2}(\tau)| &\leq C ||\eta||_{3,\infty}^2 |\tau|^4,
\end{align*}
für $|\tau| \leq \rho$.
\end{Lemma}
\begin{proof}
Die Entwicklungen von $\eta$ und $n$ ergeben sich aus der Taylorformel aus Satz \ref{STaylor}, wobei wegen der Voraussetzungen an die Parametrisierung die gemischten Terme verschwinden (siehe Prop. 5.10.3 in \cite{Borceux}). 
Aus der Integraldarstellung des Restterms ergeben sich auch direkt die Abschätzungen für $R_\eta$ und $R_n$.
Die Darstellung von $r^2$ ergibt sich aus der Entwicklung von $\eta(\tau)$ und der Orthonormalität von $T_1, T_2$ und $n_0$. Dazu rechnen wir nach
\begin{align*}
r^2 =&|\eta(\tau)|^2 = \eta(\tau) \cdot \eta(\tau) \\
&=\left(\sum_{j=1}^2\left(T_j(0) \tau_j + \frac{\kappa_j}{2} n_0 \tau_j^2 \right) + R_{\eta}(\tau)\right) \cdot \left( \sum_{j=1}^2\left(T_j(0) \tau_j + \frac{\kappa_j}{2} n_0 \tau_j^2 \right) + R_{\eta}(\tau) \right) \\
&=\tau_1^2 + \tau_2^2 + \left( \frac{\kappa_1}{2}\tau_1^2 +\frac{\kappa_2}{2} \tau_2^2\right)^2 + 2 R_{\eta}(\tau) \cdot \left( \sum_{j=1}^2 \left(T_j(0)\tau_j+\frac{\kappa_j}{2}n_0 \tau_j^2 \right) \right) + |R_{\eta}(\tau)|^2.
\end{align*}
Da sich die Größen $T_j(0), \kappa_j$ und $n_0$ alle durch $||\eta||_{2,\infty}\leq ||\eta||_{3,\infty}$ abschätzen lassen folgt die Abschätzung mit der bereits bewiesenen für $R_\eta$.
\end{proof}

Zusätzlich zu den Entwicklungen aus Lemma \ref{Leta} werden wir den Transformationssatz anwenden um die relevanten Integrale zu vereinfachen. Dass die dabei verwendete Transformation ein (lokaler) Diffeomorphismus ist, sehen wir mit dem nächsten Lemma.


\begin{Lemma}\label{Lxi}
Sei $\eta \in  C_{\per}^4 (\R^2, \partial D)$.
Unter den obigen Voraussetzungen existieren offene Umgebungen $U_{\tau},U_{\xi}$ von $0 \in \R^2$, so dass die Abbildung \mbox{$\xi : U_{\tau} \to U_{\xi}, \tau \mapsto \xi(\tau) = \frac{|\eta(\tau)|}{|\tau|}\tau$} ein Diffeomorphismus ist. Hierbei wird $\xi(0) = 0$ gesetzt. \\
Ferner existieren Konstanten $c_2 \geq c_1 >0$ und $C>0$ mit 
\[
c_1|\tau| \leq |\xi| \leq c_2 |\tau|, \qquad \tau \in U_{\tau}, \quad \xi \in U_\xi
\]
und Funktionen $R_\xi \in C^1(U_{\tau}) , R_\tau \in C^1(U_{\xi})$ mit 
\begin{align*}
\xi_j(\tau) &= \tau_j (1+ R_\xi(\tau)), \qquad \tau \in U_{\tau}, \\
\tau_j(\xi) &= \xi_j (1+R_\tau(\xi)), \qquad \xi \in U_{\xi},
\end{align*}
wobei gilt
\begin{align*}
|R_\xi(\tau)| &\leq C ||\eta||_{4,\infty}|\tau|^2,\qquad \tau \in U_{\tau},   \\
|R_\tau(\xi)| &\leq C ||\eta||_{4,\infty} |\xi|^2,\qquad \xi \in U_{\xi}.
\end{align*}
\end{Lemma}


\begin{proof}
Aus dem Beweis von Lemma \ref{Leta} entnehmen wir, dass
\[
R_{r^2}(\tau) =\left( \frac{\kappa_1}{2}\tau_1^2 +\frac{\kappa_2}{2} \tau_2^2\right)^2 + 2 R_{\eta}(\tau) \cdot \left( \sum_{j=1}^2 \left(T_j(0)\tau_j+\frac{\kappa_j}{2}n_0 \tau_j^2 \right) \right) + |R_{\eta}(\tau)|^2
\]
Aus der Integraldarstellung von $R_\eta$ und der Produktregel ergibt sich die Differenzierbarkeit von $R_{r^2}$, sowie wegen 
\[
R_{\eta}(\tau)= \sum_{|\alpha|=3} \tau^\alpha\frac{3}{\alpha!}\int_0^1 (1-s)^3 D^\alpha \eta(s\tau) \dd s 
\] aus dem Satz von Taylor, die Abschätzung
\[
|R_{r^2}'(\tau)| \leq C ||\eta||^2_{4,\infty} |\tau|^3, \qquad |\tau| < \rho.
\]
An dieser Stelle nutzen wir auch, dass $\eta \in  C_{\per}^4 (\R^2, \partial D)$ und damit $R_{\eta} \in C^1(\R^2,\R^3)$
Daraus folgt auch die Differenzierbarkeit von \[
\tau \mapsto \frac{|\eta(\tau)|}{|\tau|}= \sqrt{1+\frac{R_{r^2}(\tau)}{\tau_1^2+\tau_2^2}} \quad \mathrm{ mit }\left( \frac{|\eta(\cdot)|}{|\cdot|}\right)'(\tau) = \frac{(\tau_1^2 +\tau_2^2)R_{r^2}'(\tau)-2R_{r^2}(\tau)\tau^T}{2(\tau_1^2 + \tau_2^2)^2 \sqrt{1+\frac{R_{r^2}(\tau)}{\tau_1^2+\tau_2^2}}}.
\]
Insbesondere folgt
\[
\left| \left( \frac{|\eta(\tau)|}{|\tau|}\right) ' \right| \leq C || \eta||_{4,\infty}^2 |\tau|, \qquad |\tau| < \rho.
\]
Nach der Produktregel gilt für die Jacobi-Matrix $\xi'$ von $\xi$ die Darstellung
\[
\xi'(\tau) = \frac{|\eta(\tau)|}{|\tau|}I_2 + \tau \left( \frac{|\eta(\tau)|}{|\tau|}\right) '
\]
mit der Einheitsmatrix $I_2 \in \R^{2 \times 2}$. Wegen der gefundenen Abschätzungen finden wir ein $\rho_2>0$, so dass $\det \xi'(\tau)>0$ für $|\tau| < \rho_2$. Nach dem lokalen Umkehrsatz existieren damit offene Umgebungen $U_{\tau}$ und $U_{\xi}$ von $0$, so dass $\xi : U_{\xi} \to U_{\tau}$ ein $C^1$-Diffeomorphismus ist. \\
Mit der Taylorentwicklung der Wurzelfunktion finden wir $R_\xi \in C^1(B_\tau)$ mit 
\[
\xi_j(\tau) = \tau_j (1+ R_\xi(\tau)), \qquad \tau \in U_{\tau},
\]
wobei $|R_\xi(\tau)| \leq C ||\eta||_{4,\infty} |\tau|^2$ gilt. Insbesondere folgt die Existenz von Konstanten $c_2 \geq c_1 >0$ mit
\[
c_1 |\tau| \leq |\xi(\tau)| \leq c_2 |\tau|, \qquad \tau \in U_{\tau}.
\] 
Für die Umkehrfunktion gilt damit
\[
\frac{1}{c_2}|\xi| \leq | \tau(\xi)| \leq \frac{1}{c_1}|\xi|, \qquad \xi \in U_{\xi}. 
\]
Mit der geometrischen Reihe sehen wir damit die Existenz von $R_\tau \in C^1(U_{\xi})$ mit 
\[
\tau_j(\xi) = \xi_j(1+R_\tau(\xi)), \qquad \xi \in U_{\xi},
\]
wobei $|R_\eta(\xi)| \leq C || \eta||_{4,\infty} |\xi|^2$ gilt.
\end{proof}

Das nächste Lemma zeigt, warum wir ohne Einschränkung annehmen können, dass die Zusatzvoraussetzungen erfüllt sind.
\begin{Lemma} \label{Llokaleparam}
Sei $\gamma : \R^2 \to \partial D$ die $2 \pi$-periodische Fortsetzung von  $\eta\in C^4(Q,\partial D)$.
Dann gibt es für alle $t \in Q=[-\pi,\pi]^2$ ein $\rho(t) \geq \rho_0>0$ und eine lokale Parametrisierung $ \eta_t : B_{\rho_t} \to \partial D$ mit der gleichen Glattheit wie $\eta$ und
\[
\eta_t(\tau) = \sum_{j=1}^2\left(T_j(0) \tau_j + \frac{\kappa_j}{2} n_0 \tau_j^2 \right) + R_{\eta_t}(\tau), \qquad \tau \in B_{\rho_t},
\]
sowie $\eta_t(0)= \gamma(t)$. Dabei sind $T_j(0)$ die Hauptkrümmungsrichtungen von $\partial D$ in $\gamma(t)$.
\end{Lemma} 




\begin{proof}
Wir stellen die Hauptkrümmungsrichtungen als Linearkombination der Spalten von $\gamma'(t)$ dar, also
\[
A_j(t) = P[\gamma'(t) n_0]^{-1} T_j(0), \qquad j=1,2.
\]
Hierbei ist P die Orthogonalprojektion von $\R^3$ auf $\R^2$ unter Weglassung der 3.Koordinate  $P(x_1,x_2,x_3)^T = (x_1,x_2)^T$. Da $T_j(0)$ die Tangentialvektoren von $\gamma$ sind, ist diese 3.Koordinate hier immer 0. \\
Wir definieren die affine Abbildung 
\[
\zeta_t(\tau):=t+A_1(t)\tau_1 +A_2(t)\tau_2, \qquad \tau \in \R^2,
\]
und $\eta_t(\tau):=\gamma(\zeta_t(\tau))$ für $\tau \in \R^2$. Wie man einfach nachrechnet ist $\partial_j \eta_t (0) =T_j(0)$. Wir wählen nun $B_{\rho(t)}$ als größten Kreis mit Mittelpunkt in $0$ und Radius $\rho (t)$, so dass $\overline{\zeta_{t}(B_{\rho(t)})} \subset t + (-\pi,\pi)^2$. Da $A_j(t)$ stetig von $t$ abhängt, ist damit auch $t \mapsto \rho(t)$ stetig. Da $\gamma$ $2\pi$-periodisch ist, nimmt $t \mapsto \rho(t)$ damit sein Minimum $\rho_0>0$ auf $Q$ an. Insgesamt erhalten wir also die gewünschte Darstellung von $\eta_t$. 
\end{proof}
%Damit erhalten wir genauso wie in Lemma \ref{Leta} die Darstellungen für $\eta, n$ und $r^2$, allerdings nur für die lokalen Parametrisierungen $\eta_t$. Allerdings lässt sich $Q$ wegen der Minimalgröße $\rho_0$ der Umgebungen mit endlich vielen dieser Umgebungen überdecken, wobei in jeder die Abschätzungen aus Lemma \ref{Leta} gelten.
\begin{Bemerkung} \label{Bmindestg}
Um die Koordinatentransformation $\tau \to \xi$ aus Lemma \ref{Lxi} ohne die zusätzlichen Voraussetzungen aus Abschnitt \ref{ParametrisierungVoraussetzungen} nutzen zu können, müssen wir allerdings noch einsehen, dass die Umgebungen $U_{\tau}$ und $U_{\xi}$ unabhängig von $t$ gewählt werden können. \\
Für die Existenz von $\xi^{-1}$ im Beweis von Lemma \ref{Lxi} nutzen wir dafür statt des Umkehrsatzes die Bijektivität von $|\tau| \mapsto |\xi|$ für feste Richtung $\hat{\tau}=\tau/|\tau|\in \mathbb{S}^1$, da $\tau \to \xi$ richtungserhaltend ist. 
Dafür betrachtet man die Ableitung 
\[
\frac{\dd |\xi(\rho \hat{\tau})|}{\dd \rho}= \frac{\dd}{\dd \rho} \sqrt{\rho^2 +R_{r^2}(\rho \hat{\tau})}, \qquad \rho>0.
\]
Mit ähnlichen Argumenten wie im Beweis von Lemma \ref{Lxi} und den Abschätzungen für die Restglieder gleichmäßig in $t$, erhält man beispielsweise die Existenz eines $c>0$, so dass die obige Ableitung größer gleich $\frac{1}{2}$ ist für $ \rho \leq c$. Insgesamt folgt wieder die Bijektivität von $\xi$ auf $B_c$ wobei $c$ unabhängig von $t$ ist.
Damit können wir auch ohne zusätzliche Voraussetzungen an $\eta$ die Lemmata aus Abschnitt \ref{ParametrisierungVoraussetzungen} benutzen. Um die Notation nicht unnötig zu erschweren verzichten wir darauf die entsprechenden Restfunktionen $R_\eta, R_n, R_{r^2}, R_\tau$ und $R_\xi$ für unterschiedliche $t$ zu unterscheiden.
% und nehmen ohne Einschränkung an, dass die Bijektivität von $\xi$ auf ganz $Q$ gilt, statt den Integrationsbereich in endlich viele Kugeln zu zerlegen. \\
\end{Bemerkung}


\subsection{Fehleranteil ohne Singularität} \label{chaepsohnesing}
Um den lokalen Fehler $\varepsilon = (A-A_{j,\delta}) \varphi(t)$ für festes $t \in Q$ zu bestimmen, bietet es sich an, den Integrationsbereich der Integraloperatoren zu zerlegen. Wir nutzen nun die Stetigkeit der Kernfunktionen
$K_{j,\delta}$ und $K$. Wählen wir eine offene Menge $\widetilde S$, die den singulären Streifen $S=\{(\tau,\tau)| \tau \in Q \}$ enthält, dann ist $Q \backslash \widetilde S$ kompakt und die Kernfunktionen sind auf dieser Menge beschränkt.
Damit kann der lokale Fehler auf diesem Teil wie bei Integraloperatoren mit stetigem Kern gleichmäßig für alle $t \in Q$ (vergleiche Beweis von Satz \ref{SbeschrIntop}) abgeschätzt werden.
Dies werden wir im Folgenden tun.
Zunächst nutzen wir Satz \ref{k1ddef} und erhalten
\begin{align*}
\varepsilon &= (A_{1,\delta} - A)(\varphi)(t) \\
&= \int_{Q} \left[K_{1,\delta}(x,\eta(\tau))- K(x,\eta(\tau))\right]\varphi(\tau) |\partial_1 \eta (\tau) \times \partial_2 \eta(\tau)| \dd \tau \\
& \overset{\eqref{k1ddefeq}}{=} \int_{Q} n(\eta(\tau)) \cdot \grad_y \left[ \Phi (x,\eta(\tau)) \erf \left(\frac{|x-\eta(\tau)|}{\delta}\right) - \Phi (x,\eta(\tau)) \right]\varphi(\tau) |\partial_1 \eta (\tau) \times \partial_2 \eta(\tau)| \dd \tau\\
& =\int_{Q} -n(\eta(\tau)) \cdot \grad_y \left[ \erfc \left(\frac{|x-\eta(\tau)|}{\delta}\right)  \Phi (x,\eta(\tau)) \right]\varphi(\tau) |\partial_1 \eta (\tau) \times \partial_2 \eta(\tau)| \dd \tau \\
&=: I_1(Q),
\end{align*}
bzw. für die zweite Glättungsfunktion
\begin{align*}
\varepsilon &= (A_{2,\delta} - A)(\varphi)(t) \\
&= \int_{Q} \left[K_{2,\delta}(x,\eta(\tau))- K(x,\eta(\tau))\right]\varphi(\tau) |\partial_1 \eta (\tau) \times \partial_2 \eta(\tau)| \dd \tau \\
& =\int_{Q} -n(\eta(\tau)) \cdot \grad_y \left[ f \left(\frac{|x-\eta(\tau)|}{\delta}\right)  \Phi (x,\eta(\tau)) \right]\varphi(\tau) |\partial_1 \eta (\tau) \times \partial_2 \eta(\tau)| \dd \tau \\
&=: I_2(Q),
\end{align*}
wobei 
\[
f(x)=
1- \left(\erf(x)+\frac{2}{3\sqrt{\pi}}x e^{-x^2}\right) , \quad x \in \R_{\geq 0}.
\]
Wir wählen nun eine Kugel $B_d=B_d(t)$ um $t$ mit Radius $d>0$ und zerlegen den Integrationsbereich $Q= Q \backslash B_d \cup B_d$. Dabei benutzen wir die abkürzende Schreibweise
\begin{align*}
\varepsilon_1&:=I_1(Q \backslash B_d) \\
\varepsilon_2&:=I_2(Q \backslash B_d).
\end{align*}
für die Fehleranteile ohne Singularität, bzw.
\begin{align*}
\widetilde \varepsilon_1&:=I_1(B_d) \\
\widetilde \varepsilon_2&:=I_2(B_d).
\end{align*}
für die Fehleranteile mit Singularität. Das heißt der gesamte lokale Fehler ist $\varepsilon =\varepsilon_j + \widetilde \varepsilon_j$ für $j=1$ oder $j=2$.
Die nächsten beiden Lemmata zeigen, dass die beiden Faktoren $\erfc\left(\frac{|x-\eta(\tau)|}{\delta}\right)$ und $f\left(\frac{|x-\eta(\tau)|}{\delta}\right)$ schnell fallen für $\delta \to 0$.
\begin{Lemma} \label{LErfOrdnung}
Sei  $r \in \R^+$, $0<\delta \leq r$  und $z \in \R_{\geq r}$. Dann gilt
\[
\erfc \left(\frac{z}{\delta} \right) \leq C \delta^5
\]
\end{Lemma}
mit einer von $z$ unabhängigen Konstante $C>0$.
\begin{proof}
Wir betrachten die Funktion $z \mapsto \erfc(\frac{z}{\delta})$. Es gilt für $z \in \R_{\geq r}$ 
\begin{align*}
\erfc \left(\frac{z}{\delta} \right)&= 1- \erf \left(\frac{z}{\delta}\right) \\
&= \frac{2}{\sqrt{\pi}}\int_{\frac{x}{\delta}}^\infty e^{-s^2} \dd s \\
&\leq \frac{2}{\sqrt{\pi}}\int_{\frac{r}{\delta}}^\infty e^{-s^2} \dd s \\
& \leq \frac{2}{\sqrt{\pi}}\int_{\frac{r}{\delta}}^\infty e^{-s} \dd s \\
&= \frac{2}{\sqrt{\pi}} \left(\lim_{s \to \infty}-e^{-s} + e^{-\frac{r}{\delta}} \right) \\
&= \frac{2}{\sqrt{\pi}} e^{-\frac{r}{\delta}} \\
&=\frac{2}{\sqrt{\pi}} \frac{1}{e^{\frac{r}{\delta}}} \\
&=: g(\delta)
\end{align*}
Wir betrachten nun die Abbildung $\delta \mapsto g(\delta)$ und zeigen, dass
\begin{align*}
\frac{g(\delta)}{\delta^5} &\leq C
\end{align*}
für alle $0<\delta\leq r$. Wir setzen $y=1/\delta$ und haben damit
\begin{align*}
\lim_{\delta \to 0} \frac{g(\delta)}{\delta^5} &= \lim_{y \to \infty} y^5 e^{-y r} \\
&= 0,
\end{align*}
nach Analysis 1 (z.B über den Satz von L'Hospital). Insbesondere ist $\frac{g(\delta)}{\delta^5}$ damit für $0<\delta \leq r$ beschränkt und wir erhalten
\[
g(\delta) \leq C \delta^5.
\]
\end{proof}
\begin{Lemma} \label{LErf2Ordnung}
Sei  $r \in \R^+$, $0<\delta \leq r_1$ und
\[ f(z)=
1- \left(\erf(z)+\frac{2}{3\sqrt{\pi}}z e^{-z^2}\right) , \quad z \in \R_{\geq 0}
\]
Dann gilt für $r_1 \leq z \leq r_2$
\[
f \left(\frac{z}{\delta} \right) \leq C \delta^5
\]
mit einer von $z$ unabhängigen Konstante $C>0$.
\end{Lemma}
\begin{proof}
Es gilt für $z \in \R_{\geq 0}$
\begin{align*}
f \left(\frac{z}{\delta} \right) 
&= \erfc \left(\frac{z}{\delta} \right) - \frac{2}{3\sqrt{\pi}}\frac{z}{\delta} e^{-\left(\frac{z}{\delta}\right)^2}
\end{align*}
Wir wählen $z \in \R$ mit $r_1 \leq x \leq r_2$ fest und betrachten die Funktion $g(\delta):=\frac{2}{3\sqrt{\pi}}\frac{z}{\delta} e^{-\left(\frac{z}{\delta}\right)^2}$ für $0 < \delta \leq r$. Dann ist  wegen $z/\delta \geq 1$
\begin{align*}
g(\delta) &\leq \frac{2}{3\sqrt{\pi}}\frac{z}{\delta} e^{-\frac{z}{\delta}} \\
& \leq \frac{2}{3\sqrt{\pi}}\frac{r_2}{\delta} e^{-\frac{r_1}{\delta}} \\
&=: \tilde g(\delta)
\end{align*}
und wir folgern, da $e^y$ schneller steigt als jedes Polynom in $y$ für $y \to \infty$ (Analysis 1),
\begin{align*}
\lim_{\delta \to 0} \frac{\tilde g(\delta)}{\delta^5} &= \lim_{\delta \to 0} \frac{2}{3\sqrt{\pi}}\frac{r_2}{\delta^6} e^{-\frac{r_1}{\delta}} \\
&= 0.
\end{align*}
Insbesondere ist  $\frac{\tilde g(\delta)}{\delta^5} \leq C $ und damit und wir erhalten
\[
g(\delta) \leq \tilde g(\delta) \leq C \delta^5
\] 
für $0 < \delta \leq r_1$, mit einer von $x$ unabhängigen Konstante $C>0$. Zusammen mit Lemma \ref{LErfOrdnung} folgt damit die Behauptung.
\end{proof}

\begin{Lemma} \label{LepsohneSing}
Es gibt ein $\delta_0>0$, so dass für die Fehleranteile $\varepsilon_j, j=1,2$ ohne Singularität gilt
\begin{align*}
\varepsilon_j \leq C ||\varphi||_\infty \delta^5,
\end{align*}
für alle $\delta \leq \delta_0$. 
\end{Lemma}
\begin{proof}
Nach Definition ist
\begin{align*}
\varepsilon_1& =\int_{Q \backslash B_d} -n(\eta(\tau)) \cdot \grad_y \left[ \erfc \left(\frac{|x-\eta(\tau)|}{\delta}\right)  \Phi (x,\eta(\tau)) \right]\varphi(\tau) |\partial_1 \eta (\tau) \times \partial_2 \eta(\tau)| \dd \tau \\
\end{align*}
Da $\eta$ stetig ist, existiert $\delta_0 := \min_{\tau \in Q  \backslash B_d} |x- \eta(\tau)|>0$ . Wir nutzen Lemma \ref{LErfOrdnung} und schätzen ab
\begin{align*}
|\varepsilon_1| 
&\leq \int_{Q \backslash B_d} \left|-n(\eta(\tau)) \cdot \grad_y \left[ \erfc \left(\frac{|x-\eta(\tau)|}{\delta}\right)  \Phi (x,\eta(\tau)) \right]\varphi(\tau) |\partial_1 \eta (\tau) \times \partial_2 \eta(\tau)| \right| \dd \tau \\
& \leq C \delta^5 \int_{Q \backslash B_d}| K(x,\eta(\tau))||\varphi(\tau)| |\partial_1 \eta (\tau) \times \partial_2 \eta(\tau)| \dd \tau \\
& \leq C ||\varphi||_{\infty} \delta^5
\end{align*}
für alle $\delta \leq \delta_0$ mit einer von $\delta$ unabhängigen Konstanten $C>0$. \\ 
Für die Abschätzung von $\varepsilon_2$ brauchen wir die Beschränktheit von $Q$ und definieren $r_2:=\max_{\tau \in Q  \backslash B_d} |x- \eta(\tau)|>0$. Damit können wir Lemma \ref{LErf2Ordnung} verwenden (mit $\delta_0=:r_1$ und $r_2$) und erhalten mit 
\[
f(z)=
1- \left(\erf(z)+\frac{2}{3\sqrt{\pi}}z e^{-z^2}\right) , \quad z \in \R_{\geq 0}
\]
die Abschätzung
\begin{align*}
|\varepsilon_2| 
&\leq \int_{Q \backslash B_d} \left|-n(\eta(\tau)) \cdot \grad_y \left[ f \left(\frac{|x-\eta(\tau)|}{\delta}\right)  \Phi (x,\eta(\tau)) \right]\varphi(\tau) |\partial_1 \eta (\tau) \times \partial_2 \eta(\tau)| \right| \dd \tau \\
& \leq C \delta^5 \int_{Q \backslash B_d}| K(x,\eta(\tau))||\varphi(\tau)| |\partial_1 \eta (\tau) \times \partial_2 \eta(\tau)| \dd \tau \\
& \leq C ||\varphi||_{\infty} \delta^5
\end{align*}
für $\delta \leq \delta_0$ mit einer von $\delta$ unabhängigen Konstanten $C>0$.
%\ref{SbeschrIntop} 
\end{proof}
Tatsächlich könnte man die Aussage von Lemma \ref{LepsohneSing} genauso für jedes Monom $\delta^k, k \in \N_0$ formulieren. Wir beschränken uns an dieser Stelle jedoch auf die Ordnung, die wir  im letzten Abschnitt des Kapitels auch für den Fehleranteil des Integrals um die Singularität zeigen werden.
\begin{Bemerkung} \label{BQZerlegung}
Wir wollen die Aussage von Lemma \ref{LepsohneSing} für eine von $x$ unabhängige Konstante $C>0$ beweisen, um eine Abschätzung für $||A_{j,\delta}-A||, j=1,2$ zu erhalten. Dafür ist es wichtig, die Abhängigkeit der Konstante $C=C(B_d(t))$ im Beweis genauer zu betrachten. Die Konstanten, die aus der Anwendung von Lemma \ref{LErfOrdnung} und \ref{LErf2Ordnung} stammen, hängen dabei nicht von $x$, sondern nur vom Radius $d$ der entfernten Kugel $B_d$ ab.
Somit bekommen wir, wenn wir für alle $t \in Q$ den gleichen Radius $d$ wählen, eine gleichmäßige Abschätzung der Glättungsfaktoren $\erfc$ und $f$ im Beweis. Wichtig ist also nur noch, den Integranden in 
\[
\int_{Q \backslash B_d}| K(x,\eta(\tau))||\varphi(\tau)||\partial_1 \eta (\tau) \times \partial_2 \eta(\tau)| \dd \tau 
\]
gleichmäßig zu beschränken. Wie zuvor festgestellt ist die Kernfunktion $K$ stetig. 
Definieren wir nun zum Streifen $S=\{(\tau,\tau)| \tau \in Q \}$ auf dem $K$ singulär ist, die Menge $\widetilde S = S+ B_d(0) \subset \R^2 \times \R^2$, wobei $B_d(0) \subset \R^4$ , so ist $(Q \times Q)\backslash \widetilde S$ kompakt. Damit nimmt $K$ als stetige Funktion sein Maximum auf $(Q \times Q) \backslash \widetilde S$ an und wir finden eine gleichmäßige Abschätzung in Lemma \ref{LepsohneSing}. 

Allerdings würden wir mit dieser Wahl von $\widetilde S$ bei der Abschätzung des Fehleranteils um die Singularität im Beweis von Satz \ref{Seps1} eine niedrigere Ordnung erhalten. Die wesentliche Idee dieses Beweises ist es, den Integranden zu entwickeln und in gerade und ungerade Terme zu zerlegen. Dann kann damit argumentiert werden, dass die Integrale ungerader Anteile verschwinden, wenn das Integrationsgebiet symmetrisch um 0 ist. Wir wollen $\widetilde S$ also so wählen, dass wir nach der notwendigen Transformation mit der Abbildung $\xi$ aus Lemma \ref{Lxi} einen um $0$ symmetrischen Integrationsbereich haben. 

Wir nehmen gemäß Lemma \ref{Llokaleparam} o.B.d.A. an, dass $t=0$ und dass $\eta$ die Voraussetzungen aus Abschnitt \ref{ParametrisierungVoraussetzungen} erfüllt. Außerdem nehmen wir an, dass $\eta \in C_{\per}^4(\R^2,\partial D)$. Dann liefert uns Lemma \ref{Lxi} die Existenz eines lokalen Diffeomorphismus $\xi : U_\tau \to U_\xi, \tau \mapsto \frac{|\eta_t(\tau)|}{|\tau|}\tau$ mit Umgebungen $U_\tau$ und $U_\xi$ um 0. Diese haben nach Bemerkung \ref{Bmindestg} eine von $t$ unabhängige Mindestgröße $d>0$. Damit ist $\xi: B_d(0) \to \xi_t(B_d(0))$ immer noch ein Diffeomorphismus, der den Abschätzungen aus Lemma \ref{Lxi} für entsprechende Funktionen $R_\xi$ und $R_\tau$ genügt. 
Zerlegen wir also für den lokalen Fehler den Integrationsbereich in $Q= Q \backslash \xi(B_d(0)) \cup \xi(B_d(0))$, so lässt sich Lemma \ref{LepsohneSing} für den Fehleranteil ohne Singularität $\varepsilon_j, j=1,2$, wegen

\begin{align*}
|\varepsilon_j|& \leq \int_{Q \backslash \xi(B_d(0))}\left| \left[K_{j,\delta}(\eta(t),\eta(\tau))- K(\eta(t),\eta(\tau))\right]\varphi(\tau) |\partial_1 \eta (\tau) \times \partial_2 \eta(\tau)| \right| \dd \tau \\
& \leq \int_{Q \backslash B_{c}(0)} \left| \left[K_{j,\delta}(\eta(t),\eta(\tau))- K(\eta(t),\eta(\tau))\right]\varphi(\tau) |\partial_1 \eta (\tau) \times \partial_2 \eta(\tau)| \right| \dd \tau \\
& \leq C ||\varphi||_\infty \delta^5
\end{align*}
 anwenden, für jede Kugel $B_{c}(0)\subset\xi(B_d(0))$. Mit ähnlichen Argumenten wie in Bemerkung \ref{Bmindestg} finden wir ein $c>0$, so dass für alle $t\in Q$ gilt $B_c(0) \subset \xi_t(B_d(0))$. Hierbei ist $\xi_t$ die jeweilige Abbildung, die Lemma \ref{Lxi} nach Koordinatenwechsel über Lemma \ref{Llokaleparam} liefert.  Damit können wir den Integranden nach Lemma \ref{LepsohneSing} gleichmäßig auf $(Q \times Q) \backslash \tilde S$ mit $\tilde S= S+ B_c(0)$ und $B_c(0)\subset \R^4$ abschätzen und im Folgenden nur noch den verbleibenden Fehleranteil 
 \begin{align*}
 \widetilde \varepsilon_j = \int_{\xi(B_d(0))} \left[K_{j,\delta}(\eta(t),\eta(\tau))- K(\eta(t),\eta(\tau))\right]\varphi(\tau) |\partial_1 \eta (\tau) \times \partial_2 \eta(\tau)| \dd \tau
 \end{align*}
 mit Singularität berechnen.
\end{Bemerkung}


\subsection{Fehleranteil mit Singularität}
Wir betrachten nun den nach Bemerkung \ref{BQZerlegung} noch fehlenden Fehleranteil $\widetilde \varepsilon_j, j=1,2$ mit Singularität, wobei wir von nun an $\widetilde d$ und $d$ wie in der Bemerkung wählen. Gemäß Lemma \ref{Llokaleparam} sei o.B.d.A $0=x= \eta(t) \in \partial D$ und $t=0$. Für den Fehleranteil bei der einfacheren Kernfunktion, die wir durch Glättung mit $s_1$ erhalten, hatten wir in Kapitel \ref{chaepsohnesing} gesehen, dass
\begin{align*}
\widetilde \varepsilon_1 &= \int_{\xi(B_d(0))} n(\eta(\tau)) \cdot \grad_y \left[ \left(\erf \left(\frac{|x-\eta(\tau)|}{\delta}\right) - 1\right) \Phi (x,\eta(\tau)) \right]\varphi(\tau) |\partial_1 \eta (\tau) \times \partial_2 \eta(\tau)| \dd \tau,
\end{align*}
wobei $\xi$ der Diffeomorphismus aus Lemma \ref{Lxi} ist.

\begin{Lemma}
Der lokale Fehleranteil ohne Singularität $\widetilde\varepsilon_1$ bezüglich $s_1$ die Darstellung 
\begin{equation} \label{eps1}
\widetilde\varepsilon_1= \frac{1}{4 \pi} \int_{\xi(B_d(0))}  \frac{- \eta(\tau)\cdot n(\eta(\tau)) }{r^3} h(r/\delta) \varphi(\tau) |\partial_1 \eta (\tau) \times \partial_2 \eta(\tau)| \dd \tau,
\end{equation}
wobei $h(s):= \erfc(s)+(2/\sqrt{\pi})s e ^{-s^2 }$ für $s \in \R_{\geq 0}$.
\end{Lemma}


\begin{proof}
Sei $\tau \in Q$ fest. Mit der Produktregel erhalten wir damit
\begin{align*}
\grad_y\left[ \left(\erf \left(\frac{|x-\eta(\tau)|}{\delta}\right) - 1\right) \Phi (x,\eta(\tau)) \right] &= - \grad_y\left(\frac{\erfc(\frac{r}{\delta})}{4 \pi r}\right) \\
&= - \frac{1}{4 \pi} \left(\erfc'(\frac{r}{\delta})\frac{-\eta(\tau)}{\delta r^2} + \erfc(\frac{r}{\delta})\frac{\eta(\tau)}{r^3} \right)
\end{align*} und 
\[
\erfc'(s)=-\erf'(s)=-(2/\sqrt{\pi})e^{-s^2} \fa s \in \R,
\] womit die Behauptung folgt.
\end{proof}

Sei von nun an 
\[
h(s):=\erfc(s)+(2/\sqrt{\pi})s e ^{-s^2 } \fa s\in \R
\] wie in obigem Lemma.
Die Abhängigkeit des Integranden in Gleichung \eqref{eps1} von $r$ lässt sich nun mithilfe des Koordinatenwechsels $\tau \mapsto \xi(\tau)=\xi$ aus Lemma \ref{Lxi} vereinfachen.
%Um den Koordinatenwechsel anwenden zu können müssen wir allerdings noch den Integrationsbereich in Gebiete zerlegen, auf denen $\xi$ ein Diffeomorphismus ist. An dieser Stelle zeigen wir, dass wir auch ohne die zusätzlichen Voraussetzungen an $\eta$ aus Abschnitt \ref{ParametrisierungVoraussetzungen} auskommen.
%
%
%\begin{Lemma}
%Sei $t \in Q=(-\pi,\pi]^2$ und $\gamma : \R^2 \to \partial D$ die $2 \pi$-periodische Fortsetzung von  $\eta\in C^4(Q,\partial D)$.
%Dann gibt es ein $\rho(t) \geq \rho_0>0$ und eine lokale Parametrisierung $ \eta_t : B_{\rho_t} \to \partial D$ mit der gleichen Glattheit wie $\eta$ und
%\[
%\eta_t(\tau) = \sum_{j=1}^2\left(T_j(0) \tau_j + \frac{\kappa_j}{2} n_0 \tau_j^2 \right) + R_{\eta_t}(\tau), \qquad \tau \in B_{\rho_t},
%\]
%sowie $\eta_t(0)= \gamma(t)$. Dabei sind $T_j(0)$ die Hauptkrümmungsrichtungen von $\partial D$ in $\gamma(t)$.
%\end{Lemma} 
%
%
%
%
%\begin{proof}
%Wir stellen die Hauptkrümmungsrichtungen als Linearkombination der Spalten von $\gamma'(t)$ dar, also
%\[
%A_j(t) = P[\gamma'(t) n_0]^{-1} T_j(0), \qquad j=1,2.
%\]
%Hierbei ist P die Orthogonalprojektion $\R^2$ $P(x_1,x_2,x_3)^T = (x_1,x_2)^T$ von $\R^3$ auf $\R^2$. Da $T_j(0)$ die Tangentialvektoren von $\gamma$ sind, ist diese 3.Koordinate hier immer 0. \\
%Wir definieren die affine Abbildung 
%\[
%\zeta_t(\tau):=t+A_1(t)\tau_1 +A_2(t)\tau_2, \qquad \tau \in \R^2,
%\]
%und $\eta_t(\tau):=\gamma(\zeta_t(\tau))$ für $\tau \in \R^2$. Wie man einfach nachrechnet ist $\partial_j \eta_t (0) =T_j(0)$. Wir wählen nun $B_{\rho_t}$ als größten Kreis mit Mittelpunkt in $0$ und Radius $\rho (t)$, so dass $\overline{\zeta_{t}(B_{\rho(t)})} \subset t + (-\pi,\pi)^2$. Da $A_j(t)$ stetig von $t$ abhängt, ist damit auch $t \mapsto \rho(t)$ stetig. Da $\gamma$ $2\pi$-periodisch ist, nimmt $t \mapsto \rho(t)$ damit sein Minimum $\rho_0$ auf $Q$ an. Insgesamt erhalten wir also die gewünschte Darstellung von $\eta_t$. 
%\end{proof}
%Damit erhalten wir genauso wie in Lemma \ref{Leta} die Darstellungen für $\eta, n$ und $r^2$, allerdings nur für die lokalen Parametrisierungen $\eta_t$. Allerdings lässt sich $Q$ wegen der Minimalgröße $\rho_0$ der Umgebungen mit endlich vielen dieser Umgebungen überdecken, wobei in jeder die Abschätzungen aus Lemma \ref{Leta} gelten.
%\begin{Bemerkung}
%Um die Koordinatentransformation $\tau \to \xi$ aus Lemma \ref{Lxi} ohne die zusätzlichen Voraussetzungen aus Abschnitt \ref{ParametrisierungVoraussetzungen} nutzen zu können, müssen wir allerdings noch einsehen, dass die Umgebungen $U_{\tau}$ und $U_{\xi}$ unabhängig von $t$ gewählt werden können. \\
%Für die Existenz von $\xi^{-1}$ im Beweis von Lemma \ref{Lxi} nutzen wir dafür statt des Umkehrsatzes die Bijektivität von $|\tau| \mapsto |\xi|$ für feste Richtung $\hat{\tau}=\tau/|\tau|\in \mathbb{S}^1$, da $\tau \to \xi$ richtungserhaltend ist. 
%Dafür betrachtet man die Ableitung 
%\[
%\frac{\dd |\xi(\rho \hat{\tau})|}{\dd \rho}= \frac{\dd}{\dd \rho} \sqrt{\rho^2 +R_{r^2}(\rho \hat{\tau})}, \qquad \rho>0.
%\]
%Mit ähnlichen Argumenten wie im Beweis von Lemma \ref{Lxi} und den Abschätzungen für die Restglieder gleichmäßig in $t$, erhält man beispielsweise die Existenz eines $c>0$, so dass die obige Ableitung größer gleich $\frac{1}{2}$ ist für $ \rho \leq c$. Insgesamt folgt wieder die Bijektivität von $\xi$ auf $B_c$ wobei $c$ unabhängig von $t$ ist.
%Damit können wir auch ohne zusätzliche Voraussetzungen an $\eta$ die Lemmata aus Abschnitt \ref{ParametrisierungVoraussetzungen} benutzen. Um die Notation nicht unnötig zu erschweren verzichten wir an dieser Stelle darauf die entsprechenden Restfunktionen $R_\eta, R_n, R_{r^2}, R_\tau$ und $R_\xi$ für unterschiedliche Umgebungen zu unterscheiden und nehmen ohne Einschränkung an, dass die Bijektivität von $\xi$ auf ganz $Q$ gilt, statt den Integrationsbereich in endlich viele Kugeln zu zerlegen. \\
%\end{Bemerkung}

%\subsection{Voraussetzungen an die Parametrisierung}
%Da wir die Taylorentwicklung aller von $\tau$ abhängigen Terme brauchen, nehmen wir vereinfachend an, dass $\eta(0)=0$ und $x = 0$, wobei $n_0=n(\eta(0))$ die Einheitsnormale auf $\partial D$ bei $\eta(0)$ ist. Für den allgemeinen Fall kann für beliebiges fest gewähltes $x \in \partial D$ eine Verschiebung des Koordinatensystems vorgenommen werden und eine passende Parametrisierung $\eta$ gewählt werden, sodass $x=0$.
%\\
%Außerdem fordern wir, dass die Tangenten der Parametrisierung $T_j(\tau):=\partial_j \eta(\tau)$ in $0$ senkrecht zueinander stehen und normiert sind, d.h. der metrische Tensor $g_{ij}(\tau) := T_i(\tau) \cdot T_j(\tau),\, j=1,2$ ist bei $\tau=0$ die Identität. Weiterhin fordern wir, dass $\partial g_{ij}/\partial \tau_k (0)=0, \, i,j,k=1,2$ und $T_1$ und $T_2$ in Richtung der Hauptkrümmungen zeigen.  \\
%\begin{Lemma} \label{Leta}
%\todo[inline]{Kurzform übernehmen}
%Unter obigen Voraussetzungen und $\eta\in C^3(Q,D)$ ergibt sich für die Parametrisierung die Taylorentwicklung
%\begin{align*}
%\eta(\tau) &= T_1 (0) \tau_1 +T_2(0) \tau_2 + \frac{1}{2} \kappa_1 n_0 \tau_1^2+ \frac{1}{2} \kappa_2 n_0 \tau_2^2 +  R_\eta(\tau)
%\end{align*}
%mit $R_\eta(\tau)= \sum_{|\alpha|=3} R_\alpha(\tau)\tau^\alpha \in O(|\tau|^3)$ und $|R_\alpha(\tau)|\leq \frac{1}{\alpha!} \max_{|\beta|=|\alpha|} ||D^\beta \eta||_\infty $ und für die Normale
%\begin{align*}
%n(\tau) = n_0 - \kappa_1 T_1(0) \tau_1 - \kappa_2 T_2(0) \tau_2 + R_n (\tau)
%\end{align*}
%mit $R_n (\tau)=\sum_{|\alpha|=2} \widetilde R_\alpha(\tau)\tau^\alpha \in O(|\tau|^2)$ und $|\widetilde R_\alpha(\tau)|\leq \frac{1}{\alpha!} \max_{|\beta|=|\alpha|} ||D^\beta n||_\infty $. \\ Für $r=|x- \eta(\tau)|=|\eta(\tau)|$ ergibt sich die Darstellung 
%\begin{equation} \label{r2Ent}
%r^2 =|\eta(\tau)|^2  = |\tau|^2 + R_{r^2}(\tau)
%\end{equation}
%wobei $R_{r^2}(\tau):=\frac{1}{4}(\kappa_1 \tau_1^2+ \kappa_2 \tau_2^2)^2 + \eta(\tau) \cdot R_\eta(\tau) \in O(|\tau|^4)$.
%\end{Lemma}
%
%\begin{proof}
%Die Entwicklungen von $\eta$ und $n$ ergeben sich aus der Taylorformel, wobei wegen der Voraussetzungen an die Parametrisierung die gemischten Terme verschwinden (siehe Prop. 5.10.3 in \cite{Borceux})  Die Darstellung von $r^2$ ergibt sich aus der Entwicklung von $\eta(\tau)$ und der Orthogonalität von $T_1, T_2$ und $n_0$. Dazu rechnen wir nach
%\begin{align*}
%r^2 =&|\eta(\tau)|^2 = \eta(\tau) \cdot \eta(\tau) \\
%=& T_1 (0) \tau_1 \cdot \left(T_1 (0) \tau_1 +T_2(0) \tau_2 + \frac{1}{2} \kappa_1 n_0 \tau_1^2+ \frac{1}{2} \kappa_2 n_0 \tau_2^2 +  R_\eta(\tau) \right) \\
%&+T_2(0) \tau_2 \cdot \left(T_1 (0) \tau_1 +T_2(0) \tau_2 + \frac{1}{2} \kappa_1 n_0 \tau_1^2+ \frac{1}{2} \kappa_2 n_0 \tau_2^2 +  R_\eta(\tau) \right) \\
%&+\frac{1}{2}n_0( \kappa_1 \tau_1^2+ \kappa_2 \tau_2^2) \cdot \left(T_1 (0) \tau_1 +T_2(0) \tau_2 + \frac{1}{2} \kappa_1 n_0 \tau_1^2+ \frac{1}{2} \kappa_2 n_0 \tau_2^2 +  R_\eta(\tau) \right) \\
%&+R_\eta(\tau) \cdot \left(T_1 (0) \tau_1 +T_2(0) \tau_2 + \frac{1}{2} \kappa_1 n_0 \tau_1^2+ \frac{1}{2} \kappa_2 n_0 \tau_2^2 +  R_\eta(\tau) \right) \\
%=& \tau_1 ^2 + \tau_2^2 + \frac{1}{4}(\kappa_1 \tau_1^2+ \kappa_2 \tau_2^2)^2 + \eta(\tau) \cdot R_\eta(\tau).
%\end{align*}
%
%\end{proof}
 


%\begin{Lemma} \label{Lxi}
%\todo[inline]{Kurzform übernehmen}
%Sei $\xi=\xi(\tau) \in \R^2$ gegeben durch 
%\[
%\xi_j:=\tau_j \frac{r}{|\tau|} \mathrm{,\, d.h. \quad} |\xi|^2 = r^2 \mathrm{\quad und \quad } \xi_j /|\xi| = \tau_j/|\tau|.
%\]
%Damit ergeben sich die Entwicklungen
%\[
%|\xi| = |\tau| (1 + R_{|\tau|}(\tau))
%\] 
%mit 
%\[
%R_{|\tau|}(\tau) :=\frac{1}{2}R_{|\xi|}(\tau)+ R_{\sqrt{\cdot}}(R_{|\xi|}(\tau)) \in O(|\tau|^2),
%\]
%wobei $R_{|\xi|}(\tau):=R_{r^2}(\tau)/|\tau|^2 \in O(|\tau|^2)$ und $R_{\sqrt{\cdot}}$ der Restterm der Wurzelentwicklung ist,\\ d.h. $\sqrt{1+x}=1+x/2 + R_{\sqrt{\cdot}}(x)$ mit $R_{\sqrt{\cdot}}(x) \in O(x^2)$. \\ Außerdem lässt sich $\tau_j$ darstellen als 
%\[
%\tau_j = R_\xi(\tau)\xi_j 
%\]
%mit $R_\xi(\tau):= \frac{1}{1+ R_{|\tau|}(\tau)}$. 
%Also 
%\[
%\tau_j = (1+R_\tau(\tau))\xi_j 
%\]
%mit $R_\tau(\tau):=-R_{|\tau|}(\tau)+ R_{1/\cdot}(R_{|\tau|}(\tau)) \in O(|\tau|^2)$, wobei $R_{1/\cdot}$ der Restterm der Entwicklung 
%\[
%\frac{1}{1+x}= 1- x + R_{1/\cdot}(x)
%\] mit  $R_{1/\cdot}(x)\in O(x^2)$ ist.
%\end{Lemma}
%\begin{proof}
%Es gilt 
%\[
% |\xi|^2 = \xi_1^2+\xi_2^2=(\tau_1^2+\tau_2^2)\frac{r^2}{|\tau|^2}=r^2
%\]
%und damit $|\xi|=r$ also nach Def.
%\[
%\xi_j/|\xi| = \tau_j /|\tau|.
%\]
%Für die Entwicklung von $|\xi|$ nutzen wir \eqref{r2Ent} und die Taylorentwicklung der Wurzelfunktion $\sqrt{1+x}= 1+ x/2 + R_{\sqrt{\cdot}}(x)$ mit $R_{\sqrt{\cdot}}(x) \in O(x^2)$ und erhalten
%\begin{align*}
%|\xi|=\sqrt{r^2}&= \sqrt{|\tau|^2 + R_{r^2}(\tau)} \\
%&=|\tau|\sqrt{1+\frac{R_{r^2}(\tau)}{|\tau|^2}} \\
%&=|\tau|(1+\frac{1}{2}R_{|\xi|}(\tau)+ R_{\sqrt{\cdot}}(R_{|\xi|}(\tau))
%\end{align*}
%mit $R_{|\xi|}(\tau)=R_{r^2}(\tau)/|\tau|^2$. Weiterhin ist nach Definiton von $\xi_j$
%\begin{align*}
%\xi_j &= \tau_j \frac{r}{|\tau|} \\
%&= \tau_j \frac{|\xi|}{|\tau|} \\
%&= \tau_j \frac{|\tau| (1 + R_{|\tau|}(\tau))}{|\tau|} \\
%&= \tau_j (1 + R_{|\tau|}(\tau)).
%\end{align*}
%\end{proof}

%\begin{Lemma} \label{Entww}
%Der lokale Fehler $\varepsilon$ besitzt die Darstellung
%\begin{equation} \label{eps2}
%\varepsilon = \frac{1}{4 \pi} \int_{Q}  \frac{h(|\xi|/\delta)}{|\xi|^3} w(\xi)\dd \xi,
%\end{equation}
%mit nichtradialem Anteil $w(\tau)= -\eta(\tau) \cdot n(\tau) \varphi(\tau)|\partial_1 \eta (\tau) \times \partial_2 \eta(\tau)||\operatorname{det}\frac{\partial\tau}{\partial\xi}(\tau)|$. \\ Der nichtradiale Anteil $w$ besitzt bezüglich $\tau=\xi^{-1}(\xi)$ die Entwicklung
%\begin{align*} 
%w(\tau)=&\frac{1}{2}\left(\kappa_1 \tau_1^2 + \kappa_2 \tau_2^2 +R_{\eta n}(\tau)\right)
%\sum_{i=1}^2\left(\varphi_i \tau_i + \frac{1}{2}(\varphi_{i1}\tau_i\tau_1 + \varphi_{i2}\tau_i\tau_2) + \frac{1}{2}R_\varphi(\tau)\right)R_1(\tau)
%\end{align*}
%wobei $\varphi_i=\frac{\partial \varphi}{\partial \tau_i}(0)$, $\varphi_{ij}=\frac{\partial^2 \varphi}{\partial \tau_i \partial \tau_j}(0)$ und $R_\varphi(\tau)$ der Taylorrestterm der Entwicklung von $\varphi$ bis zum Grad 2 ist, d.h.
%\[ 
%\varphi(\tau) = \sum_{i=1}^2\left(\varphi_i \tau_i + \frac{1}{2}(\varphi_{i1}\tau_i\tau_1 + \varphi_{i2}\tau_i\tau_2) + \frac{1}{2}R_\varphi(\tau)\right)\] 
%mit $R_\varphi(\tau)=\sum_{|\alpha|=3} R_{\varphi,\alpha}(\tau)\tau^\alpha $ und 
%$|R_{\varphi,\alpha}(\tau)|\leq \frac{1}{\alpha!} \max_{|\beta|=|\alpha|} ||D^\beta \varphi||_\infty$ und 
%\begin{align*}
%R_1(\tau)&=\left(1+ R_{T_1 \times T_2}(\tau) \right)\left(1+ R_{\det}(\tau)\right) \\
% &= 1 + R_2(\tau)
%\end{align*}
%mit $\{R_{T_1 \times T_2}(\tau),R_{\det}(\tau),R_2(\tau)\}\subset O(|\tau|^2)$
%und $R_{\eta n}(\tau)= 2 R_\eta (\tau) \cdot n(\tau) \in O(|\tau|^3)$
%\end{Lemma}



\begin{Lemma} \label{Entww}
Sei $\varphi \in C_{\per}^3(\R^2)$. Der lokale Fehler $\varepsilon$ bezüglich $s_1$ besitzt die Darstellung
\begin{equation} \label{eps2}
\widetilde \varepsilon_1 = \frac{1}{4 \pi} \int_{B_d(0)}  \frac{h(|\xi|/\delta)}{|\xi|^3} w(\xi)\dd \xi,
\end{equation}
mit nichtradialem Anteil $w(\tau(\xi))= -\eta(\tau) \cdot n(\tau) \varphi(\tau)|\partial_1 \eta (\tau) \times \partial_2 \eta(\tau)||\operatorname{det}\frac{\partial\tau}{\partial\xi}(\tau)|$ für $\tau= \tau(\xi), \xi \in B_d(0)$. \\ Der nichtradiale Anteil $w$ besitzt bezüglich $\tau=\xi^{-1}(\xi)$ für $\xi \in B_d(0)$ die Entwicklung
\begin{align*} 
w(\tau)=&\frac{1}{2} \left(\sum_{j=1}^2 \kappa_j \tau_j^2 +R_{\eta n}(\tau) \right)
\sum_{i=1}^2\left(\varphi_i \tau_i + \frac{1}{2}\left(\sum_{j=1}^2\varphi_{ij}\tau_i\tau_j \right) + \frac{1}{2}R_\varphi(\tau)\right)(1+ R_1(\tau)),
\end{align*}
wobei $\varphi_i=\frac{\partial \varphi}{\partial \tau_i}(0)$, $\varphi_{ij}=\frac{\partial^2 \varphi}{\partial \tau_i \partial \tau_j}(0)$, $R_\varphi(\tau) \in C(\R^2), R_1(\tau) \in C(\R^2), R_{\eta n} \in C(\R^2)$ 
und 
\begin{align*}
|R_\varphi(\tau)| &\leq C ||\varphi||_{3,\infty} |\tau|^3\\
|R_{\eta n}(\tau)| & \leq C |\tau|^3 \\
|R_1(\tau)| & \leq C |\tau|^2
\end{align*}
für alle $\tau \in \xi(B_d(0))$.
\end{Lemma}



\begin{proof}
Nach Anwendung des Transformationssatzes auf \eqref{eps1} mit $\tau \mapsto \xi$ wie in Lemma \ref{Lxi} ergibt sich unmittelbar die Darstellung \eqref{eps2} mit $w$ wie oben. \\
Für die Entwicklung von $w$ nutzen wir die Entwicklungen aus Lemma \ref{Leta} und Lemma \ref{Lxi}.

Nach Lemma \ref{Leta} gilt
%\eta(\tau) &= \sum_{j=1}^2\left(T_j(0) \tau_j + \frac{\kappa_j}{2} n_0 \tau_j^2 \right) + R_{\eta}(\tau),\\
%n(\tau) &= n_0 - \sum_{j=1}^2 \kappa_j T_j (0) \tau_j +R_n(\tau),
\begin{align*}
- \eta(\tau) \cdot n(\tau) &= - \left(\sum_{j=1}^2\left(T_j(0) \tau_j + \frac{\kappa_j}{2} n_0 \tau_j^2 \right) + R_{\eta}(\tau)\right) \\
&\quad \cdot \left( n_0 - \sum_{j=1}^2 \kappa_j T_j (0) \tau_j +R_n(\tau) \right) \\
&=-(T_1 (0) \tau_1 +T_2(0) \tau_2 + \frac{1}{2} \kappa_1 n_0 \tau_1^2+ \frac{1}{2} \kappa_2 n_0 \tau_2^2 +  R_\eta(\tau))\\ 
&\quad\cdot ( n_0 - \kappa_1 T_1(0) \tau_1 - \kappa_2 T_2(0) \tau_2 + R_n (\tau)) \\
&= \kappa_1 |T_1(0)|^2 \tau_1^2 + \kappa_2 |T_2(0)|^2 \tau_2^2 -\frac{1}{2}\kappa_1 |n_0|^2 \tau_1^2  -\frac{1}{2}\kappa_2 |n_0|^2 \tau_2^2 - \eta(\tau) \cdot R_n(\tau) \\
&=\frac{1}{2}(\kappa_1 \tau_1^2 + \kappa_2 \tau_2^2 - 2\left( \sum_{j=1}^2\left(T_j(0) \tau_j + \frac{\kappa_j}{2} n_0 \tau_j^2 \right) + R_{\eta}(\tau)  \right) \cdot R_n(\tau)) \\
&= \frac{1}{2} \left(\sum_{j=1}^2 \kappa_j \tau_j^2 +R_{\eta n}(\tau) \right)
\end{align*}
mit $|R_{\eta n}(\tau)| \leq C |\tau|^3$, wobei wir die Abschätzungen für $R_\eta$ und $R_n$ aus Lemma \ref{Leta} verwenden und $T_j(0), \kappa_j$ und $n_0$ gegen $||\eta||_{3,\infty}$ abschätzen.
 
 
 
Wegen Lemma \ref{Lxi} gilt
\begin{align*}
\frac{\partial\tau_i}{\partial\xi_i} &=\frac{\partial}{\partial \xi_i}\left[\xi_i(1+R_\tau(\xi))\right] \\
  &=\frac{\partial R_\tau}{\partial \xi_i} (\xi)\xi_i + (1+R_\tau(\xi))
\end{align*}
und für $i \neq j$ 
\begin{align*}
\frac{\partial\tau_i}{\partial\xi_j} &=\frac{\partial}{\partial \xi_j}\left[\xi_i(1+R_\tau(\xi)) \right] \\
&=\frac{\partial R_\tau}{\partial \xi_j} (\xi)\xi_i,
\end{align*}
d.h. $|\det \frac{\partial \tau}{\partial \xi}(\xi)|= 1+ R_{\det}(\xi)$ mit $|R_{\det}(\xi)|\leq C |\xi|^2$. 




Für $|\partial_1 \eta (\tau) \times \partial_2 \eta(\tau)|= |T_1(\tau) \times T_2(\tau)|$ erhalten wir mithilfe der Entwicklung von $\eta(\tau)$ aus \ref{Leta}
\[
T_j(\tau)=T_j(0)+ \kappa_j n_0 \tau_j + \frac{\partial}{\partial \tau_j} R_\eta(\tau),
\] also aufgrund der Orthogonalität von $n_0,T_1(0)$ und $T_2(0)$, dass




\begin{align*}
|T_1(\tau) \times T_2(\tau)|^2 &= |n_0 + \kappa_2\tau_2 T_1(0) \times n_0 + \kappa_1 \tau_1 n_0 \times T_2(0) + \frac{\partial}{\partial \tau_1} R_\eta(\tau) \times T_2(\tau)|^2 \\
&= |n_0 + \kappa_2\tau_2 T_2(0) + \kappa_1 \tau_1 T_1(0) + \frac{\partial}{\partial \tau_1} R_\eta(\tau) \times T_2(\tau)|^2 \\
&= 1 + \kappa_2^2 \tau_2^2 + \kappa_1^2\tau_1^2 \\
&\quad + (n_0 + \kappa_2\tau_2 T_2(0) + \kappa_1 \tau_1 T_1(0) +T_1(\tau)\times T_2(\tau)) \cdot \left(\frac{\partial}{\partial \tau_1} R_\eta(\tau) \times T_2(\tau)\right). \\
\end{align*}
Mit der Entwicklung der Wurzelfunktion folgt damit

%$R_{|T_1\times T_2|^2}(\tau):=\kappa_2^2 \tau_2^2 + \kappa_1^2\tau_1^2 + (T_1(\tau)\times T_2(\tau)) \cdot %\left(\frac{\partial}{\partial_{\tau_1}} R_\eta(\tau) \times T_2(\tau)\right)$.

\begin{align*}
|T_1(\tau) \times T_2(\tau)|&= 1+ R_{T_1 \times T_2}(\tau)
\end{align*}
mit $|R_{T_1 \times T_2}(\tau)| \leq C ||\eta||^2_{3,\infty} |\tau|^2$.
Damit ist mit Lemma \ref{Lxi}
\begin{align*}
\left|\det \frac{\partial \tau}{\partial \xi}(\tau)\right||T_1(\tau) \times T_2(\tau)|
&= (1+ R_{\det}(\xi(\tau))(1+ R_{T_1 \times T_2}(\tau)) \\
&= 1+ R_1(\tau)
\end{align*}
mit $|R_1(\tau)| \leq C |\tau|^2$.
Mit der Entwicklung 
\[
\varphi(\tau) = \sum_{i=1}^2\left(\varphi_i \tau_i + \frac{1}{2}\left(\sum_{j=1}^2\varphi_{ij}\tau_i\tau_j \right) + \frac{1}{2}R_\varphi(\tau)\right)
\]
für $\tau \in \R^2$ folgt damit die Behauptung.


%mit $R_{T_1 \times T_2}(\tau):=\frac{1}{2} R_{|T_1\times T_2|^2}(\tau) + R_{\sqrt{\cdot}}(R_{|T_1\times T_2|^2}(\tau)) \in O(|\tau|^2)$
\end{proof}

Mithilfe der Entwicklung von $w$ erhalten wir jetzt eine Abschätzung für den Fehler.

\begin{Satz} \label{Seps1}
Sei $\varphi \in C^3_{\per}(\R^2)$ und $1\geq\delta >0$. Dann gilt für den lokalen Fehler bezüglich $s_1$
\begin{align*}
\widetilde \varepsilon_1 & \leq C ||\varphi||_{3,\infty} \delta^3
\end{align*}
für eine von $\delta$ und $\varphi$ unabhängige Konstante $C>0$.
\end{Satz}
\begin{proof}
Wir multiplizieren die Entwicklung von $w$ aus und trennen die Summanden nach den Potenzen in $\tau$.
Es gilt nach Lemma \ref{Entww}
\begin{equation*}
\widetilde \varepsilon_1 = \frac{1}{4 \pi} \int_{B_d(0)}  \frac{h(|\xi|/\delta)}{|\xi|^3} w(\xi)\dd \xi,
\end{equation*}
mit 

\begin{align*}
w(\tau)&= -\eta(\tau) \cdot n(\tau) \varphi(\tau)|\partial_1 \eta (\tau) \times \partial_2 \eta(\tau)||\operatorname{det}\frac{\partial\tau}{\partial\xi}(\tau)| \\
&=\frac{1}{2} \left(\sum_{j=1}^2 \kappa_j \tau_j^2 +R_{\eta n}(\tau) \right)
\sum_{i=1}^2\left(\varphi_i \tau_i + \frac{1}{2}\left(\sum_{j=1}^2\varphi_{ij}\tau_i\tau_j \right) + \frac{1}{2}R_\varphi(\tau)\right)(1+ R_1(\tau)) \\
&=(w_1(\tau) + w_2(\tau) +w_3(\tau))(1+R_1(\tau)),
\end{align*} 
wobei 
\[
w_1(\tau)=\frac{1}{2}\left(\sum_{j=1}^2 \kappa_j \tau_j^2\right)\sum_{i=1}^2\varphi_i \tau_i \in O(|\tau|^3),
\]
\[
w_2(\tau)=\left[
\frac{1}{4}\left(\sum_{j=1}^2 \kappa_j \tau_j^2\right) \left(\sum_{i=1}^2 \sum_{j=1}^2\varphi_{ij}\tau_i\tau_j \right)
+ \frac{1}{2}R_{\eta n}(\tau)  \left(\sum_{i=1}^2\varphi_i \tau_i\right)
\right]  \in O(|\tau|^4),
\]


\[
w_3(\tau)= \frac{1}{2} \left( \left( 
\sum_{j=1}^2 \kappa_j \tau_j^2 \right)
 R_\varphi(\tau) + \frac{1}{2} R_{\eta n}(\tau)
 \sum_{i=1}^2\left(\sum_{j=1}^2\varphi_{ij}\tau_i\tau_j  + R_\varphi(\tau) \right) \right) \in O(|\tau|^5),
\]
für alle $\tau \in \xi(B_d(0)$.
Nutzen wir nun die Darstellung $\tau_j(\xi)=\xi_j(1+R_\tau(\xi))$ mit $|R_\tau(\xi)| \leq C||\eta||_{4,\infty} |\xi|^2$ aus Lemma \ref{Lxi}, so erhalten wir Darstellungen von $w_i$, $i=1,2,3$ in den $\xi$ Koordinaten 
\[
w_1(\xi)=\frac{1}{2}(1+R_\tau(\xi))^3\left(\sum_{j=1}^2 \kappa_j \xi_j^2\right)\sum_{i=1}^2\varphi_i \xi_i \in O(|\xi|^3),
\]
\begin{align*}
w_2(\xi)&=
\frac{1}{4}\left((1+R_\tau(\xi))^4\sum_{j=1}^2 \kappa_j \xi_j^2\right) \left(\sum_{i=1}^2 \sum_{j=1}^2\varphi_{ij}\xi_i\xi_j \right) \\
& \quad + \frac{1}{2}(1+R_\tau(\xi))R_{\eta n}(\xi(1+R_\tau(\xi)))  \left(\sum_{i=1}^2\varphi_i \xi_i\right)
  \in O(|\xi|^4),
\end{align*}
\begin{align*}
w_3(\xi)&= \frac{1}{2} \bigg[\left( (1+R_\tau(\xi))^2
\sum_{j=1}^2 \kappa_j \xi_j^2 \right)
 R_\varphi(\xi(1+R_\tau(\xi))) \\
 & \qquad + \frac{1}{2} R_{\eta n}(\xi(1+R_\tau(\xi)))
 \sum_{i=1}^2\left(\sum_{j=1}^2(1+R_\tau(\xi))^2\varphi_{ij}\xi_i\xi_j  + R_\varphi(\xi(1+R_\tau(\xi))) \right) \bigg] \in O(|\xi|^5),
\end{align*}
für alle $\xi \in B_d(0)$.

Da der Integrationsbereich symmetrisch um $0$ und $h(|\xi|/\delta)/|\xi|^3$ gerade ist, leisten die ungeraden Anteile in $w$ keinen Beitrag zum Fehler, das heißt alle Terme mit ungeraden Potenzen in $\xi$.
Insbesondere verschwindet damit der Beitrag der Terme von Ordnung 3 aus $w_1$. 
Damit können wir $\widetilde \varepsilon_1$ also darstellen als 
\begin{align*}
\widetilde \varepsilon_1 = \frac{1}{4 \pi} \int_{B_d(0)}  \frac{h(|\xi|/\delta)}{|\xi|^3} \tilde{w}(\xi)\dd \xi
\end{align*}  
mit $\tilde{w}(\xi) \in O(|\xi|^4)$.
Substituieren wir nun $\xi = \delta \zeta$, dann ist die Funktionaldeterminante $\left|\frac{\partial \xi}{\partial \zeta}\right|= \delta^2$ und wir erhalten mit $\tilde w(\xi) = \delta^4 \overline{w}(\zeta)$ mit einer beschränkten Funktion $\overline{w}\in C(\R^2)$
\[
\varepsilon= \delta^{-3+4+2} \int_{B_{\delta d}(0) } \frac{h(|\zeta|)}{|\zeta|^3}\overline{w}(\zeta) \dd \zeta \leq  C ||\varphi||_{3,\infty} \delta^3 ,
\]
indem wir $\delta^3 \geq \delta^k$ für $k>3, |\delta|\leq 1$ nutzen und $\varphi_i$ und $\varphi_{ij}$ durch $||\varphi||_{3,\infty}$ bzw. $|R_\varphi(\xi(1+R_\tau(\xi)))| \leq C ||\varphi||_{3,\infty} |\xi|^3$ abschätzen.
\end{proof}
\subsection{Fehler für die Glättung mit $s_2$}
Wir wir im Beweis von Satz \ref{Seps1} festgestellt haben, verschwindet der Beitrag ungerader Anteile von $w$ am Fehler. Damit können wir die Fehlerordnung auf $\delta^5$ verbessern, wenn wir unsere Glättungsfunktion so wählen, dass auch der Anteil der Terme in $w$ die proportional zu $|\xi|^4$ sind, verschwindet. Die zweite Glättungsfunktion $s_2$ erfüllt genau diese Bedingung, wie der folgende Satz zeigt. 
\begin{Satz} \label{Seps2}
Sei $\varphi \in C_{\per}^3(\R^2)$ und $1\geq\delta >0$. Dann gilt für den Fehleranteil mit Singularität bezüglich $s_2$
\begin{align*}
\widetilde \varepsilon_2 &\leq C ||\varphi||_{3,\infty} \delta^5
\end{align*}
für eine von $\delta$ und $\varphi$ unabhängige  Konstante $C>0$.
\end{Satz}     
\begin{proof}
Im Beweis von Satz \ref{Seps1} sieht man, dass die Terme 4-ter Ordnung in $w$ einen Fehler proportional zu 
\begin{align*}
\int_{B_d(0)} \frac{h(|\xi|/\delta)}{|\xi|^3}|\xi|^4\dd \xi = \delta^3 \int_{B_{\delta d}(0)} h(\zeta) |\zeta| \dd \zeta \leq 2\pi \delta^3 \int_0^\infty h(r) r^2 \dd r
\end{align*} 
liefern, wobei wir wieder die Transformation $\xi=\delta\zeta$ und anschließend Polarkoordinaten verwendet haben. Setzen wir $h_5(r)= h(r) + \frac{1}{3}r h'(r)$ für $r \geq 0$ so erhalten wir mit partieller Integration
\begin{align*}
\int_0^\infty h_5(r) r^2 \dd r &= \int_0^\infty h(r)r^2 \dd r  + \int_0^\infty \frac{1}{3}r^3h'(r) \dd r \\
&=\int_0^\infty h(r)r^2 \dd r + \big[\frac{1}{3} r^3 h(r)\big]_0^\infty - \int_0^\infty h(r)r^2 \dd r \\
&= \lim_{r \to \infty} \frac{1}{3} r^3 h(r) - \lim_{r \to 0} \frac{1}{3} r^3 h(r) \\
&= 0,
\end{align*}
da $h(s)= \erfc(s)+(2/\sqrt{\pi})s e ^{-s^2 }$, $s \geq 0$, wobei die Grenzwerte beispielsweise aus dem Satz von L'Hospital folgen. Rechnen wir nun Lemma \ref{eps1} für den Kern bezüglich der Glättungsfunktion $s_2$ nach, so erhalten wir gerade $h_5(\rho)$ als nichtradialen Anteil von $\widetilde \varepsilon_2$. Es gilt nämlich nach Satz \ref{k1ddef} für $\tau \in Q$
\begin{align*}
& \quad K_{2,\delta}(x,\eta(\tau))- K(x,\eta(\tau)) \\
&= n(\eta(\tau)) \cdot \grad_y \left[ \left(\erf \left(\frac{|x-y|}{\delta}\right) +\frac{2}{3\sqrt{\pi}}\frac{|x-y|}{\delta}e^{-\frac{|x-y|^2}{\delta^2}} - 1\right) \Phi (x,\eta(\tau)) \right]
\end{align*}


und mit $r=|x - \eta(\tau)|$, dass
\begin{align*}
&\quad\grad_y\left[\left(\erf \left(\frac{r}{\delta}\right) +\frac{2}{3\sqrt{\pi}}\frac{r}{\delta}e^{-\frac{r^2}{\delta^2}} - 1\right) \Phi (x,\eta(\tau)) \right] \\
&= - \grad_y\left(\frac{\erfc(\frac{r}{\delta})-\frac{2}{3\sqrt{\pi}}\frac{r}{\delta}e^{-\frac{r^2}{\delta^2}}}{4 \pi r}\right) \\
&= - \frac{1}{4 \pi} \left(\erfc'\left(\frac{r}{\delta}\right)\frac{-\eta(\tau)}{\delta r^2} + \erfc\left(\frac{r}{\delta}\right)\frac{\eta(\tau)}{r^3} \right)
-\frac{1}{4 \pi} \left( \frac{2}{3 \sqrt{\pi} \delta^3} e^{-\frac{r^2}{\delta^2}}2 \eta(\tau)  \right) \\
&=-\frac{1}{4 \pi}\frac{\eta(\tau)}{r^3} \left( h \left(\frac{r}{\delta} \right) + \frac{1}{3}\frac{r}{\delta}h' \left(\frac{r}{\delta}\right) \right) \\
&=-\frac{1}{4 \pi}\frac{\eta(\tau)}{r^3} h_5 \left(\frac{r}{\delta} \right)
\end{align*} 
und 
\[
\erfc'(s)=-\erf'(s)=-(2/\sqrt{\pi})e^{-s^2}, \qquad s \in \R.
\]

Damit verschwinden für den Fehler bezüglich $s_2$ auch die Anteile der Terme in $w$ von Ordnung 4 und 5, womit wir eine Darstellung
\begin{align*}
\widetilde \varepsilon_2 &= \frac{1}{4 \pi} \int_{Q}  \frac{h(|\xi|/\delta)}{|\xi|^3} \tilde{w}(\xi)\dd \xi
\end{align*}  
mit $\tilde{w}(\xi) \in O(|\xi|^6)$ finden. Wir gehen weiter wie in Satz \ref{Seps1} vor und bekommen mit der Substitution $\xi = \delta \zeta$ eine beschränkte Funktion $\overline{w}\in C(\R^2)$ mit $\tilde w(\xi) = \delta^6 \overline{w}(\zeta)$ und
\[
\widetilde \varepsilon_2= \delta^{-3+6+2} \int_{\tilde{Q} } \frac{h(|\zeta|)}{|\zeta|^3}\overline{w}(\zeta) \dd \zeta \leq  C ||\varphi||_{3,\infty} \delta^5.
\]
\end{proof}
%Bilden wir nun das Maximum über alle Konstanten $C>0$ aus den Sätzen \ref{Seps1} und \ref{Seps2} für die endlich vielen Umgebungen, die wir brauchen um $Q$ zu überdecken, so erhalten wir eine gleichmäßige Abschätzung für den Fehler.
Mit Lemma \ref{LepsohneSing} und den Sätzen \ref{Seps1} und \ref{Seps2} erhalten wir wegen $\varepsilon=  \varepsilon_j + \widetilde \varepsilon_j$ für $j=1$ oder $j=2$ eine Abschätzung für den Gesamtfehler.
\begin{Satz}
Sei $\varphi \in C_{\per}^3(\R^2)$, $1\geq\delta >0$, $\eta\in C^4(Q,\partial D)$. Dann gilt für $j=1,2$
\begin{align*}
||(A-A_{j,\delta})\varphi||_\infty \leq C ||\varphi||_{3,\infty}\delta^{2j+1}.
\end{align*}

\end{Satz}  


%
%\begin{Lemma}
%Der lokale Fehler $\varepsilon$ genügt der Abschätzung
%\begin{equation}
%\varepsilon \leq \delta^3 \frac{1}{4 \pi} \int_{\widetilde{Q}}  \frac{h(z)}{z^3} \tilde{w}(z)\dd z
%\end{equation}
%
%\end{Lemma}
%\begin{proof}
%Wir multiplizieren die Entwicklung von $w$ aus und sortieren nach aufsteigenden Potenzen in $\tau$. Um die Übersichtlichkeit beizubehalten verzichten wir solange es geht darauf $\tau$ durch $\tau= \xi^{-1}(\xi)$ zu ersetzen. Nach Lemma \ref{Entww} besitzt $w$ die Entwicklung
%\begin{align*} 
%w(\tau)&=\frac{1}{2}\left(\kappa_1 \tau_1^2 + \kappa_2 \tau_2^2 +R_{\eta n}(\tau)\right)
%\sum_{i=1}^2\left(\varphi_i \tau_i + \frac{1}{2}(\varphi_{i1}\tau_i\tau_1 + \varphi_{i2}\tau_i\tau_2) + \frac{1}{2}R_\varphi(\tau)\right)R_1(\tau) \\
%&= [\frac{1}{2}\left(\kappa_1 \tau_1^2 + \kappa_2 \tau_2^2\right)\sum_{i=1}^2\varphi_i \tau_i+\left[\frac{1}{4}\left(\kappa_1 \tau_1^2 + \kappa_2 \tau_2^2\right)\sum_{i=1}^2(\varphi_{i1}\tau_i\tau_1 + \varphi_{i2}\tau_i\tau_2) +\sum_{i=1}^2 R_{\eta n}(\tau) \varphi_i \tau_i\right] \\
%&+ \frac{1}{2}\left(\kappa_1 \tau_1^2 + \kappa_2 \tau_2^2\right)R_\varphi(\tau) +R_{\eta n}(\tau) \sum_{i=1}^2\left(\frac{1}{2}(\varphi_{i1}\tau_i\tau_1 + \varphi_{i2}\tau_i\tau_2) + \frac{1}{2}R_\varphi(\tau)\right)
%] (1+R_2 (\tau)) \\
%&= (w_1(\tau) + w_2(\tau) +w_3(\tau) )(1+R_2(\tau)) 
%\end{align*}
%mit 
%\[
%w_1(\tau):=\frac{1}{2}\left(\kappa_1 \tau_1^2 + \kappa_2 \tau_2^2\right)\sum_{i=1}^2\varphi_i \tau_i \in O(|\tau|^3)
%\]
%\[
%w_2(\tau):=\left[\frac{1}{4}\left(\kappa_1 \tau_1^2 + \kappa_2 \tau_2^2\right)\sum_{i=1}^2(\varphi_{i1}\tau_i\tau_1 + \varphi_{i2}\tau_i\tau_2) +\sum_{i=1}^2 R_{\eta n}(\tau) \varphi_i \tau_i\right]  \in O(|\tau|^4)
%\]
%\[
%w_3(\tau):=\frac{1}{2}\left(\kappa_1 \tau_1^2 + \kappa_2 \tau_2^2\right)R_\varphi(\tau) +R_{\eta n}(\tau) \sum_{i=1}^2\left(\frac{1}{2}(\varphi_{i1}\tau_i\tau_1 + \varphi_{i2}\tau_i\tau_2) + \frac{1}{2}R_\varphi(\tau)\right)  \in O(|\tau|^5)
%\]
%Wir können die Integraldarstellung des Fehlers vereinfachen, indem wir $\varphi(\xi)=0$ für $\xi \notin \xi^{-1}(Q)$ setzen womit
%\begin{align*}
%\varepsilon &= \frac{1}{4 \pi} \int_{\xi^{-1}(Q)}  \frac{h(|\xi|/\delta)}{|\xi|^3} w(\xi)\dd \xi \\
%&=\frac{1}{4 \pi} \int_{\R^2}  \frac{h(|\xi|/\delta)}{|\xi|^3} w(\xi)\dd \xi
%\end{align*}
%Da die Abbildung $m(\xi):=\frac{h(|\xi|/\delta)}{|\xi|^3}$ gerade ist, d.h. $m(-\xi)=m(\xi)$, tragen die ungerade Terme von $w(\xi)$ nichts zu $\varepsilon$ bei. Insbesondere heißt das, dass 
%$\int_{\R^2}  \frac{h(|\xi|/\delta)}{|\xi|^3} w_1(\xi)\dd \xi =0$. 
%  
%\end{proof}
\newpage





\section{Fehler der Diskretisierung} \label{chaDiskErr}
Um den lokalen Fehler der Diskretisierung (vergleiche Kapitel \ref{DiskProblem})
\begin{align*}
\varepsilon &= |(A_{j,\delta}-A_{j,\delta,N})\varphi(t)| \\
  &= |\int_Q K_{j,\delta}(t,\tau) (\varphi(\tau)- \varphi(t)) \dd \tau-\frac{\pi^2}{N^2}\sum_{\nu\in\Z_N^2}  K_{j,\delta}(t,t^{(\nu)}) \left(\varphi(t^{(\nu)})- \varphi(t)\right)|
\end{align*}
für $\delta>0, \varphi \in C_{\per}^2(Q)$, $t \in Q$, $N\in \N$, $j=1,2$ abzuschätzen, zerlegen wir zunächst das Integral in einen lokalen Anteil um die Singularität und einen global gleichmäßig glatten Anteil. Wir gehen zusätzlich davon aus, dass die Parametrisierung $\eta \in C_{\per}^\infty(\R^2,\partial D)$ glatt ist und die Lösung $\varphi\in C_{\per}^2(\R^2)$, also mindestens 2-mal stetig differenzierbar ist. Im Laufe des Kapitels werden wir sehen, dass wir für eine Abschätzung mit Ordnung $2$ mindestens $\varphi\in C_{\per}^{12}(\R^2)$ fordern müssen (siehe Bemerkung \ref{Bck12}) und ein noch stärkeres Ergebnis erhalten, wenn $\varphi\in C_{\per}^{\infty}(\R^2)$ gilt.
\subsection{Eine erste Abschätzung}
Wegen der Periodizität von $K$ gilt für $t \in Q$
\[
A_{j,\delta}\varphi(t) = \int_{t+Q} K_{j,\delta}(t,\tau) (\varphi(\tau)- \varphi(t)) \dd \tau.
\]
Für die Zerlegung wählen wir eine Abschneidefunktion $\chi \in C^\infty(\R^2)$ mit $\supp\chi \subset Q$ und $\chi \equiv 1$ in einer Umgebung von 0. Damit können wir den regularisierten Operator darstellen als
\[
A_{j,\delta}\varphi(t) = A^{(0)}_{j,\delta}\varphi(t) + A^{(1)}_{j,\delta}\varphi(t)
\] mit
\[
A^{(0)}_{j,\delta}\varphi(t)=\int_{t+Q} K_{j,\delta}(t,\tau)\chi(t-\tau)(\varphi(\tau)- \varphi(t)) \dd \tau
\]
\[
A^{(1)}_{j,\delta}\varphi(t)=\int_{t+Q} K_{j,\delta}(t,\tau)(1-\chi(t-\tau))(\varphi(\tau)- \varphi(t)) \dd \tau
\]

Um die Differenz dieser Integrale zur Summe abzuschätzen wollen wir das folgende Lemma anwenden.
\begin{Lemma} \label{epsdisk}
Es seien $ 1\geq \delta, h>0$ mit $\delta/h = \rho \geq \rho_0$. Es sei $\lambda \in \N_0^2$ ein Multiindex, $q\in \Z, p \in \N_{\geq |\lambda|+3}$, und $F \in C_0^p(\R^2)$. Ferner sei $P\in C^\infty(\R^2)$ mit 
\[
|D^\alpha P(z)| \leq \frac{C}{|z|^{|\alpha|}}, \qquad |z|\geq 1, \qquad \alpha \in \N_0^2.
\]
Dann gilt
\begin{align*}
\left|\int_{\R^2} \delta^{-q}F(\tau)\tau^\lambda P\left(\frac{\tau}{\delta}\right) \dd \tau - h^2 \delta^{-q} \sum_{\nu \in \Z^2}F(t^{(\nu)})t^{(\nu)^\lambda} P\left(\frac{t^{(\nu)}}{\delta}\right)\right| \\
\leq C( \rho_0^{-p}h^{|\lambda|+2-q}+\rho_0^{-q}h^{p-q})||F||_{p,\infty}
\end{align*}
\end{Lemma}
Für den Beweis nutzen wir die folgende Abschätzung für den Fehler der Trapezregel.
\begin{Lemma} \label{Lepsdisk}
Sei $g \in C^r(\R^n)$ mit kompaktem Träger, d.h. $\supp(g)=\{x\in \R^n | g(x) \neq 0 \}$ ist kompakt, und $||g||_r = \max_{l=1, \dots n}||\partial_l^r g ||_{L^1(\R^n)}$. Dann gilt für $r \geq n+1, h>0$
\begin{align}
\left|\sum_{i \in \Z^n}g(i h) h^n - \int_{\R^n}g(x) \dd x \right| \leq \frac{52}{(2\pi)^r}||g||_r h^r
\end{align}
\end{Lemma}
\begin{proof}
Siehe Lemma 2.2 in \cite{Anderson}.
\end{proof}
\begin{proof}[Beweis von Lemma \ref{epsdisk}]
Wir wenden Lemma \ref{Lepsdisk} auf die Funktion $g(\tau)=F(\tau) \tau^\lambda P(\tau/\delta)$ mit $n=2$ und $r=p$ an und erhalten
\begin{align*}
\bigg|\int_{\R^2} \delta^{-q}F(\tau)\tau^\lambda P\left(\frac{\tau}{\delta}\right) \dd \tau - &  h^2 \delta^{-q} \sum_{\nu \in \Z^2}F(t^{(\nu)})t^{(\nu)^\lambda} P\left(\frac{t^{(\nu)}}{\delta}\right)\bigg| \\ 
&\leq \frac{52}{(2\pi)^p}\delta ^{-q}\max_{l=1, \dots n}||\partial_l^p g ||_{L^1(\R^n)} h^p \\
&= C h^p \delta^{-q} \max_{j=1,2} \int_{\R^2} \left| \frac{\partial^p}{\partial \tau_j^p}\left[F(\tau) \tau^\lambda P\left(\frac{\tau}{\delta}\right) \right] \right| \dd \tau
\end{align*}
mit einer Konstante $C>0$. Nach der allgemeinen Produktregel haben die Ableitungen die Darstellung
\[
\frac{\partial^p}{\partial \tau_j^p}\left[F(\tau) \tau^\lambda P\left(\frac{\tau}{\delta}\right) \right] = \sum_{k=0}^p \sum_{l=0}^{\min\{p-k,\lambda_j\}} C_{k,l} \frac{\partial^{p-k-l}F(\tau)}{\partial \tau_j}\tau^{\lambda-l e_j} \frac{\dd^k}{\dd z^k}P\left(\frac{\tau}{\delta}\right) \delta^{-k}.
\]
mit den Einheitsvektoren $e_1=(1,0), e_2=(0,1)$ und nur von $k$ und $l$ abhängigen Konstanten $C_{k,l}>0$.
Da $F$ nach Voraussetzung einen beschränkten Träger hat, können wir $R>0$ so groß wählen, dass $\supp F \subset B_R (0)$ ist. Damit lässt sich das obige Integral in Polarkoordinaten $(r,\vartheta)$ transformieren, d.h. mit $\tau = r( \cos (\vartheta),\sin(\vartheta))^T$ und Funktionaldeterminante $r$. Mit der Dreiecksungleichung erhalten wir
\begin{align*}
 \int_{\R^2} \bigg| &\frac{\partial^p}{\partial \tau_j^p} \left[F(\tau) \tau^\lambda P\left(\frac{\tau}{\delta}\right) \right] \bigg| \dd \tau \\ 
  \leq & ||F||_{p,\infty} \int_{-\pi}^\pi \int_0^R   \sum_{k=0}^p \sum_{l=0}^{\min\{p-k,\lambda_j\}} C_{k,l} \bigg| \begin{pmatrix}
r\cos(\vartheta) \\ r\sin(\vartheta)
 \end{pmatrix}^{\lambda-l e_j}\delta^{-k}\bigg| \bigg|\frac{\dd^k}{\dd z^k}P\left(\frac{r( \cos (\vartheta),\sin(\vartheta))^T}{\delta}\right)\bigg|r \dd r \dd \vartheta \\
  \leq & C ||F||_{p,\infty} \sum_{k=0}^p \sum_{l=0}^{\min\{p-k,\lambda_j\}}
 \int_{-\pi}^\pi \int_0^R r
^{|\lambda|-l+1} \delta^{-k} \bigg|\frac{\dd^k}{\dd z^k}P\left(\frac{r( \cos (\vartheta),\sin(\vartheta))^T}{\delta}\right)\bigg| \dd r \dd \vartheta
\end{align*} 
Der Term mit $P$ lässt sich abschätzen indem wir das Integral über $r$ in eines über das Intervall $(0,\delta)$ und eines über $(\delta,R)$ zerlegen. 
Damit erhalten wir nach Voraussetzung an $P$ die von $\vartheta$ unabhängige Abschätzung 
\[
\bigg|\frac{\dd^k}{\dd z^k}P\left(\frac{r( \cos (\vartheta),\sin(\vartheta))^T}{\delta}\right)\bigg| \leq \bigg\lbrace
\begin{matrix}
C \quad &\text{ für } &0\leq r<\delta \\ 
C / \left(\frac{r}{\delta}\right)^k \quad &\text { für } &\delta < r \leq R,
\end{matrix}
\] und damit für das Integral
\[
\int_{\R^2} \bigg| \frac{\partial^p}{\partial \tau_j^p} \left[F(\tau) \tau^\lambda P\left(\frac{\tau}{\delta}\right) \right] \bigg| \dd \tau \leq C ||F||_{p,\infty} \sum_{k=0}^p \sum_{l=0}^{\min\{p-k,\lambda_j\}} \left[ \int_0^\delta r^{|\lambda|-l+1} \delta^{-k} \dd r + \int_\delta^R r^{|\lambda|-l-k+1} \dd r \right]
\]
Die Integrale ergeben sich zu
\begin{align*}
\int_0^\delta r^{|\lambda|-l+1} \delta^{-k} \dd r + \int_\delta^R r^{|\lambda|-l-k+1} \dd r 
&= (|\lambda| - l +2)^{-1} \delta^{|\lambda|-l-k+2}  \\ 
& \quad +(|\lambda| - l - k +2)^{-1}\left[ R^{|\lambda| - l - k + 2} - \delta^{|\lambda| - l - k + 2} \right] \\
&\leq C_1 R^{|\lambda| - l - k + 2} + C_2 \delta^{|\lambda| - l - k + 2}
\end{align*}
mit von $\delta$ unabhängigen Konstanten $C_1,C_2>0$.
Wegen $\delta \leq 1$ und $l + k \leq p$  können wir $\delta^{|\lambda| - l - k + 2} \leq \delta^{|\lambda|+2-p}$ abschätzen. Insgesamt erhalten wir damit 
\[
\int_{\R^2} \bigg| \frac{\partial^p}{\partial \tau_j^p} \left[F(\tau) \tau^\lambda P\left(\frac{\tau}{\delta}\right) \right] \bigg| \dd \tau \leq C ||F||_{p,\infty}(1 + \delta^{|\lambda|+2-p})
\]
für eine von $\delta$ unabhängige Konstante $C>0$. Dies liefert die Behauptung.
\end{proof}
Um Lemma \ref{epsdisk} auf den lokalisierten Operator $A^{(0)}_{j,\delta}$ anzuwenden, verwenden wir eine Taylorentwicklung des Integranden. 
%\begin{Satz} \label{STaylor}
%Sei $F\in C^r(D)$ mit $r\in \N$ und $D \subset \R^n$, dann gilt für $z,z_0 \in D$
%\begin{align}
%F(z) = \sum_{k=0}^{r-1} \sum_{|\alpha|=k} \frac{(z-z_0)^\alpha}{\alpha!}D^\alpha F(z_0) + R_r(z)
%\end{align}
%mit 
%\[
% R_r(z) = \sum_{|\alpha|=r}\frac{r}{\alpha!}\int_0^1 (1-\xi)^r D^\alpha F(z_0 + \xi(z-z_0)) \dd \xi(z-z_0)^\alpha
%\]
%\end{Satz}  
%\begin{proof}
%Siehe $\S 7$ Satz 2 in \cite{Forster} \todo{Quelle mit Integraldarstellung angeben}
%\end{proof}
Da sich der lokalisierte Operator schreiben lässt als
\begin{align*}
A^{(0)}_{j,\delta}\varphi(t) & =\int_{t+Q} K_{j,\delta}(t,\tau)\chi(t-\tau)(\varphi(\tau)- \varphi(t)) \dd \tau \\
&=\int_{t+Q}\frac{\partial \Phi(\eta(t),\eta(\tau))}{\partial n (\eta(\tau))}|\partial_1 \eta (\tau) \times \partial_2 \eta(\tau)|s_j \left(\frac{|\eta(t)-\eta(\tau)|}{\delta}\right)\chi(t-\tau)(\varphi(\tau)- \varphi(t)) \dd \tau \\
&=\int_{t+Q}\frac{(\partial_1 \eta (\tau) \times \partial_2 \eta(\tau))\cdot (\eta(t)-\eta(\tau))}{4 \pi |\eta(t)-\eta(\tau)|^3}s_j \left(\frac{|\eta(t)-\eta(\tau)|}{\delta}\right)\chi(t-\tau)(\varphi(\tau)- \varphi(t)) \dd \tau \\
&=\int_{t+Q}(\partial_1 \eta (\tau) \times \partial_2 \eta(\tau)) G_\delta(\eta(t)-\eta(\tau)) \chi(t-\tau)(\varphi(\tau)- \varphi(t)) \dd \tau
\end{align*}
mit 
\[
G_\delta(z) := \frac{z}{4 \pi |z|^3}s_j\left(\frac{|z|}{\delta}\right), \qquad z \in \R^3.
\]
 entwickeln wir zunächst die Komponenten von $G_\delta$. 
\begin{Lemma} \label{LDa}
Für jeden Multiindex $\alpha \in \N_0^3$ gibt es beschränkte analytische Funktionen $P_{\alpha,\beta,\ell}$ unabhängig von $\delta$ mit
\[
D^\alpha G_{\delta,\ell}(z) = \sum_{|\beta|\leq |\alpha|+1}\frac{1}{\delta^{|\beta|+|\alpha|+2}}z^\beta P_{\alpha,\beta,\ell} \left(\frac{|z|^2}{\delta^2}\right), \qquad z\in\R^3, \qquad l=1,2,3,
\] wobei $G_\delta(z) = (G_{\delta,1}(z),G_{\delta,2}(z),G_{\delta,3}(z))$ für $z\in\R^3$. Weiter gilt \[|P_{\alpha,\beta,\ell}^{(n)}(t)|\leq C \exp(-t)\] für alle $t>0,\ell=1,2,3, n \in \N_0,\alpha,\beta \in N_0^3$ mit $|\beta|\leq|\alpha|+1$. 
\end{Lemma}
\begin{proof}
Wir erinnern an die Potenzreihendarstellung der Fehlerfunktion $\erf$ aus \\ Lemma \ref{erfseries}
\[
\erf(z) = \frac{2}{\sqrt{\pi}}\sum_{n=0}^\infty \frac{(-1)^n z^{2 n +1}}{n!(2 n +1 )}, \qquad z\in\C.
\]
Damit gilt für die Potenzreihen von $s_1$ und $s_2$ für $r>0$ 
\begin{align*}
s_1(r) &= \erf(r) -\frac{2}{\sqrt{\pi}}r e^{-r^2} \\
  &= \frac{2}{\sqrt{\pi}}\sum_{n=0}^\infty \frac{(-1)^n r^{2 n +1}}{n!(2 n +1 )} - \frac{2}{\sqrt{\pi}}r \sum_{n=0}^\infty \frac{(-1)^n r^{2n}}{n!} \\
  &=\frac{2}{\sqrt{\pi}}\sum_{n=1}^\infty \frac{(-1)^{n+1}2n}{n!(2n +1)} r^{2 n +1}
\end{align*}
und 
\begin{align*}
s_2(r) 
&= \erf(r) -\frac{2}{\sqrt{\pi}}\left(r- \frac{2}{3}r^3\right) e^{-r^2} \\
&= s_1(r) + \frac{2}{\sqrt{\pi}}\frac{2}{3}r^3 e^{-r^2} \\
&= \frac{2}{\sqrt{\pi}}\left[\sum_{n=1}^\infty \frac{(-1)^{n+1}2n}{n!(2n +1)} r^{2 n +1} - \frac{2}{3} \sum_{n=0}^\infty \frac{(-1)^{n+1} r^{2n+3}(2n+1)}{n!(2n+1)} \right] \\
&= \frac{2}{\sqrt{\pi}}\left[\sum_{n=0}^\infty \frac{(-1)^{n+1}}{n!(2n +1)} \left( 2n r^{2 n +1} - \frac{2}{3} r^{2n+3}(2n+1)\right) \right] \\
\end{align*}
%\begin{align*}
%s_2(r) &= \erf(r) -\frac{2}{\sqrt{\pi}}\left(r- \frac{2}{3}r^3\right) e^{-r^2} \\
%  &=  \frac{2}{\sqrt{\pi}}\sum_{n=0}^\infty \frac{(-1)^n r^{2 n +1}}{n!(2 n +1 )} - \frac{2}{\sqrt{\pi}}\left(r- \frac{2}{3}r^3\right) \sum_{n=0}^\infty \frac{(-1)^n r^{2n}}{n!} \\
%  &= \frac{2}{\sqrt{\pi}}\sum_{n=0}^\infty \frac{(-1)^{n+1}2n r^{2 n +1}}{n!(2 n +1 )} - \frac{2}{\sqrt{\pi}}\frac{2}{3} \sum_{n=0}^\infty \frac{(-1)^{n+1} r^{2n+3}(2n+1)}{n!(2n+1)} \\
%  &= \frac{2}{\sqrt{\pi}}\sum_{n=0}^\infty \frac{(-1)^{n+1}}{n!(2n+1)}(2n r^{2 n +1} - \frac{2}{3} r^{2n+3}(2n+1)) \\
%  &= \frac{2}{\sqrt{\pi}}\sum_{n=1}^\infty \frac{(-1)^{n+1}}{n!(2n+1)}(r^{2 n +1}(2n+\frac{2n(2n+1)}{3(2n-1)}(2n-1)) \\
%  &=\frac{2}{\sqrt{\pi}}\sum_{n=1}^\infty \frac{(-1)^{n+1}}{n!(2n+1)}(r^{2 n +1}\frac{4n^2+8n}{3}) \\
%  &=\frac{2}{\sqrt{\pi}}\sum_{n=1}^\infty \frac{4(-1)^{n+1}(n+2)}{3(n-1)!(2n+1)}r^{2 n +1} .
%\end{align*}
Die Potenzreihen von $s_j$ besitzen also nur ungerade Potenzen und beginnen mit mindestens dritter Ordnung. (Die Reihe für $s_2$ beginnt sogar mit Ordnung 5, da sich die Terme mit $r^3$ aufheben.)  Es folgt
\begin{align*}
G_{\delta,l}(z) &= \frac{z_l}{4 \pi |z|^3} s_j \left( \frac{|z|}{\delta}\right) \\
&= \frac{z_l}{\delta^3}\tilde P \left(\frac{|z|^2}{\delta^2}\right).
\end{align*}
mit einer beschränkten Funktion $\tilde P$ mit $|\tilde P^{(n)}(t)|\leq C \exp(-t)$. Genauer gilt beispielsweise für $j=1$
\begin{align*}
G_{\delta,l}(z) &= \frac{z_l}{4 \pi |z|^3} s_1 \left( \frac{|z|}{\delta}\right) \\
&= \frac{2 z_l}{4\pi^{\frac{3}{2}}|z|^3}\sum_{n=1}^\infty \frac{(-1)^{n+1}2n}{n!(2n +1)} \left(\frac{|z|}{\delta}\right)^{2 n +1} \\
&= \frac{z_l}{\pi^{\frac{3}{2}}\delta^3}\sum_{n=1}^\infty \frac{(-1)^{n+1}}{(n-1)!(2n +1)} \left(\frac{|z|}{\delta}\right)^{2(n - 1)} \\
&= \frac{z_l}{\pi^{\frac{3}{2}}\delta^3}\sum_{n=0}^\infty \frac{(-1)^{n}}{n!(2n +3)} \left(\frac{|z|}{\delta}\right)^{2n} \\ 
&=\frac{ z_l}{\delta^3}\sum_{n=0}^\infty \frac{c(n)}{n!} \left(\frac{|z|}{\delta}\right)^{2n}
\end{align*}
mit $c(n) = \frac{(-1)^n}{\pi^{\frac{3}{2}}(2n+3)}$ für $n\in\N$ also $|c(n)|\leq 1$ für alle $n \in \N$. Für $j=2$ findet man analog eine Darstellung obiger Form.\\
Mit vollständiger Induktion folgt die Darstellung für die Ableitungen.
\end{proof}
Nun können wir die Taylorentwicklung von $G_{\delta,\ell}$ bestimmen.
\begin{Lemma} \label{LGdeltaTaylor}
Seien $\delta>0, \ell \in \{1,2,3\}$, $t,\tau \in \R^2$ und 
\begin{align*}
&J=\{(k,\alpha,\nu,\beta,\gamma)|k\in \{0,1,2,3\}, \alpha , \beta \in \N_0^3, \nu,\gamma \in \N_0^2\\
&\qquad \qquad \qquad \qquad \textrm{ mit } |\alpha|=k, |\nu|=2k, |\beta| \leq k+1, |\gamma|=|\beta|\}.
\end{align*}
Dann gibt es glatte Funktionen 
\begin{align*}
T_{\beta,\gamma}&: \R^{3 \times 2} \to \R,\\
R_{\alpha,\nu}&:\R^2 \times \R^2 \to \R, \\
R_\ell&: \R^3 \to \R,
\end{align*} 
für $(k,\alpha,\nu,\beta,\gamma)\in J$, so dass gilt

%\[
%	G_{\delta,l}(\eta(t)-\eta(\tau)) = \sum_{k=0}^3 \sum_{|\alpha|=k}\sum_{|\nu|=2k}  \sum_{|\beta|\leq k+1}S_{\alpha,\beta_,\nu,%\ell}(t,\tau) + R_\ell(\eta(t)-\eta(\tau))
%\]
\[
	G_{\delta,\ell}(\eta(t)-\eta(\tau)) = \sum_{(k,\alpha,\nu,\beta,\gamma)\in J} S_{\alpha,\beta_,\nu,\gamma,\ell}(t,\tau) + R_\ell(\eta(t)-\eta(\tau))
\]
mit 
\[
S_{\alpha,\beta_,\nu,\gamma,\ell}(t,\tau) = \frac{(t-\tau)^{\nu+\gamma}}{\delta^{|\beta|+|\alpha|+2}}T_{\beta,\gamma}(\eta'(t))P_{\alpha,\beta,\ell} \left(\frac{| \eta'(t)(t-\tau)|^2}{\delta^2}\right)R_{\alpha,\nu}(t,\tau),
\]
 und den analytischen Funktionen $P_{\alpha,\beta,\ell}$ aus Lemma \ref{LDa}.
Der Restterm $R_\ell$ besitzt die Darstellung
\begin{align*}
R_\ell(\eta(t)-\eta(\tau))&= \sum_{|\alpha|=4}\sum_{|\nu|=8}\sum_{|\beta|\leq |\alpha|+1}\frac{4 (t-\tau)^\nu  }{\alpha! \delta^{|\beta|+|\alpha|+2}}R_{\alpha,\nu}(t,\tau) \\
 & \quad  \cdot \int_0^1(1-s)^4  (z_0+s(z-z_0))^\beta P_{\alpha,\beta,\ell} \left(\frac{|z_0+s(z-z_0)|^2}{\delta^2}\right)  \dd s,
\end{align*}
wobei $z=\eta(t)-\eta(\tau)$ und $z_0=\eta'(t)(t-\tau)$.
\end{Lemma}
\begin{proof}
Wir entwickeln zunächst das Argument $z(\tau)=\eta(t) - \eta(\tau)$ an der Stelle $\tau=t$ bis zur Ordnung $2$ und erhalten nach Satz \ref{STaylor}
\begin{align*}
z(\tau) &= \eta(t) - \eta(\tau) \\
&= (\eta(t)-\eta(t)) - \eta'(t)(\tau-t) - \sum_{|\gamma|=2}Q_\gamma (t,\tau) (\tau- t)^\gamma \\
&= \eta'(t)(t-\tau) - \sum_{|\gamma|=2}Q_\gamma (t,\tau) (\tau- t)^\gamma
\end{align*} 
mit glatten Funktionen $Q_\gamma$. Entwickeln wir nun $G_{\delta,l}(z)$ an der Stelle $z_0= \eta'(t)(t-\tau)$ bis zur Ordnung 3 und werten an $z=\eta(t) - \eta(\tau)$ aus, so erhalten wir
\begin{align*}
G_{\delta,\ell}(\eta(t)-\eta(\tau)) &= \left[\sum_{k=0}^3 \sum_{|\alpha|=k} D^\alpha G_{\delta,\ell}(\eta'(t)(t-\tau))\left(- \sum_{|\gamma|=2}Q_\gamma (t,\tau) (\tau- t)^\gamma\right)^\alpha \frac{1}{\alpha!}\right]  \\
& \quad + R_\ell(\eta(t)-\eta(\tau)) \\
&= \left[\sum_{k=0}^3 \sum_{|\alpha|=k}\sum_{|\nu|=2k}  D^\alpha G_{\delta,\ell}(\eta'(t)(t-\tau)) (t-\tau)^\nu R_{\alpha,\nu}(t,\tau)\right] +  R_\ell(\eta(t)-\eta(\tau))
\end{align*}
mit glatten Funktionen $R_{\alpha,\nu}$ und $R_\ell$.
Bevor wir die Darstellung von $D^\alpha G_{\delta,\ell}$ aus Lemma \ref{LDa} einsetzen, berechnen wir 
mit dem binomischen Satz für Multiindizes
\begin{align*}
z_0^\beta &= (\eta'(t)(t-\tau))^\beta \\
  &= (\partial_1 \eta(t)(t_1-\tau_1)+\partial_2 \eta(t)(t_2-\tau_2))^\beta \\
 &=\sum_{m \leq \beta}
\begin{pmatrix}
  \beta \\ m
\end{pmatrix}   (\partial_1 \eta(t)(t_1-\tau_1))^m (\partial_2 \eta(t)(t_2-\tau_2))^{\beta-m} \\
  &=\sum_{m \leq \beta}
\begin{pmatrix}
  \beta \\ m
\end{pmatrix}   \partial_1 \eta(t)^m \partial_2 \eta(t)^{\beta-m} (t-\tau)^{\gamma_m} \\
& = \sum_{|\gamma|=|\beta|} T_{\beta,\gamma}(\eta'(t)) (t-\tau)^\gamma,
\end{align*}
wobei $\gamma \in \N_0^2$, $\gamma_m= (|m|,|\beta-m|)\in \N_0^2$, d.h. $|\gamma_m|=|\beta|$ und $T_{\beta,\gamma}$ entsprechende glatte Koeffizienten-Funktionen aus dem binomischen Satz sind.

Somit erhalten wir mit Lemma \ref{LDa}  
\begin{align*}
G_{\delta,l}(\eta(t)-\eta(\tau)) &=\left[\sum_{k=0}^3 \sum_{|\alpha|=k}\sum_{|\nu|=2k}  \sum_{|\beta|\leq k+1}\frac{1}{\delta^{|\beta|+|\alpha|+2}}z_0^\beta P_{\alpha,\beta,\ell} \left(\frac{|z_0|^2}{\delta^2}\right) (t-\tau)^\nu R_{\alpha,\nu}(t,\tau)\right] \\
&\quad +  R_\ell(\eta(t)-\eta(\tau)) \\
&=\sum_{(k,\alpha,\nu,\beta,\gamma)\in J} S_{\alpha,\beta_,\nu,\gamma,\ell}(t,\tau) + R_\ell(\eta(t)-\eta(\tau))
\end{align*}
wobei $J=\{(k,\alpha,\nu,\beta,\gamma)|k\in \{0,1,2,3\}, \alpha , \beta \in \N_0^3, \nu,\gamma \in \N_0^2 \textrm{ mit } |\alpha|=k, |\nu|=2k, |\beta| \leq k+1, |\gamma|=|\beta|\}$ und
\[
S_{\alpha,\beta_,\nu,\gamma,\ell}(t,\tau) = \frac{(t-\tau)^{\nu+\gamma}}{\delta^{|\beta|+|\alpha|+2}}T_{\beta,\gamma}(\eta'(t))P_{\alpha,\beta,\ell} \left(\frac{| \eta'(t)(t-\tau)|^2}{\delta^2}\right)R_{\alpha,\nu}(t,\tau),
\]
ist.
%wobei die Darstellung von $z_0^\beta$ mit $\gamma_m= (|m|,|\beta-m|)\in \N_0^2$, d.h. $|\gamma_m|=|\beta|$ aus dem binomischen Satz 
%\begin{align*}
%z_0^\beta &= (\eta'(t)(t-\tau))^\beta \\
%  &= (\partial_1 \eta(t)(t_1-\tau_1)+\partial_2 \eta(t)(t_2-\tau_2))^\beta \\
% &=\sum_{m \leq \beta}
%\begin{pmatrix}
%  \beta \\ m
%\end{pmatrix}   (\partial_1 \eta(t)(t_1-\tau_1))^m (\partial_2 \eta(t)(t_2-\tau_2))^{\beta-m} \\
%  &=\sum_{m \leq \beta}
%\begin{pmatrix}
%  \beta \\ m
%\end{pmatrix}   \partial_1 \eta(t)^m \partial_2 \eta(t)^{\beta-m} (t-\tau)^{\gamma_m} \\
%& = \sum_{|\gamma|=|\beta|} T_{\beta,\gamma}(\eta'(t)) (t-\tau)^\gamma
%\end{align*}folgt, mit entsprechenden Koeffizienten-Funktionen $T_{\beta,\gamma}$ aus dem binomischen Satz.

Für den Restterm $R_\ell$ liefert der Satz von Taylor, zusammen mit der Entwicklung von $\eta(t)-\eta(\tau)$ von oben, 
\begin{align*}
R_\ell(\eta(t)-\eta(\tau))&=\sum_{|\alpha|=4} \left[ \sum_{|\gamma|=2} (-Q_\gamma (t,\tau) (\tau- t)^\gamma)^\alpha \frac{4}{\alpha!} \right]\\
& \quad \cdot\int_0^1(1-s)^4 D^\alpha G_{\delta,\ell}\left(\eta'(t)(t-\tau) - s \sum_{|\gamma|=2}Q_\gamma (t,\tau) (\tau- t)^\gamma\right)\dd s 
 \\
 &= \sum_{|\alpha|=4}\sum_{|\nu|=8}\sum_{|\beta|\leq |\alpha|+1}\frac{4 (t-\tau)^\nu  }{\alpha! \delta^{|\beta|+|\alpha|+2}}R_{\alpha,\nu}(t,\tau) \\
 & \quad  \cdot \int_0^1(1-s)^4  (z_0+s(z-z_0))^\beta P_{\alpha,\beta,\ell} \left(\frac{|z_0+s(z-z_0)|^2}{\delta^2}\right)  \dd s.
\end{align*}

\end{proof}
Für die Entwicklung des ganzen Integranden von $A^{(0)}_{j,\delta}\varphi(t)$ brauchen wir noch die Entwicklung von
\[
(\partial_1 \eta (\tau) \times \partial_2 \eta(\tau))\chi(t-\tau)(\varphi(\tau)- \varphi(t))
\]
Für $\varphi$ erhalten wir mit der Taylorentwicklung am Punkt $t \in \R^2$ bis zur Ordnung 2 aus Satz \ref{STaylor}
\[
\varphi(\tau) = \varphi(t)+  (\tau-t) \cdot (\grad\varphi(t) + R_\varphi(t,\tau)) \fa \tau \in \R^2
\] 
mit \[
R_\varphi(t,\tau)= \int_0^1 (1-\xi) \varphi''(t+\xi(\tau-t))(\tau-t) \dd \xi \in O(|\tau - t|^2), \quad \tau \in \R^2,
\] für $|\tau- t| \to 0$.
Insgesamt erhalten wir damit eine Entwicklung des ganzen Integranden von $A^{(0)}_{j,\delta}\varphi(t) ,t\in \R^2$.
\begin{Lemma} \label{LTint}
Für $l \in \{1,2,3\}$ und $t,\tau \in \R^2$, $t\neq \tau$ gilt
\begin{align*}
& \quad (\partial_1 \eta (\tau) \times \partial_2 \eta(\tau))_l G_{\delta,l}(|\eta(t)-\eta(\tau)|) \chi(t-\tau)(\varphi(\tau)- \varphi(t)) \\
&= \left(\sum_{(k,\alpha,\nu,\beta,\gamma)\in J} S_{\alpha,\beta_,\nu,\gamma,\ell}(t,\tau) + R_\ell(\eta(t)-\eta(\tau))\right)\\
&\quad\cdot  \left((\partial_1 \eta (\tau) \times \partial_2 \eta(\tau))_l\chi(t-\tau)(\tau-t) \cdot\left[\grad\varphi(t) + R_\varphi(t,\tau) \right] \right),
\end{align*}
wobei $J$ und $S_{\alpha,\beta_,\nu,\gamma,\ell}$ wie in Lemma \ref{LGdeltaTaylor} definiert sind.
%mit $J=\{(k,\alpha,\nu,\beta,\gamma)|k\in \{1,2,3\}, \alpha , \beta \in \N_0^3, \nu,\gamma \in \N_0^2 \textrm{ mit } |\alpha|=k, |\nu|=2k, |\beta| \leq k+1, |\gamma|=|\beta|\}$ und
%\[
%S_{\alpha,\beta_,\nu,\gamma,\ell}(t,\tau) = \frac{(t-\tau)^{\nu+\gamma}}{\delta^{|\beta|+|\alpha|+2}}T_{\beta,\gamma}(\eta'(t)) P_{\alpha,\beta,\ell} \left(\frac{| \eta'(t)(t-\tau)|^2}{\delta^2}\right)R_{\alpha,\nu}(t,\tau)
%\]
\end{Lemma}
Definieren wir nun die zum lokalisierten Operator zugehörige Diskretisierung
\[
S^{(0)}_{j,\delta}\varphi(t) := \frac{\pi^2}{N^2}\sum_{\nu\in\Z_N^2}  K_{j,\delta}(t,t^{(\nu)})\chi(t-t^{(\nu)}) \left(\varphi(t^{(\nu)})- \varphi(t)\right) \fa t \in Q,
\]
so können wir den zum lokalisierten Operator zugehörigen Diskretisierungsfehler abschätzen.
\begin{Satz} \label{Sdiskeps}
Es seien $ 1\geq \delta, 1 \geq h>0$ mit $\delta/h = \rho \geq \rho_0$ und $\varphi \in C_{\per}^{15}(\R^2)$. Dann gilt für $j=1,2$ und alle $t \in Q$
\[
|A^{(0)}_{j,\delta}\varphi(t) - S^{(0)}_{j,\delta}\varphi(t)| \leq ||\varphi||_{15,\infty}\left( \sum_{k=2}^4 c_k^{(j)}( \rho) h^k + C h^5 \right)
\]
mit von $\rho$ abhängigen Konstanten $c_k^{(j)}(\rho)\geq 0$ und einer von $\delta$ und $\rho$ unabhängigen Konstante $C>0$.
\end{Satz}
\begin{proof}
Um Lemma \ref{epsdisk} anzuwenden transformieren wir das Integral zunächst mit $\xi = t- \tau$ für $t,\tau \in \R^2$, d.h.
\begin{align*}
A^{(0)}_{j,\delta}\varphi(t) &=\int_{t+Q} K_{j,\delta}(t,\tau)\chi(t-\tau)(\varphi(\tau)- \varphi(t)) \dd \tau
\\
&= \int_{Q} K_{j,\delta}(t,t-\xi)\chi(\xi)(\varphi(t-\xi)- \varphi(t)) \dd \xi \\
&= \int_{\R^2} \sum_{\ell=1}^3(\partial_1 \eta (t- \xi) \times \partial_2 \eta(t- \xi))_\ell G_{\delta,l}(\eta(t)-\eta(t - \xi)) \chi(\xi)(\varphi(t- \xi)- \varphi(t)) \dd \xi.
\end{align*}
Seien von nun an $\xi, t \in \R^2$ und $J$ und $S_{\alpha,\beta_,\nu,\gamma,\ell}$ wie in Lemma \ref{LGdeltaTaylor}.\\
Nach Lemma \ref{LTint} lässt sich der Integrand vollständig als Summe von Termen der Form 

\begin{enumerate}[(i)]
\item 
$S_{\alpha,\beta_,\nu,\gamma,\ell}(t,t-\xi)\left((\partial_1 \eta (t-\xi) \times \partial_2 \eta(t-\xi))_\ell\chi(\xi)\xi \cdot \left[ \grad\varphi(t) + R_\varphi(t,t-\xi) \right] \right)$ \\
 mit $(k,\alpha,\nu,\beta,\gamma) \in J$, $\ell=1,2,3$ oder
\item $\sum_{j=1}^2 \xi^{e_j} R_\ell(\eta(t)-\eta(t-\xi))\left((\partial_1 \eta (t-\xi) \times \partial_2 \eta(t-\xi))_\ell \chi(\xi)\left[\varphi_j(t) + R_{\varphi,j}(t,t-\xi) \right] \right) $ \\
mit $\ell=1,2,3$, $R_{\varphi}= (R_{\varphi,1},R_{\varphi,2})$
\end{enumerate}
darstellen.

Wir rechnen zunächst die Fehlerbeiträge der Terme der Form (i) nach.
Dazu multiplizieren wir noch das Skalarprodukt $\xi \cdot \left[\grad\varphi(t) + R_\varphi(t,t-\xi) \right]$ aus und erhalten
\begin{align*}
&\quad S_{\alpha,\beta_,\nu,\gamma,\ell}(t,t-\xi)\left((\partial_1 \eta (t-\xi) \times \partial_2 \eta(t-\xi))_\ell\chi(\xi)\xi \cdot\left[ \grad\varphi(t) + R_\varphi(t,t-\xi) \right] \right) \\
&=\frac{\xi^{\nu+\gamma}}{\delta^{|\beta|+|\alpha|+2}}T_{\beta,\gamma}(\eta'(t)) P_{\alpha,\beta,\ell} \left(\frac{| \eta'(t)\xi|^2}{\delta^2}\right)R_{\alpha,\nu}(t,t-\xi) \\
& \quad \cdot \left((\partial_1 \eta (t-\xi) \times \partial_2 \eta(t-\xi))_\ell\chi(\xi)\xi \cdot \left[\grad\varphi(t) + R_\varphi(t,t-\xi) \right] \right) \\
&=\sum_{j=1}^2 \frac{\xi^{\nu+\gamma+e_j}}{\delta^{|\beta|+|\alpha|+2}}T_{\beta,\gamma}(\eta'(t)) P_{\alpha,\beta,\ell} \left(\frac{| \eta'(t)\xi|^2}{\delta^2}\right)R_{\alpha,\nu}(t,t-\xi) \\
& \quad\cdot \left((\partial_1 \eta (t-\xi) \times \partial_2 \eta(t-\xi))_\ell\chi(\xi)\left[\varphi_j(t) + R_{\varphi,j}(t,t-\xi) \right] \right)
\end{align*}
für $(k,\alpha,\nu,\beta,\gamma) \in J$.  \\
%und
%\[
%\sum_{j=1}^2 \xi^{e_j} R_\ell(\eta(t)-\eta(t-\xi))\left((\partial_1 \eta (t-\xi) \times \partial_2 \eta(t-\xi))_\ell \chi(\xi)\left[\varphi_j(t) + R_\varphi(t,t-\xi) \right] \right) 
%\]
%darstellen. \\
   Für $j=1,2$ setzen wir nun $q=|\beta|+|\alpha|+2$, $\lambda = \nu + \gamma + e_j $, 
\[
P(z)= P_{\alpha,\beta,\ell} \left(|\eta'(t)z|^2 \right) \fa z \in \R^2
\]
und 
\begin{align*}
F(\xi) &= T_{\beta,\gamma}(\eta'(t)) R_{\alpha,\nu}(t,t-\xi) \\
& \quad \cdot \left((\partial_1 \eta (t-\xi) \times \partial_2 \eta(t-\xi))_\ell\chi(\xi)\left[ \varphi_j(t) + R_\varphi(t,t-\xi) \right] \right).
\end{align*}
Damit liefert Lemma \ref{epsdisk}
\begin{align*}
\bigg|\int_{\R^2} \delta^{-q}F(\tau)\tau^\lambda P\left(\frac{\tau}{\delta}\right) \dd \tau &- h^2 \delta^{-q} \sum_{\nu \in \Z^2}F(t^{(\nu)})t^{(\nu)^\lambda} P\left(\frac{t^{(\nu)}}{\delta}\right)\bigg| \\
&\leq C( \rho_0^{-p}h^{|\lambda|+2-q}+\rho_0^{-q}h^{p-q})||F||_{p,\infty} \\
&= C( \rho_0^{-p}h^{|\nu|+1-|\alpha|}+\rho_0^{-q}h^{p-q})||F||_{p,\infty} \\
&= C( \rho_0^{-p}h^{k+1}+\rho_0^{-q}h^{p-q})||F||_{p,\infty},
\end{align*}
für alle $p$, so dass $F \in C_0^p(\R^2)$.
%\todo{Nachrechnen}
 \\
Für $k=0$, d.h. $\alpha=0$ entwickeln wir zusätzlich die Normale in $t$ und erhalten
\[
(\partial_1 \eta (t-\xi) \times \partial_2 \eta(t-\xi)) = (\partial_1 \eta (t) \times \partial_2 \eta(t)) - \xi^{T} R_n (t-\xi)
\]
mit einer glatten Funktion $R_n: \R^2 \to \R^{3 \times 2}$. Wegen 
\[
(\partial_1 \eta (t) \times \partial_2 \eta(t)) \cdot \eta'(t) = 0
\] ist damit 
\begin{align*}
(\partial_1 \eta (t) \times \partial_2 \eta(t)) \cdot G_{\delta}(\eta'(t)\xi) &=  (\partial_1 \eta (t) \times \partial_2 \eta(t)) \cdot \left[\frac{(\cdot)}{\delta^3}\tilde P \left(\frac{|\cdot|^2}{\delta^2}\right)\right](\eta'(t)\xi) \\
&= 0
\end{align*}
mit der glatten Funktion $\tilde P$ aus dem Beweis von Lemma \ref{LDa}.\\
Das heißt die Summanden für $k=0$ enthalten alle den Faktor 
\[
- \xi^{T} R_n (t-\xi)
\] und wir erhalten für die Abschätzung durch Lemma \ref{epsdisk} eine höhere Potenz in $\xi$, d.h wir können das Lemma mit $\lambda=\nu + \beta + e_j +e_i$ mit $i,j \in\{1,2\}$ anwenden. Das Lemma \ref{epsdisk} liefert damit für $k=0$ die Abschätzung
\begin{align*}
\bigg|\int_{\R^2} \delta^{-q}F(\tau)\tau^\lambda P\left(\frac{\tau}{\delta}\right) \dd \tau &- h^2 \delta^{-q} \sum_{\nu \in \Z^2}F(t^{(\nu)})t^{(\nu)^\lambda} P\left(\frac{t^{(\nu)}}{\delta}\right)\bigg| \\
&\leq C( \rho_0^{-p}h^{k+2}+\rho_0^{-q}h^{p-q})||F||_{p,\infty}.
\end{align*}

Für die noch fehlenden Terme der Form (ii)
\[
\sum_{j=1}^2 \xi^{e_j} R_\ell(\eta(t)-\eta(t-\xi))\left((\partial_1 \eta (t-\xi) \times \partial_2 \eta(t-\xi))_\ell \chi(\xi)\left[\varphi_j(t) + R_\varphi(t,t-\xi) \right] \right) 
\] mit $l=1,2,3$ nutzen wir die Darstellung von $R_l$ aus Lemma \ref{LGdeltaTaylor}. Wie im Beweis von Lemma \ref{LGdeltaTaylor} lässt sich mit der binomischen Formel nachrechnen, dass wir nur Summanden der Form
\begin{align*}
&\sum_{j=1}^2 \frac{\xi^{\nu+\gamma+e_j}}{\delta^{|\beta|+|\alpha|+2}}\widetilde P_{\alpha,\beta,\eta,l}(\xi)R_{\alpha,\nu}(t,t-\xi) \\
&\quad \cdot \left((\partial_1 \eta (t-\xi) \times \partial_2 \eta(t-\xi))_\ell\chi(\xi)\left[\varphi_j(t) + R_\varphi(t,t-\xi) \right] \right)
\end{align*}
mit $|\alpha|=4, |\nu|=8, |\beta|\leq 5, |\gamma|=|\beta|$ haben. Wieder liefert Lemma \ref{epsdisk} Fehlerabschätzungen der Form
\begin{align*}
\bigg|\int_{\R^2} \delta^{-q}F(\tau)\tau^\lambda P\left(\frac{\tau}{\delta}\right) \dd \tau &- h^2 \delta^{-q} \sum_{\nu \in \Z^2}F(t^{(\nu)})t^{(\nu)^\lambda} P\left(\frac{t^{(\nu)}}{\delta}\right)\bigg| \\
&\leq C( \rho_0^{-p}h^{k+1}+\rho_0^{-q}h^{p-q})||F||_{p,\infty}.
\end{align*} mit $k=4$, $q= |\alpha| +|\beta| +2 \leq 4+5+2 = 11$. \\
 Die Aussage des Satzes erhalten wir nun, indem wir $h \leq \delta / \rho_0$ ausnutzen und die durch Lemma \ref{epsdisk} erhaltenen Abschätzungen summieren. Dabei ist zu beachten, dass wegen $\varphi \in C^{15}(\R^2)$ auch $p \geq 14$ und $q \leq  9$ für jeden Summanden gewählt werden kann und damit 
$h^{p-q} \leq h^5$ ist. 
\end{proof}
\begin{Bemerkung} \label{Bck12}
Es ist zu beachten, dass im obigen Beweis nur die Terme der Form (i), d.h.
\[
S_{\alpha,\beta_,\nu,\gamma,\ell}(t,t-\xi)\left((\partial_1 \eta (t-\xi) \times \partial_2 \eta(t-\xi))_\ell\chi(\xi)\left[\xi \cdot \grad\varphi(t) + R_\varphi(t,t-\xi) \right] \right)
\] 
\\
 mit $(k,\alpha,\nu,\beta,\gamma) \in J$, $\ell=1,2,3$ einen Beitrag zu den Fehlern proportional zu $h^k$, $k \leq 4$ leisten, falls $\varphi \in C_{\per}^{15}(\R^2)$. Die Aussage von Satz \ref{Sdiskeps} gilt tatsächlich sogar für
 $\varphi \in C_{\per}^{12}(\R^2)$, allerdings liefern dann auch die Terme der Form (ii) aus dem Beweis einen Fehler proportional zu $h^2$. Zur genaueren Bestimmung der Konstanten $c_k^{(j)}(\rho)$ im folgenden Abschnitt werden wir verwenden, dass die $c_k^{(j)}(\rho), k=2,3,4$ nicht von den Summanden der Form (ii) abhängen, falls $\varphi$ genügend glatt ist.
\end{Bemerkung}
Um den gesamten Fehler der Diskretisierung
\begin{align*}
\varepsilon &= |(A_{j,\delta}-A_{j,\delta,N})\varphi(t)| \\
& \leq |A^{(0)}_{j,\delta}\varphi(t) - S^{(0)}_{j,\delta}\varphi(t)| +|A^{(1)}_{j,\delta}\varphi(t) - S^{(1)}_{j,\delta}\varphi(t)|,
\end{align*}
wobei
\[
S^{(1)}_{j,\delta}\varphi(t) := \frac{\pi^2}{N^2}\sum_{\nu\in\Z_N^2}  K_{j,\delta}(t,t^{(\nu)})(1-\chi(t-t^{(\nu)})) \left(\varphi(t^{(\nu)})- \varphi(t)\right) \fa t \in Q,
\]
abzuschätzen, fehlt noch der zu $A^{(1)}_{j,\delta}\varphi(t)$ gehörige Fehler. Da der Integrand mindestens die gleiche Glattheit wie $\varphi$ besitzt und periodisch ist, können wir folgende Abschätzung für die 1D-Trapezregel verwenden.
\begin{Satz} \label{Trapez2}
Sei $g \in C_{\per}^m(-\pi,\pi)$. Dann gilt für den Fehler der Trapezregel
\[
\left| \int_{-\pi}^{\pi}f(x) \dd x - \frac{\pi}{N} \sum_{i \in \Z_N} g\left(\frac{\pi i }{N}\right)\right| \leq \frac{C}{N^m}||g^{(m)}||_\infty
\]
für alle $N \in \N$, wobei $\Z_N= \{-N+1, \dots, N \}$.
\end{Satz}
\begin{proof}
Direkte Folgerung aus Theorem 12.6 in \cite{kress}.
\end{proof}

\begin{Satz} \label{SDiskeps2}
Es sei $ 1 \geq h>0$ und $\varphi \in C_{\per}^{5}(\R^2)$. Dann ist
\[
|A^{(1)}_{j,\delta}\varphi(t) - S^{(1)}_{j,\delta}\varphi(t)| \leq C ||\varphi||_{5,\infty} h^5
\]
\end{Satz}
\begin{proof}
Wir wenden das 1D Resultat aus Satz \ref{Trapez2} an und verallgemeinern dieses mit dem Satz von Fubini.
Sei $f \in C_{\per}^m(Q)$, $N \in \N$ und $t^{(i)}= i\frac{\pi}{N}$ für $i \in \Z_N= \{-N+1, \dots, N \}$ und $t^{(\nu)}=\nu \frac{\pi^2}{N^2}$ für $\nu \in \Z_N^2$. Dann gilt
\begin{align*}
\int_Q f(\tau)\dd \tau 
&= \int_{-\pi}^\pi \int_{-\pi}^\pi f(\tau_1,\tau_2) \dd \tau_1 \tau_2 \\
&= \int_{-\pi}^\pi \left( \int_{-\pi}^\pi f(\tau_1,\tau_2) \dd \tau_1 - \frac{\pi}{N} \sum_{i \in \Z_N} f(t^{(i)},\tau_2) \right) \dd \tau_2  \\
& \quad + \int_{-\pi}^\pi \frac{\pi}{N} \sum_{i \in \Z_N} f(t^{(i)},\tau_2) \dd \tau_2.
\end{align*} 
Nach Voraussetzung ist für $\tau_2 \in [-\pi,\pi]$ die Funktion $f(\cdot,\tau_2) \in C_{\per}^m([-\pi,\pi])$ und es folgt 
\begin{align*}
&\quad \left|\int_Q f(\tau)\dd \tau - \frac{\pi^2}{N^2} \sum_{\nu \in \Z_N^2} f(t^{(\nu)})\right| \\
&\leq \left|\int_{-\pi}^\pi \left( \int_{-\pi}^\pi f(\tau_1,\tau_2) \dd \tau_1 - \frac{\pi}{N} \sum_{i \in \Z_N} f(t^{(i)},\tau_2) \right) \dd \tau_2 \right| \\
& \quad + \left|\int_{-\pi}^\pi \frac{\pi}{N} \sum_{i \in \Z_N} f(t^{(i)},\tau_2) \dd \tau_2 - \frac{\pi^2}{N^2} \sum_{\nu \in \Z_N^2} f(t^{(\nu)}) \right| \\
& \leq \left|\int_{-\pi}^\pi \frac{C_1}{N^m}||f(\cdot,\tau_2)||_{\infty,m} \dd \tau_2 \right| +  \frac{C_2}{N^m}||f||_{\infty,m} \\
& \leq \frac{C}{N^m}||f||_{\infty,m}
\end{align*}
für Konstanten $C,C_1,C_2>0$.\\
Wenden wir die eben gezeigte Abschätzung auf 
\[
f(\tau)= K_{j,\delta}(t,\tau)(1-\chi(t-\tau)) \left(\varphi(\tau)- \varphi(t)\right), \qquad \tau \in Q,
\] an, so folgt die Behauptung.




%Das Lemma \ref{Lepsdisk} lässt sich leider nicht ohne Weiteres anwenden, da wir nicht über ganz $\R^2$ integrieren. Stattdessen nutzen wir die allgemeineren Resultate für trigonometrische Interpolationsoperatoren für Sobolevräume aus \cite{Arens}. Wir nutzen Theorem 5.9 aus \cite{Arens} mit $N_1=N_2=N$ und $A=I^{(1)}$. Dabei ist $I^{(1)}=A^{(1)}_{j,\delta}$, $A_{\mathcal{N}}=S^{(1)}_{j,\delta}$ und $F^{(1)}$ der glatte Anteil des Integranden, d.h. 
%\[
%F^{(1)}(t,\tau)=K_{j,\delta}(t,\tau)(1-\chi(t-\tau)) \fa t,\tau \in Q.
%\] 
%Die im Theorem verwendeten Sobolevnormen sind für $s \in \N$ und $\varphi \in C^s(Q)$ definiert durch
%\begin{align*}
%||\varphi||_s:=\left( \sum_{|\mu|\leq s} \int_Q |D^\mu \varphi(\tau)|^2 \dd \tau \right)^{1/2} .
%\end{align*}
%und der Sobolevraum $H_Q^s$ als Vervollständigung von $C_{\per}^s(Q)$ unter dieser Norm,
%\[
%H_Q^s:=\overline{C^s(Q)}^{||\cdot||_s}
%\]
%(vergleiche Kapitel 2.1 in \cite{Arens}).
%Das Theorem liefert für $m=?$, $s=
%\begin{align*}
%||(A-A_{\mathcal{N}})\varphi||_{t+1} \leq C N^{t-s} ||\varphi||_s
%\end{align*} 
%für alle $\varphi \in H^s$ und $N \in \N$.

\end{proof}

Im nächsten Abschnitt wollen wir die Konstanten $c_k^{(j)}( \rho)$ aus Satz \ref{Sdiskeps} genauer abschätzen, für den Fall, dass sogar $\varphi \in C_{\per}^\infty(\R^2)$ gilt.
\subsection{Bestimmung der Konstanten} \label{chaFourierKonstanen}
Ist die Abschätzung aus Lemma \ref{epsdisk} scharf, so lässt sich der führende Koeffizient mithilfe der Fouriertransformation bestimmen. Wie das geht, zeigt das folgende Lemma.
\begin{Lemma} \label{LFourierconst}
Es sei $h>0$, $\rho \geq \rho_0 >0$ und $p \in \N$. Für alle $\delta >0$ sei $F_\delta \in C^2(\R^2)$ mit $h^{-p} F_{\rho h}(\tau) = F_\rho(\tau/h)$ für alle $h>0,\rho \geq \rho_0$ und $\tau \in \R^2$. Ferner gelte für alle $f \in C_0^\infty(\R^2)$
\[
\int_{\R^2} F_{\rho h}(\tau)f(\tau) \dd \tau - h^2 \sum_{\nu \in \Z^2} F_{\rho h } (t^{(\nu)})f(t^{(\nu)}) =c h^{p+2} + O(h^{p+3}) \qquad (h \to 0).
\]
Dann gilt 
\[
c=(2 \pi)^2 f(0) \sum_{\mu \in \Z^2 \backslash \{0\} } \F F_\rho(2 \pi \mu)
\]
\end{Lemma}
\begin{proof}
Nach Voraussetzung ist  
\[
c = \lim_{h \to 0} h^{-p-2} \left( \int_{\R^2} F_{\rho h}(\tau)f(\tau) \dd \tau - h^2 \sum_{\nu \in \Z^2} F_{\rho h } (t^{(\nu)})f(t^{(\nu)})\right)
\]
Durch Substitution $\tau= h\sigma$, d.h. $|\det\frac{\partial \tau}{\partial \sigma}|=h^2$ erhalten wir mit dem Transformationssatz und der Eigenschaft von $F_\rho$ aus der Voraussetzung
\begin{align*}
c &= \lim_{h \to 0} h^{-p-2} \left( \int_{\R^2} F_{\rho h}(\tau)f(\tau) \dd \tau - h^2 \sum_{\nu \in \Z^2} F_{\rho h } (t^{(\nu)})f(t^{(\nu)})\right) \\
&= \lim_{h \to 0} h^{-p} \left( \int_{\R^2} F_{\rho h}(h \sigma)f(h \sigma) \dd \sigma - \sum_{\nu \in \Z^2} F_{\rho h } (t^{(\nu)})f(t^{(\nu)})\right) \\
&=\lim_{h \to 0} \left( \int_{\R^2} F_{\rho}(\sigma)f(h \sigma) \dd \sigma - \sum_{\nu \in \Z^2} F_{\rho} (\nu)f(h\nu)\right) \\
&=\lim_{h \to 0} \left( 2 \pi \F (F_\rho f(h \cdot))(0)  - \sum_{\nu \in \Z^2} F_{\rho} (\nu)f(h\nu)\right).
\end{align*}
Im Folgenden werden wir die Eigenschaften der Fouriertransformation aus Satz \ref{Feig} benutzen um die gewünschte Darstellung für $c$ zu erhalten. 
Zunächst wenden wir die Poisson'sche Summenformel auf die Summe an und erhalten
\begin{align*}
c &= 2 \pi \lim_{h \to 0} \left(\F [F_\rho f(h \cdot)](0) - \sum_{\nu \in \Z^2} \F [F_{\rho}f(h \cdot)](2 \pi \nu) \right) \\
&= - 2\pi \lim_{h \to 0}\sum_{\nu \in \Z^2 \backslash \{0\}} \F [F_{\rho}f(h \cdot)](2 \pi \nu).
\end{align*}
Der Faltungssatz erlaubt uns die Fouriertransformation des Produkts $F_{\rho}f(h \cdot)$ als Faltung darzustellen, womit sich zusammen mit der Skalierungseigenschaft
\begin{align*}
c &= - 2\pi \lim_{h \to 0}\sum_{\nu \in \Z^2 \backslash \{0\}} \int_{\R^2} \F F_{\rho}(\xi) \F [f(h \cdot)](2 \pi \nu-\xi)\dd \xi \\
&= - 2\pi \lim_{h \to 0}\sum_{\nu \in \Z^2 \backslash \{0\}}\frac{1}{h^2} \int_{\R^2} \F F_{\rho}(\xi) \F f\left(\frac{2 \pi \nu-\xi}{h}\right)\dd \xi
\end{align*}
ergibt. Wir benutzen den Satz von Lebesgue bezüglich des Zählmaßes um $\lim_{h \to 0}$ und Summation zu vertauschen. Nach Satz \ref{Feig} \eqref{Feig1} existiert der Grenzwert 
\[
\lim_{h \to 0}\frac{1}{h^2} \int_{\R^2} \F F_{\rho}(\xi) \F f\left(\frac{2 \pi \nu-\xi}{h}\right)\dd \xi = 2 \pi \F F_{\rho}(2 \pi \nu) \F \F f(0)
\]
und wegen Satz \ref{Feig} \eqref{Feig2} ist $\F F_{\rho}(2 \pi \nu) \leq \frac{C}{(2 \pi \nu)^2}$ für ein $C>0$ und alle $\nu \in \Z^2$ mit $|\nu|>R$ für ein $R>0$, d.h. es gibt eine summierbare Majorante und Grenzwert und Reihe dürfen vertauscht werden. Wegen $\F \F f(0) = f(0)$ (siehe Satz \ref{Feig} \eqref{Feig3}) folgt damit die Behauptung.
\end{proof}
Um die Darstellung für die Konstante aus dem Lemma zu bestimmen ist insbesondere die Fouriertransformation einer entsprechenden Funktion $F_\rho$ notwendig, die wir im folgenden berechnen wollen. Dazu sei im Folgenden stets $\delta,h>0$ und $\delta/h= \rho \geq \rho_0 >0$. Außerdem werden wir von nun an zusätzlich $\varphi \in C_{\per}^\infty(\R^2)$ voraussetzen, damit die Voraussetzungen an die Glattheit von $f$ aus dem Lemma erfüllt sind.

Wie aus dem Beweis von Satz \ref{Sdiskeps} ersichtlich, stammen die Terme $c_k^{(j)}(\rho) h^k$, $k=2,3,4$ in der Fehlerabschätzung alle von den Summanden
\begin{align*}
&&S_{\alpha,\beta_,\nu,\gamma,\ell}(t,t-\xi)\left((\partial_1 \eta (t-\xi) \times \partial_2 \eta(t-\xi))_\ell\chi(\xi)\left[\xi \cdot \grad\varphi(t) + R_\varphi(t,t-\xi) \right] \right),
\end{align*}  
falls $\varphi \in C_{\per}^\infty(\R^2)$ gilt.
Die Terme $S_{\alpha,\beta_,\nu,\gamma,\ell}$ ergeben sich durch Einsetzen der Darstellung aus Lemma \ref{LDa} in die Taylorentwicklung
\begin{align*}
G_{\delta,l}(\eta(t)-\eta(t-\xi)) &= \sum_{k=0}^3 \sum_{|\alpha|=k} D^\alpha G_{\delta,l}(\eta'(t)(\xi))(- \sum_{|\gamma|=2}Q_\gamma (t,t-\xi) (-\xi)^\gamma)^\alpha \frac{1}{\alpha!} + R(\eta(t)-\eta(t-\xi)) \\
&= \sum_{k=0}^3 \sum_{|\alpha|=k}\sum_{|\nu|=2k}  D^\alpha G_{\delta,l}(\eta'(t)\xi) \xi^\nu R_{\alpha,\nu}(t,t-\xi) +  R(\eta(t)-\eta(t-\xi))
\end{align*} (vergleiche Beweis von Lemma \ref{LGdeltaTaylor}). Damit brauchen wir zum Anwenden von Lemma~\ref{LFourierconst} die Fouriertransformation von
\[
F_\rho(\xi):= D^\alpha G_{\delta,l}(\eta'(t)\xi) \xi^\nu \fa \xi \in \R^2
\]
für alle Multiindizes $\alpha \in \N_0^3$ und $\nu\in \N_0^2$ mit $|\alpha|=k , |\nu|=2k+1, k=0,1,2,3$. Die um eins erhöhte Potenz in $\xi^\nu$ erhalten wir dabei aus der Entwicklung von $\varphi$, wie wir das im Beweis von Satz \ref{Sdiskeps} gesehen haben. Für $\alpha=0$ hatten wir zusätzlich durch die Entwicklung der Normalen gezeigt, dass der Fehlerbeitrag höchstens $C_1 h^2+C_2 h^5$ ist, mit Konstanten $C_1,C_2>0$. Tatsächlich werden wir in Lemma \ref{Lck} aber sehen, dass wir eine von $\alpha$ und der gewählten Glättungsfunktion unabhängige obere Schranke für die $c_k^{(j)}(\rho), k=2,3,4$ finden.
\\
Wir betrachten zunächst die Funktion 
\[
\widetilde G_\delta (z):= \frac{\erf\left(\frac{|z|}{\delta}\right)}{4 \pi |z|}
\]
für $z \in\R^3$, wobei wir $\widetilde G_\delta (0):=0$ definieren. 
\paragraph{Vorüberlegung:} Unsere Kernfunktion lässt sich nach Satz \ref{k1ddef} für $x,y \in \R^3, x \neq y$ schreiben als
\begin{align*}
K_{1_,\delta}(x,y) &= n(y) \cdot \grad_y \left[ \Phi (x,y) \erf \left(\frac{|x-y|}{\delta}\right) \right] \\
&= n(y) \cdot \grad \widetilde G_\delta( |x-y|),
\end{align*} bzw. 
\begin{align*}
K_{2_,\delta}(x,y) &=  n(y) \cdot \grad_y \left[ \Phi (x,y) (\erf \left(\frac{|x-y|}{\delta}\right) +\frac{2}{3\sqrt{\pi}}\frac{|x-y|}{\delta}e^{-\frac{|x-y|^2}{\delta^2}}) \right] \\
&=n(y) \cdot  \left( \grad \widetilde G_\delta( |x-y|) + \frac{2}{3 \sqrt{\pi} \delta } e^{-\frac{|x-y|^2}{\delta^2}}\right).
\end{align*}
Das heißt \begin{align}
G_{\delta}(z)= \grad \widetilde G_\delta(z)
\end{align}
 bei der Glättung mit der Glättungsfunktion $s_1$ und 
\begin{align} \label{EqGdeltas2}
G_{\delta}(z) = \grad \left(\widetilde G_\delta(z) + \frac{2}{3 \sqrt{\pi} \delta } e^{-\frac{|z|^2}{\delta^2}} \right)
\end{align}
für die Glättungsfunktion $s_2$ für $z \in \R^3$. Dass die Gleichheit auch für $z=0$ gilt, sieht man beispielsweise mit Lemma \ref{LDa}, da die Singularität durch die Multiplikation mit $\erf$ gehoben wird.
%\begin{Lemma} \label{LFGdelta}
%Die dreidimensionale Fouriertransformation von $\widetilde G_\delta$ ist für $z\in \R^3$ gegeben durch
%\[
%\F \widetilde G_\delta(z)= (2 \pi)^{-\frac{3}{2}}\frac{e^{-\frac{\delta^2 |z|^2}{4}}}{|z|^2}
%\]
%\end{Lemma}
%\begin{proof}
%Siehe Lemma III.13 in \cite{Collet}.
%\end{proof}
%Zusätzlich werden wir die 1D-Fouriertransformation der Funktion $x \mapsto \sqrt{2\pi}\frac{e^{-\gamma^2 x^2}}{a^2+x^2}$ brauchen.
%\begin{Lemma}\label{LFGdelta2}
%Sei $a,\gamma \in \R_{>0}$. Dann gilt für $\xi \in \R$
%\[
%\int_\R \frac{e^{-\gamma^2 x^2}}{a^2+x^2}e^{-ix \xi} \dd x = \frac{\pi}{2a}e^{a^2 \gamma^2} 
%\left[e^{a\xi} \erfc\left(\frac{\xi}{2 \gamma}+a \gamma \right) + e^{-a \xi} \erfc \left( -\frac{\xi}{2 \gamma}+a \gamma \right)
%\right]
%\]
%und $\xi \mapsto \int_\R \frac{e^{-\gamma^2 x^2}}{a^2+x^2}e^{-ix \xi} \dd x \in L^1(\R)$.
%\end{Lemma}
%\begin{proof}
%Siehe Lemma III.12 in \cite{Collet}.
%\end{proof}
Um die dreidimensionale Fouriertransformation von $\widetilde G_\delta$ verwenden zu können betrachten wir die Abbildung 
\begin{align*}
K_{\delta,\alpha}^{(\ell)}(\xi,b)
&:= D^\alpha G_{\delta,\ell}(\eta'(t) \xi - b n(t))
\end{align*}
für $\ell=1,2,3$ und  $\xi\in \R^2,b\in \R$. 
Damit haben die zu behandelnden Terme die Darstellung
$F_\rho(\xi) = K_{\delta,\alpha}^{(\ell,)}(\xi,0)\xi^\nu$ mit 
\[
K_{\delta,\alpha}^{(\ell)}(\xi,b)=[D^{\alpha+e_\ell}\widetilde{G}_\delta](\eta'(t) \xi - b n(t))
\]
für die Glättung mit $s_1$ und
\[
K_{\delta,\alpha}^{(\ell)}(\xi,b)=D^{\alpha+e_\ell}\left[z \mapsto \widetilde G_\delta(z) + \frac{2}{3 \sqrt{\pi} \delta } e^{-\frac{|z|^2}{\delta^2}}\right](\eta'(t) \xi - b n(t))
\]
für die Glättung mit $s_2$, für $\xi \in \R^2$ und $b \in \R$.

\begin{Lemma} \label{LFGdelta}
Die dreidimensionale Fouriertransformation von $\widetilde G_\delta$ ist für $z\in \R^3$ gegeben durch
\[
\F \widetilde G_\delta(z)= (2 \pi)^{-\frac{3}{2}}\frac{e^{-\frac{\delta^2 |z|^2}{4}}}{|z|^2}
\]
\end{Lemma}
\begin{proof}
Siehe Lemma III.13 in \cite{Collet}.
\end{proof}
Zusätzlich werden wir die 1D-Fouriertransformation der Funktion $x \mapsto \sqrt{2\pi}\frac{e^{-\gamma^2 x^2}}{a^2+x^2}$ brauchen.
\begin{Lemma}\label{LFGdelta2}
Sei $a,\gamma \in \R_{>0}$. Dann gilt für $\xi \in \R$
\[
\int_\R \frac{e^{-\gamma^2 x^2}}{a^2+x^2}e^{-ix \xi} \dd x = \frac{\pi}{2a}e^{a^2 \gamma^2} 
\left[e^{a\xi} \erfc\left(\frac{\xi}{2 \gamma}+a \gamma \right) + e^{-a \xi} \erfc \left( -\frac{\xi}{2 \gamma}+a \gamma \right)
\right]
\]
und $\xi \mapsto \int_\R \frac{e^{-\gamma^2 x^2}}{a^2+x^2}e^{-ix \xi} \dd x \in L^1(\R)$.
\end{Lemma}
\begin{proof}
Siehe Lemma III.12 in \cite{Collet}.
\end{proof}
Im nächsten Lemma berechnen wir zunächst die 2D-Fouriertransformation $\F [K_{\delta,\alpha}^{(\ell)}(\cdot,0)]$ für die Glättung mit $s_1$.
\begin{Lemma} \label{LFKdelta}
Sei $k \in \R^2$, $t\in \R^2$, $\alpha=(\alpha_1,\alpha_2,\alpha_3) \in \N_0^3$ mit $|\alpha| \leq 3$.
Es gibt die Matrizen $B:=(\eta'(t)^T\eta'(t))^{1/2} \in \R^{2 \times 2}$ und $B^{-1}=(\eta'(t)^T\eta'(t))^{-1/2} \in \R^{2 \times 2}$. \\
Sei zusätzlich $m:=(m_1,m_2):=m(k_1,k_2):=B^{-1}(k_1,k_2)$, $a=\frac{|m|\delta}{2}$ und 
\[
g(b) := e^{|m|b} \erfc \left(\frac{b}{\delta}+ a \right) + e^{-|m| b} \erfc \left( - \frac{b}{\delta}+ a \right), \qquad b\in \R.
\]
Dann gilt für die Glättung mit der Glättungsfunktion $s_1$ für $\ell=1,2$ 
\begin{align*}
\F [K_{\delta,\alpha}^{(\ell)}(\cdot,0)](k_1,k_2) 
&= \frac{i^{\alpha_1+\alpha_2+1} m^{\tilde \alpha+ \tilde e_\ell}}{8 \pi|\det B| |m|} \left(\frac{\dd}{\dd b}\right)^{\alpha_3}g(0) \\
&= \frac{i^{\alpha_1+\alpha_2+1} m^{\tilde \alpha+ \tilde e_\ell}}{8 \pi|\det B| |m|} \cdot
\begin{cases}
2 \erfc(a) & \text{ für }\alpha_3=0\\
0 &\text{ für }\alpha_3=1\\
2 |m|^2 \erfc(a)-4|m|/(\delta \sqrt{\pi})e^{-a^2} &\text{ für }\alpha_3=2\\
0 &\text{ für }\alpha_3=3
\end{cases}
\end{align*} und für $l=3$
\begin{align*}
\F [K_{\delta,\alpha}^{(\ell)}(\cdot,0)](k_1,k_2) 
&= \frac{i^{\alpha_1 + \alpha_2}  m^{\tilde \alpha}}{8 \pi |\det M| |m|}\left(\frac{\dd}{\dd b}\right)^{\alpha_3 +1}g(0) \\
&= \frac{i^{\alpha_1+\alpha_2} m^{\tilde \alpha}}{8 \pi|\det B| |m|} \cdot
\begin{cases}
0 &\text{ für }\alpha_3=0\\
2 |m|^2 \erfc(a)-4|m| \frac{e^{-a^2}}{\delta \sqrt{\pi}} &\text{ für }\alpha_3=1\\
0 &\text{ für }\alpha_3=2 \\
2|m|^4 \erfc(a)+ \left(\frac{4|m|}{\delta^2}-|m|^2 \right)\frac{2 e^{-a^2}}{\delta \sqrt{\pi}}  &\text{ für }\alpha_3=3,
\end{cases}
\end{align*}
wobei $\tilde e_\ell \in \R^2$ der $\ell$-te Einheitsvektor und $\tilde \alpha = (\alpha_1,\alpha_2)$ ist.
\end{Lemma}


\begin{proof}
Die Existenz von $B=(\eta'(t)^T\eta'(t))^{1/2}$, d.h. einer Matrix $B \in \R^{2 \times 2}$ mit $(g_{ij}(t))_{i,j}:=B^2= (\eta'(t)^T\eta'(t))$ sieht man leicht ein. Einerseits ist $(\eta'(t)^T\eta'(t))$ symmetrisch, andererseits auch positiv definit, denn für $\R^2 \ni \xi \neq 0$ ist
\begin{align*}
0 &< |\eta'(t) \xi|^2  \\
&= (\eta'(t)\xi)^T  \eta'(t)\xi \\
&= \xi^T \eta'(t)^T \eta'(t) \xi,
\end{align*}
da wir $\eta$ als regulär vorausgesetzt hatten. Damit lässt sich $(\eta'(t)^T\eta'(t))$ zu $(\eta'(t)^T\eta'(t))= L D L^T$  diagonalisieren (Cholesky-Zerlegung) mit Diagonalmatrix $D$ mit positiven Einträgen und $L L^T = I$.
Die Matrix $B$ ist damit durch $B= L D^{1/2} L^T$ gegeben, wobei $D^x$ durch komponentenweises Potenzieren mit $x\in \R$ entsteht. An dieser Darstellung sieht man auch, dass $B$ symmetrisch und bijektiv mit $B^{-1}=L D^{-1/2} L^T$ ist. 

Sie von nun an $\xi \in \R^2, b\in \R$. Da $\widetilde G_\delta$ und damit $K_{\delta,\alpha}^{(\ell)}$ radial ist und $|\eta'(t) \xi - b n(t)|^2= |\eta'(t)\xi|^2 + |b n(t)|^2$, vereinfachen wir zunächst die Abhängigkeit von $\xi$. 
%Dazu setzen wir $B=(\eta'(t)^T\eta'(t))^{1/2}$ ( d.h. $B^2=g:=(g_{ij}(t))_{i,j=1,2}$ mit $g_{ij}(t)=T_i(t) \cdot T_j(t)$).
%Die Existenz von $B$ sieht man dabei leicht ein, denn einerseits ist $(\eta'(t)^T\eta'(t))$ symmetrisch, andererseits sogar positiv definit, denn für $\xi \neq 0$ ist
%\begin{align*}
%0 &< |\eta'(t) \xi|^2  \\
%&= (\eta'(t)\xi)^T  \eta'(t)\xi \\
%&= \xi^T \eta'(t)^T \eta'(t) \xi,
%\end{align*}
%da wir $\eta$ als regulär vorausgesetzt hatten. Damit lässt sich $(\eta'(t)^T\eta'(t))$ zu $(\eta'(t)^T\eta'(t))= L D L^T$  diagonalisieren (Cholesky-Zerlegung) mit Diagonalmatrix $D$ mit positiven Einträgen und $L L^T = I$.
%Die Matrix $B$ ist damit durch $B= L D^{1/2} L^T$ gegeben, wobei $D^{1/2}$ durch komponentenweises Wurzelziehen entsteht, und bijektiv mit $B^{-1}=L D^{-1/2} L^T$.
Wir folgern, da $B$ symmetrisch ist
\[
|\eta'(t)\xi|^2=\xi^T \eta'(t)^T \eta'(t) \xi = \xi^T \left( (\eta'(t)^T \eta'(t))^\frac{1}{2}) \right)^T (\eta'(t)^T \eta'(t))^\frac{1}{2} \xi= |B \xi|^2
\]
und 
\begin{align*}
|B \xi|^2 
&= (\eta'(t) \xi) \cdot (\eta'(t) \xi) \\
&=T_1(t) \cdot T_1(t) \xi_1^2 +2 T_1(t) \cdot T_2(t) \xi_1 \xi_2 + T_2(t) \cdot T_2(t) \xi_2^2 \\
&= \sum_{i,j=1}^2 g_{ij}(t) \xi_i \xi_j.
\end{align*}
Damit ist $|\eta'(t)\xi-b n(t)|=|(B\xi,b)|$ und
\begin{align*}
K_{\delta,\alpha}^{(\ell)}(\xi,b) 
&= \left(D^{\alpha + e_\ell} \widetilde{G}_\delta\right)(\eta'(t) \xi - b n(t)) \\
&=\left([D^{\alpha + e_\ell} \widetilde{G}_\delta]  \circ M \right)(\xi,b)
\end{align*} 
mit der linearen Abbildung
\[
M:\R^2 \times \R \to \R^3, (\xi, b) \mapsto (B\xi , b).
\]
mit $|\det M| = 1 \cdot |\det B|$ (Laplace'scher Entwicklungsatz).

Die 3D-Fouriertransformation von $K_{\delta,\alpha}^{(\ell)}$ ist damit für $k \in \R^3$ gegeben durch
\begin{align*}
\F [K_{\delta,\alpha}^{(\ell)}](k) 
&= \F \left([D^{\alpha + e_\ell} \widetilde{G}_\delta]  \circ M \right)(k) \\
&= \frac{1}{(2 \pi)^{\frac{3}{2}}}\int_{\R^3} D^{\alpha + e_\ell} \widetilde G_\delta (M(\xi,b)) e^{-i(\xi,b)\cdot k} \dd(\xi,b) \\
&= \frac{1}{(2 \pi)^{\frac{3}{2}}}\int_{\R^3} D^{\alpha + e_\ell} \widetilde G_\delta (\tilde \xi, \tilde b) e^{-i M^{-1}(\tilde \xi,\tilde b)\cdot k} |\det M^{-1}| \dd(\tilde \xi,\tilde b) \\
&= \frac{1}{|\det M|} \F \left(D^{\alpha + e_\ell} \widetilde{G}_\delta\right)((M^T)^{-1} k) \\
&=\frac{1}{|\det M|} i^{|\alpha|+1} ((M^T)^{-1} k)^{\alpha + e_\ell} \F \left(\widetilde{G}_\delta\right)((M^T)^{-1} k). \\
&=\frac{1}{|\det B|} i^{|\alpha|+1} (B^{-1}(k_1,k_2),k_3)^{\alpha + e_\ell} \F \left(\widetilde{G}_\delta\right)(B^{-1}(k_1,k_2),k_3), \\
\end{align*}

Zur einfacheren Notation schreiben wir für $k=(k_1,k_2,k_3)\in \R^3$ 
\begin{align*}
m=(m_1,m_2)&:=B^{-1}(k_1,k_2), \\
\widetilde m &:= (m_1,m_2,k_3),\\
\end{align*}
d.h.
\[
\F [K_{\delta,\alpha}^{(\ell)}](k)=\frac{1}{|\det B|} i^{|\alpha|+1} \widetilde m^{\alpha + e_\ell} \F \left(\widetilde{G}_\delta\right)(\widetilde m) \fa k\in \R^3
\] 
Mit der inversen Fouriertransformation (siehe Satz \ref{Feig} \ref{Feig4}) bezüglich der 3.Komponente erhalten wir nun die Fouriertransformation nur bezüglich der Koordinaten $(k_1,k_2)$. Für $(k_1,k_2) \neq 0$ und $\ell =3$ ist dabei die Integrierbarkeit gewährleistet und wir erhalten
\begin{align*}
\F [K_{\delta,\alpha}^{(\ell)}(\cdot,b)](k_1,k_2) 
& =(2\pi)^{-1/2} \int_\R \F K_{\delta,\alpha}^{(\ell)}(k_1,k_2,k_3) e^{i k_3 b} \dd k_3 \\
&= \frac{i^{|\alpha|+1}}{\sqrt{2 \pi}|\det B|} \int_\R  (m,k_3)^{\alpha + e_\ell} \F \left(\widetilde{G}_\delta\right)(m,k_3) e^{i k_3 b} \dd k_3\\
&= \frac{i^{\alpha_1 + \alpha_2} m_1^{\alpha_1} m_2^{\alpha_2}}{\sqrt{2 \pi}|\det B|} \int_\R  (i k_3)^{\alpha_3+1} \F \left(\widetilde{G}_\delta\right)(m,k_3) e^{i k_3 b} \dd k_3 \\
&= \frac{i^{\alpha_1 + \alpha_2} m_1^{\alpha_1} m_2^{\alpha_2}}{\sqrt{2 \pi}|\det B|} \left(\frac{\partial}{\partial b}\right)^{\alpha_3 + 1}\int_\R  \F \left(\widetilde{G}_\delta\right)(m,k_3) e^{i k_3 b} \dd k_3 . 
\end{align*}
Für $\ell=1,2$, $\tilde e_\ell \in \R^2$ der $\ell$-te Einheitsvektor in $\R^2$ und $\tilde \alpha= (\alpha_1,\alpha_2)$. Dann erhalten wir analog
\[
\F [K_{\delta,\alpha}^{(\ell)}(\cdot,b)](k_1,k_2) = \frac{i^{\alpha_1+\alpha_2+1} m^{\tilde \alpha+ \tilde e_\ell}}{\sqrt{2 \pi}|\det B|} \left(\frac{\partial}{\partial b}\right)^{\alpha_3}\int_\R  \F \left(\widetilde{G}_\delta\right)(m,k_3) e^{i k_3 b} \dd k_3.
\]
Wegen der Stetigkeit von $\F [K_{\delta,\alpha}^{(\ell)}(\cdot,b)]$ gilt diese Gleicheit auch für $(k_1,k_2)=0$.



Mit Lemma \ref{LFGdelta}, der Symmetrie des Integranden in $k_3$ und Lemma \ref{LFGdelta2} folgern wir
\begin{align*}
&\quad \int_\R  \F \left(\widetilde{G}_\delta\right)(m,k_3) e^{i k_3 b} \dd  k_3 \\
&=(2 \pi)^{-\frac{3}{2}} \int_\R \frac{1}{|m|^2+k_3^2}e^{-\frac{\delta^2(|m|^2 + k_3^2)}{4}} e^{i k_3 b} \dd k_3 \\
&=(2 \pi)^{-\frac{3}{2}} \int_\R \frac{1}{|m|^2+k_3^2}e^{-\frac{\delta^2(|m|^2 + k_3^2)}{4}} e^{-i k_3 b} \dd k_3 \\
&= (2 \pi)^{-\frac{3}{2}} e^{-\frac{\delta^2 |m|^2}{4}} \frac{\pi}{2 |m|} e^{\frac{\delta^2 |m|^2}{4}}
\left[
e^{|m|b} \erfc \left(\frac{b}{2 \frac{\delta}{2}}+ \frac{ |m|\delta}{2}\right) + e^{-|m| b} \erfc \left( - \frac{b}{\delta}+ \frac{|m| \delta}{2} \right) 
\right]\\
&=\frac{1}{|m| 2^{\frac{5}{2}}\sqrt{\pi}}
\left[
e^{|m|b} \erfc \left(\frac{b}{\delta}+ \frac{ |m|\delta}{2}\right) + e^{-|m| b} \erfc \left( - \frac{b}{\delta}+ \frac{|m| \delta}{2} \right) 
\right]
\end{align*}
für alle $k \in \R^2$.
Für die Berechnung der partiellen Ableitungen nach $b$ definieren wir nun für $b \in \R$ die Funktionen 
\begin{align*}
\tilde g(b)&:= e^{|m|b} \erfc \left(\frac{b}{\delta}+ \frac{ |m|\delta}{2}\right) \\
g(b)&:=\tilde g (b) + \tilde g(-b).
\end{align*}
Für $b \in \R$ erhalten wir mit $\erfc'(b)=-\erf'(b)=-(2/\sqrt{\pi})e^{-b^2}$
\begin{align*}
g'(b)&= |m| e^{|m|b} \erfc \left(\frac{b}{\delta}+ \frac{ |m|\delta}{2}\right) 
-(2/\sqrt{\pi}) e^{|m|b} e^{- \left(\frac{b}{\delta}+ \frac{ |m|\delta}{2}\right)^2}  \frac{1}{\delta} \\
& \quad -|m| e^{-|m| b} \erfc \left( - \frac{b}{\delta}+ \frac{|m| \delta}{2} \right) 
+(2/\sqrt{\pi}) e^{-|m| b} e^{- \left( - \frac{b}{\delta}+ \frac{|m| \delta}{2} \right)^2} \frac{1}{\delta} \\
&=|m| e^{|m|b} \erfc \left(\frac{b}{\delta}+ \frac{ |m|\delta}{2}\right) -|m| e^{-|m| b} \erfc \left( - \frac{b}{\delta}+ \frac{|m| \delta}{2} \right) \\
&=|m|( \tilde g(b) - \tilde g (-b)),
\end{align*}
da 
\begin{align*}
|m|b -\left(\frac{b}{\delta}+\frac{|m|\delta}{2}\right)^2 
&= |m|b - \frac{b^2}{\delta^2}- |m| b - \frac{|m|^2\delta^2}{4}  \\
&=-|m|b -\left(-\frac{b}{\delta}+\frac{|m|\delta}{2}\right)^2.
\end{align*}
Weiter ist
\begin{align*}
g''(b) &= |m|\bigg[ |m|e^{|m|b} \erfc \left(\frac{b}{\delta}+ \frac{ |m|\delta}{2}\right) 
-(2/\sqrt{\pi}) e^{|m|b} e^{- \left(\frac{b}{\delta}+ \frac{ |m|\delta}{2}\right)^2}  \frac{1}{\delta} \\
& \quad \qquad +|m| e^{-|m| b} \erfc \left( - \frac{b}{\delta}+ \frac{|m| \delta}{2} \right) 
-(2/\sqrt{\pi}) e^{-|m| b} e^{- \left( - \frac{b}{\delta}+ \frac{|m| \delta}{2} \right)^2} \frac{1}{\delta} \bigg] \\
&=|m|^2(\tilde g(b) + \tilde g(-b)) - 2 |m| g_2(b) \\
&=|m|^2 g(b) - 2|m| g_2(b)
\end{align*}
mit $g_2(b)= 2 / (\delta\sqrt{\pi}) e^{-\frac{b^2}{\delta^2}-\frac{|m|^2\delta^2}{4}}$.
Wir berechnen $g_2'(b)=-\frac{2b}{\delta^2}g_2(b)$ und
\begin{align*}
g_2''(b) &= -\frac{2}{\delta^2}g_2(b) - \frac{2b}{\delta^2}g_2'(b)  \\
&=  -\left(\frac{2}{\delta^2} - \frac{4b^2}{\delta^4} \right)g_2(b)
\end{align*}
%\begin{align*}
%g_2'''(b) &=   \left(\frac{8b}{\delta^4} \right)g_2(b)+ \left(\frac{4b}{\delta^4} - \frac{8b^3}{\delta^6} \right)g_2(b) \\
%&=  \left(\frac{12b}{\delta^4} - \frac{8b^3}{\delta^6} \right)g_2(b)
%\end{align*}
Damit ist 
\begin{align*}
g'''(b)&=|m|(|m|g'(b)-2g_2'(b)) \\
&= |m|^3 (\tilde g(b) - \tilde g(-b)) - \frac{2b |m|}{\delta^2}g_2(b)
\end{align*}
und 
\begin{align*}
\left(\frac{\dd}{\dd b} \right)^4 g(b) &= |m|^4 g(b) - 2|m|^3 g_2(b) -2 |m| g_2''(b) \\
&= |m|^4 g(b) - 2|m|^3 g_2(b) +  2 |m| \left(\frac{2}{\delta^2} - \frac{4b^2}{\delta^4} \right)g_2(b) \\
&= |m|^4 g(b) +  2 |m| \left(\frac{2}{\delta^2} - \frac{4b^2}{\delta^4} - |m|^2 \right)g_2(b). \\
\end{align*}
Insgesamt erhalten wir
\begin{align*}
\left( \frac{\dd}{\dd b}\right)^j g(0)&=\begin{cases}
2 \tilde g(0) &\text{ für } j=0\\
0 &\text{ für } j=1 \\
2|m|^2 \tilde g (0) - 2 |m| g_2(0) &\text{ für } j=2\\
0 &\text{ für } j=3\\
2|m|^4 \tilde g(0) + \left( \frac{4|m|}{\delta^2}-|m|^2 \right) g_2(0) &\text{ für } j=4
\end{cases}
\end{align*}
%\begin{align*}
%g(0)&=2 \tilde g(0),\\
%g'(0)&= 0, \\
%g''(0)&= 2|m|^2 \tilde g (0) - 2 |m| g_2(0),\\
%g'''(0)&= 0 \text{ \qquad und } \\
%g''''(0)&= 2|m|^4 \tilde g(0) + \left( \frac{4|m|}{\delta^2}-|m|^2 \right) g_2(0).
%\end{align*}
Setzen wir unsere Ergebnisse zusammen, so erhalten wir für den Fall $\ell=1,2$
\begin{align*}
\F [K_{\delta,\alpha}^{(\ell)}(\cdot,0)](k_1,k_2) &= 
\frac{1}{|m| 2^{\frac{5}{2}}\sqrt{\pi}}\frac{i^{\alpha_1+\alpha_2+1} m^{\tilde \alpha+ \tilde e_\ell}}{\sqrt{2 \pi}|\det B|} \left(\frac{\dd}{\dd b}\right)^{\alpha_3}g(0) \\
&=\frac{i^{\alpha_1+\alpha_2+1} m^{\tilde \alpha+ \tilde e_\ell}}{8 \pi|\det B| |m|} \left(\frac{\dd}{\dd b}\right)^{\alpha_3}g(0),
\end{align*}
d.h. mit $a=\frac{|m|\delta}{2}$
\begin{align*}
\F [K_{\delta,\alpha}^{(\ell)}(\cdot,0)](k_1,k_2) = \frac{i^{\alpha_1+\alpha_2+1} m^{\tilde \alpha+ \tilde e_\ell}}{8 \pi|\det M| |m|} \cdot
\begin{cases}
2 \erfc(a) & \text{ für }\alpha_3=0\\
0 &\text{ für }\alpha_3=1\\
2 |m|^2 \erfc(a)-4|m|\frac{e^{-a^2}}{\delta \sqrt{\pi}} &\text{ für }\alpha_3=2\\
0 &\text{ für }\alpha_3=3.
\end{cases}
\end{align*}
Analog ist für $\ell=3$ 
\begin{align*}
\F [K_{\delta,\alpha}^{(\ell)}(\cdot,0)](k_1,k_2) &= \frac{i^{\alpha_1 + \alpha_2}  m^{\tilde \alpha}}{8 \pi |\det B| |m|}\left(\frac{\dd}{\dd b}\right)^{\alpha_3 +1}g(0),
\end{align*} d.h.
\begin{align*}
\F [K_{\delta,\alpha}^{(\ell)}(\cdot,0)](k_1,k_2) = \frac{i^{\alpha_1+\alpha_2} m^{\tilde \alpha}}{8 \pi|\det B| |m|} \cdot
\begin{cases}
0 &\text{ für }\alpha_3=0\\
2 |m|^2 \erfc(a)-4|m|\frac{e^{-a^2}}{\delta \sqrt{\pi}} &\text{ für }\alpha_3=1\\
0 &\text{ für }\alpha_3=2 \\
2|m|^4 \erfc(a)+ \left(\frac{4|m|}{\delta^2}-|m|^2 \right)\frac{2 e^{-a^2}}{\delta \sqrt{\pi}}  &\text{ für }\alpha_3=3.
\end{cases}
\end{align*}
\end{proof}
Lemma \ref{LFKdelta} zeigt, dass sich $\F [K_{\delta,\alpha}^{(\ell)}(\cdot,0)](k_1,k_2)$ im Fall der Glättung mit $s_1$ für jedes $\ell=1,2,3$ als Produkt von Polynomen in $m$,$|m|$ und $|m|^{-1}$ und $\erfc(a)$ bzw. $e^{-a^2}$ schreiben lässt. Wir wollen letztlich 
\[
\F F_\rho(k) = \F [K_{\delta,\alpha}^{(\ell)}(\cdot,0)\cdot^\nu ](k) \fa k \in \R^2
\]
berechnen und die Fouriertransformation macht die Multiplikation mit Polynomen zu Ableitungen der Fouriertransformation. Wie wir in Lemma \ref{LFFrho} nachrechnen werden, lässt sich damit auch $\F F_\rho(k)$ als Summe von Produkten von Polynomen in $m$,$|m|$ und $|m|^{-1}$  mit den Funktionen  $k \mapsto \erfc(a)$ bzw. $k \mapsto e^{-a^2}$ schreiben. Bevor wir dies zeigen widmen wir uns noch dem bei der zweiten Glättungsfunktion $s_2$ zusätzlich auftretenden Term (vergleiche Gleichung \eqref{EqGdeltas2} aus der Vorüberlegung).

%Für die Glättung mit der Glättungsfunktion $s_2$ brauchen wir für ein analoges Resultat zu Lemma \ref{LFKdelta} wegen der Linearität der Fouriertransformation und der Ableitung noch die Fouriertransformation von  
%\[
%E_{\delta,\alpha}^{(\ell)}(\xi,b):=D^{\alpha+e_\ell}\left[\R^3 \ni z \mapsto \frac{2}{3 \sqrt{\pi} \delta } e^{-\frac{|z|^2}{\delta^2}}\right](\eta'(t) \xi - b n(t)),  \quad \xi \in \R^2, b \in \R.
%\]
Wegen der Linearität der Fouriertransformation und der Ableitung brauchen wir für ein entsprechendes Resultat zu Lemma \ref{LFKdelta} für die Glättung mit der Glättungsfunktion $s_2$ noch die Fouriertransformation von  
\[
E_{\delta,\alpha}^{(\ell)}(\xi,b):=D^{\alpha+e_\ell}\left[\R^3 \ni z \mapsto \frac{2}{3 \sqrt{\pi} \delta } e^{-\frac{|z|^2}{\delta^2}}\right](\eta'(t) \xi - b n(t)),  \quad \xi \in \R^2, b \in \R.
\]
Die Funktion $F_\rho$ für die Glättung mit $s_2$ können wir nämlich in zwei Teile zerlegen, denn in diesem Fall ist nach der Vorüberlegung
\begin{align*}
K_{\delta,\alpha}^{(\ell)}(\xi,b) &= D^{\alpha+e_\ell}\left[z \mapsto \widetilde G_\delta(z) + \frac{2}{3 \sqrt{\pi} \delta } e^{-\frac{|z|^2}{\delta^2}}\right](\eta'(t) \xi - b n(t)) \\
&=D^{\alpha+e_\ell}[\widetilde G_\delta](\eta'(t) \xi - b n(t)) + E_{\delta,\alpha}^{(\ell)}(\xi,b) \fa \xi \in \R^2, b\in \R.
\end{align*}
Der erste Summand ist gerade $K_{\delta,\alpha}^{(\ell)}(\xi,b)$ für $s_1$, womit für $s_2$ nur $E_{\delta,\alpha}^{(\ell)}(\xi,b)$ zusätzlich betrachtet werden muss.
Dafür wird uns die 1D-Fouriertransformation von $x \mapsto e^{-b^2 x^2}$ behilflich sein.
\begin{Lemma} \label{LFexp}
Sei $b \in \R_{\geq 0}$. Dann gilt
\[
\F \left[x \mapsto e^{-b^2 x^2} \right](k)= \frac{1}{\sqrt{2} b} e^{-\frac{k^2}{4 b^2}}, \qquad k \in \R
\]
und 
\[
\F^{-1}\left[k \mapsto e^{-\frac{k^2}{4 b^2}}\right](x) = \sqrt{2} b e^{-b^2 x^2}, \qquad x \in \R.
\]
\end{Lemma}
\begin{proof}
Siehe Lemma III.11 in  \cite{Collet}. Die Aussage für $\F^{-1}$ folgt direkt aus der Linearität von $\F$.
\end{proof}

\begin{Lemma} \label{LFEdelta}
Seien $\delta>0, k \in \R^2$, $\alpha=(\alpha_1,\alpha_2,\alpha_3) \in \N_0^3$ mit $|\alpha| \leq 3$ und $B=(\eta'(t)^T \eta'(t))^{1/2}$. Seien weiterhin $(m_1,m_2)=m=m(k_1,k_2)=B^{-1}(k_1,k_2)$, $a=\frac{|m|\delta}{2}$ und
\[
g(b):=\frac{\delta^2}{4 \pi}e^{-a^2} e^{-\frac{b^2}{\delta^2}}, \qquad b \in \R.
\] 
Dann gilt für $\ell=1,2$
\begin{align*}
\F [E_{\delta,\alpha}^{(\ell)}(\cdot,0) ](k_1,k_2) 
&= \frac{2}{3\sqrt{\pi}\delta} \frac{
i^{\alpha_1 +\alpha_2+1} m^{\widetilde \alpha+ \widetilde e_\ell}}{
\sqrt{2\pi} |\det B|}
\left(\frac{\partial}{\partial b}\right)^{\alpha_3} g(0) \\
&=\frac{\delta}{6 \pi^{2}} \frac{
i^{\alpha_1 +\alpha_2+1} m^{\widetilde \alpha+ \widetilde e_\ell}}{
\sqrt{2} |\det B|}
\cdot
\begin{cases}
e^{-a^2} &\text{ für } \alpha_3=0 \\
0  &\text{ für } \alpha_3=1 \\
\frac{-2}{\delta^2}e^{-a^2} &\text{ für } \alpha_3=2 \\
0  &\text{ für } \alpha_3=3 \\
\end{cases}
\end{align*}
bzw. für $\ell=3$
\begin{align*}
\F [E_{\delta,\alpha}^{(\ell)}(\cdot,0) ](k_1,k_2) &= \frac{2}{3\sqrt{\pi}\delta} \frac{i^{\alpha_1 +\alpha_2} m_1^{\alpha_1}m_2^{\alpha_2}}{\sqrt{2\pi} |\det B|}\left(\frac{\partial}{\partial b}\right)^{\alpha_3+1} g(0) \\
&=\frac{\delta}{6 \pi^{2}} \frac{i^{\alpha_1 +\alpha_2} m_1^{\alpha_1}m_2^{\alpha_2}}{\sqrt{2} |\det B|} \cdot
\begin{cases}
0 &\text{ für } \alpha_3=0 \\
\frac{-2}{\delta}e^{-a^2}  &\text{ für } \alpha_3=1 \\
0 &\text{ für } \alpha_3=2 \\
\frac{12}{\delta^4}e^{-a^2}  &\text{ für } \alpha_3=3, \\
\end{cases}
\end{align*}
wobei $\tilde e_\ell \in \R^2$ der $\ell$-te Einheitsvektor und $\tilde \alpha = (\alpha_1,\alpha_2)$ ist.
\end{Lemma}
\begin{proof}
Wir können genauso wie in Lemma \ref{LFKdelta} vorgehen. Dazu definieren wir zunächst $w(z):= e^{-\frac{|z|^2}{\delta^2}}, z \in \R^3$
Wegen Lemma \ref{LFexp} gilt für $k \in \R^3$
\begin{align*}
\F w (k) 
&=(2\pi)^{-3/2} \int_{\R^3} e^{-\frac{|z|^2}{\delta^2}} e^{-i z \cdot k} \dd z \\
&=(2\pi)^{-3/2}
\int_{\R} e^{-\frac{z_1^2}{\delta^2}} e^{-i z_1 \cdot k_1} \dd z_1 
\int_{\R} e^{-\frac{z_2^2}{\delta^2}} e^{-i z_2 \cdot k_2} \dd z_2 
\int_{\R} e^{-\frac{z_3^2}{\delta^2}} e^{-i z_3 \cdot k_3} \dd z_3 \\
&= (2 \pi)^{-3/2} \left( \frac{\delta}{\sqrt{2}}\right)^3 
e^{-\frac{\delta^2 k_1^2}{4}}
e^{-\frac{\delta^2 k_2^2}{4}}
e^{-\frac{\delta^2 k_3^2}{4}} \\
&= \frac{\delta^3}{8 \pi^{3/2}} e^{-\frac{\delta^2 |k|^2}{4}}
\end{align*}
Wir definieren wie im Beweis von Lemma \ref{LFKdelta}  $B=(\eta'(t)^T \eta'(t))^{1/2}$ und erhalten da auch $w$ radial ist analog wie dort
\[
E_{\delta,\alpha}^{(\ell)}(\xi,b) = \frac{2}{3\sqrt{\pi}\delta} \left( [ D^{\alpha+ e_\ell} w] \circ M \right)(\xi,b) \fa \xi \in \R^2, b \in \R
\]
mit der linearen Abbildung
\[
M:\R^2 \times \R \to \R^3, (\xi, b) \mapsto (B\xi , b),
\]
wobei $|\det M| = 1 \cdot |\det B|$ (Laplace'scher Entwicklungsatz).
Wir fahren genauso fort wie in Lemma \ref{LFKdelta} und erhalten mit der selben Rechnung für $k \in \R^3$
\begin{align*}
\F \left( [ D^{\alpha+ e_\ell} w] \circ M \right)(k) 
&= \frac{1}{|\det B|}i^{|\alpha|+1}(B^{-1}(k_1,k_2),k_3)^{\alpha+e_\ell}\F(w)(B^{-1}(k_1,k_2),k_3) \\
&=\frac{1}{|\det B|}i^{|\alpha|+1}\widetilde m^{\alpha +e_\ell} \F (w) (\widetilde m) 
\end{align*}
mit der Notation $m=(m_1,m_2):= B^{-1}(k_1,k_2)$ und $\widetilde m= (m_1,m_2, k_3)$. Genauso liefert uns die inverse Fouriertransformation bezüglich der 3.Komponente entsprechend für $\ell=3$
\begin{align*}
\F [E_{\delta,\alpha}^{(\ell)}(\cdot,b) ](k_1,k_2) &= \frac{2}{3\sqrt{\pi}\delta} \frac{i^{\alpha_1 +\alpha_2} m_1^{\alpha_1}m_2^{\alpha_2}}{\sqrt{2\pi} |\det B|}\left(\frac{\partial}{\partial b}\right)^{\alpha_3+1} \int_{\R} \F(w)(m,k_3) e^{i k_3 b} \dd k_3,
\end{align*}
bzw. 
\begin{align*}
\F [E_{\delta,\alpha}^{(\ell)}(\cdot,b) ](k_1,k_2) 
&= \frac{2}{3\sqrt{\pi}\delta} \frac{
i^{\alpha_1 +\alpha_2+1} m^{\widetilde \alpha+ \widetilde e_\ell}}{
\sqrt{2\pi} |\det B|}
\left(\frac{\partial}{\partial b}\right)^{\alpha_3} \int_{\R} \F(w)(m,k_3) e^{i k_3 b} \dd k_3,
\end{align*}
für $\ell=1,2$ mit dem $\ell$-ten Einheitsvektor $\widetilde e_\ell \in \R^2$ und $\widetilde \alpha= (\alpha_1,\alpha_2)$.
Die Berechnung des Integrals und der Ableitungen nach $b$ erweist sich diesmal als leichter, denn mit Lemma \ref{LFexp} und $a=\frac{|m|\delta}{2}$ ist
\begin{align*}
\int_{\R} \F(w)(m,k_3) e^{i k_3 b} \dd k_3 
&= \frac{\delta^3}{8 \pi^{3/2}} \int_{\R} e^{-\frac{\delta^2 |\widetilde m|^2}{4}} e^{i k_3 b} \dd k_3 \\
&= \frac{\delta^3}{8 \pi^{3/2}}e^{-a^2} \int_{\R} e^{-\frac{\delta^2 k_3^2}{4}} e^{i k_3 b} \dd k_3 \\
&= \frac{\delta^3}{8 \pi^{3/2}}e^{-a^2} \frac{\sqrt{2} \sqrt{2 \pi}}{\delta} e^{-\frac{b^2}{\delta^2}} \\
&= \frac{\delta^2}{4 \pi}e^{-a^2} e^{-\frac{b^2}{\delta^2}} \\
&=: g(b) 
\end{align*}
für alle $b \in \R$.
Weiter berechnen wir für $b \in \R$
\begin{align*}
\left(\frac{\partial}{\partial b}\right)^j g(b) 
&=\frac{\delta^2}{4 \pi}e^{-a^2}  \begin{cases} 
e^{-\frac{b^2}{\delta^2}} 
&\text{ für } j=0,\\ 
\frac{-2b}{\delta^2} 
e^{-\frac{b^2}{\delta^2}} &\text{ für } j=1,\\ 
\left(\frac{-2}{\delta^2}+ \frac{4b^2}{\delta^4} \right)
e^{-\frac{b^2}{\delta^2}} &\text{ für } j=2,\\ 
\left(\frac{4b}{\delta^4}+ \frac{8b}{\delta^4}+\frac{-8b^3}{\delta^6} \right)
e^{-\frac{b^2}{\delta^2}} &\text{ für } j=3,\\
\left(\frac{4}{\delta^4}+\frac{-8b}{\delta^6}+\frac{8}{\delta^4}+\frac{-16 b^2}{\delta^6}+\frac{-24b^2}{\delta^6}+\frac{16b^4}{\delta^8} \right)
e^{-\frac{b^2}{\delta^2}} &\text{ für } j=4.\\
\end{cases}
\end{align*}
Fassen wir die Ergebnisse zusammen, so erhalten wir für  $\ell=3$
\begin{align*}
\F [E_{\delta,\alpha}^{(\ell)}(\cdot,0) ](k_1,k_2) &= \frac{2}{3\sqrt{\pi}\delta} \frac{i^{\alpha_1 +\alpha_2} m_1^{\alpha_1}m_2^{\alpha_2}}{\sqrt{2\pi} |\det B|}\left(\frac{\partial}{\partial b}\right)^{\alpha_3+1} g(0) \\
&=\frac{\delta}{6 \pi^{2}} \frac{i^{\alpha_1 +\alpha_2} m_1^{\alpha_1}m_2^{\alpha_2}}{\sqrt{2} |\det B|} \cdot
\begin{cases}
0 &\text{ für } \alpha_3=0 \\
\frac{-2}{\delta^2}e^{-a^2}  &\text{ für } \alpha_3=1 \\
0 &\text{ für } \alpha_3=2 \\
\frac{12}{\delta^4}e^{-a^2}  &\text{ für } \alpha_3=3, \\
\end{cases}
\end{align*}
bzw. für $\ell=1,2$
\begin{align*}
\F [E_{\delta,\alpha}^{(\ell)}(\cdot,0) ](k_1,k_2) 
&= \frac{2}{3\sqrt{\pi}\delta} \frac{
i^{\alpha_1 +\alpha_2+1} m^{\widetilde \alpha+ \widetilde e_\ell}}{
\sqrt{2\pi} |\det B|}
\left(\frac{\partial}{\partial b}\right)^{\alpha_3} g(0) \\
&=\frac{\delta}{6 \pi^{2}} \frac{
i^{\alpha_1 +\alpha_2+1} m^{\widetilde \alpha+ \widetilde e_\ell}}{
\sqrt{2} |\det B|}
\cdot
\begin{cases}
e^{-a^2} &\text{ für } \alpha_3=0 \\
0  &\text{ für } \alpha_3=1 \\
\frac{-2}{\delta^2}e^{-a^2} &\text{ für } \alpha_3=2 \\
0  &\text{ für } \alpha_3=3, \\
\end{cases}
\end{align*}
für alle $k=(k_1,k_2)\in \R^2$.

\end{proof}

Damit können wir nun die Fouriertransformation von $F_\rho$ bezüglich $s_1$ und $s_2$ berechnen. Davor erinnern wir an die Definition von Laurent-Polynomen.
\begin{Def}
Ein Funktion $g:\R\backslash \{0\} \to \R$ heißt \emph{Laurent-Polynom}, wenn es ein $s_{\max}\in \N$ gibt, so dass
\[
g(z)= \sum_{j=-s_{\max}}^{s_{\max}} C_j z^{j} \fa z \in \R \backslash \{0\}
\]
für Konstanten $C_j\in \R$.
\end{Def}
\begin{Lemma} \label{LFFrho}
Seien $\alpha,m,a$ und $B$ wie in Lemma \ref{LFKdelta}, $\nu \in \N_0^2$ und 
$F_\rho(\xi) = K_{\delta,\alpha}^{(\ell)}(\xi,0)\xi^\nu$ mit 
\[
K_{\delta,\alpha}^{(\ell)}(\xi,b)=[D^{\alpha+e_\ell}\widetilde{G}_\delta](\eta'(t) \xi - b n(t))
\]
oder
\[
K_{\delta,\alpha}^{(\ell)}(\xi,b)=D^{\alpha+e_\ell}\left[z \mapsto \widetilde G_\delta(z) + \frac{2}{3 \sqrt{\pi} \delta } e^{-\frac{|z|^2}{\delta^2}}\right](\eta'(t) \xi - b n(t))
\]
für $\xi \in \R^2, b \in \R$. \\
Dann gibt es in beiden Fällen ein $\tilde \ell\in \N_0$ und von $B$ abhängige stetige  Funktionen \[
r_o,\tilde r_o: \R^2 \times \R_{>0} \to \R , \quad -2|\nu| \leq o \leq \tilde \ell,
\] so dass für $\ell=1,2,3$ und $k \in \R^2 \backslash \{0 \}$
\begin{align*}
\F F_\rho(k) &= \F K_{\delta,\alpha}^{(\ell)}(\cdot,0)\cdot^\nu (k) \\
&= i^{d} \left(\sum_{o=-2|\nu|}^{\tilde \ell} \frac{r_o(k,\delta)}{|m|^o}e^{-a^2} + \sum_{o=-2|\nu|}^{\tilde \ell} \frac{\tilde r_o(k,\delta)}{|m|^o}\erfc(a) \right)
\end{align*}
mit \[
d= 
\begin{cases} 
|\nu|+\alpha_1+\alpha_2+1 &\text{ für } l=1,2\\
|\nu|+\alpha_1+\alpha_2 &\text{ für } l=3
\end{cases}
\]
gilt.
Dabei besitzen die Funktionen $r_o$ und $\tilde r_o$, $-2|\nu|\leq o \leq \tilde \ell$, die Gestalt
\[
r_o(k,\delta) = \sum_{|\beta| \leq \beta_{\max}} C_{\beta} p_{\beta}(\delta)  k^{\beta}, \fa \delta>0, k \in \R^2
\]
mit von $B$ abhängigen Laurent-Polynomen $p_{\beta}$ und von $B$ abhängigen Konstanten $C_{\beta}\in \R$ für alle $\beta \in \N_0^2$ mit $|\beta|\leq \beta_{\max}\in \N$. 
 


\end{Lemma}


\begin{proof}
Sei $k \in \R^2$.
Mit den Rechenregeln für die Fouriertransformation bezüglich Ableitungen erhalten wir
\begin{align*}
\F F_\rho(k) 
&= \F \left[K_{\delta,\alpha}^{(\ell)}(\cdot,0)(\cdot)^\nu \right](k) \\
&=i^{|\nu|} D^\nu \left[ \F[ K_{\delta,\alpha}^{(l)}(\cdot,0)] \right](k).
\end{align*}
Wir erinnern an die allgemeine Produktegel für $\nu \in \N_0^2$ und $f,\psi \in C^{|\nu|}(\R^2)$ 
\[
D^\nu (f \cdot \psi) = \sum_{0 \leq \mu \leq \nu} \begin{pmatrix}
\nu \\ \mu 
\end{pmatrix} D^\mu f D^{\nu-\mu} \psi,
\] die wir auf die Darstellung von $\F[ K_{\delta,\alpha}^{(l)}(\cdot,0)]$ aus Lemma \ref{LFKdelta} anwenden wollen.
Dazu setzen wir $m=m(k_1,k_2)=B^{-1}(k_1,k_2)^T$ und $a=\frac{|m|\delta}{2}$ mit $B$ und $g$ wie in Lemma \ref{LFKdelta} und 
\begin{align*}
f(k_1,k_2) &:= i^{|d|}m^d \\
\psi(k_1,k_2)&:=\frac{1}{|m|}\left(\frac{\dd}{\dd b} \right)^j g(0)
\end{align*}
für $(k_1,k_2)\in \R^2$ und festes $d\in \N_0^2, j \in \N_0$. 
Damit ist für $e_1=(1,0),e_2=(0,1)$
\begin{align*}
D^{e_1} f(k_1,k_2) &= 
\frac{\partial}{\partial k_1}f(k_1,k_2) \\ 
&=
i^{|d|}d_1(B^{-1} k)_1^{d_1-1} (B^{-1} k)_2^{d_2} B^{-1}_{1,1} \\
& \quad + i^{|d|}d_2(B^{-1} k)_1^{d_1} (B^{-1} k)_2^{d_2-1} B^{-1}_{2,1} \\
&=i^{|d|}d!\left( m^{d-e_1} \frac{B_{1,1}^{-1}}{(d-e_1)!} + m^{d-e_2} \frac{B_{2,1}^{-1}}{(d-e_2)!} \right)
\end{align*}
und analog 
\[
D^{e_2} f(k_1,k_2) = i^{|d|}d!\left( m^{d-e_1} \frac{B_{1,2}^{-1}}{(d-e_1)!} + m^{d-e_2} \frac{B_{2,2}^{-1}}{(d-e_2)!} \right),
\]
falls $(1,1) \leq d$ für alle $k=(k_1,k_2) \in \R^2$. Definieren wir weiter $m^d=0$ und $d!=1$ für $m\in \R^2$ und $d \in \Z^2 \backslash \N_0^2$, so gelten die obigen Gleichungen für alle $d \in \N_0^2$ und wir erhalten für $\mu \in \N_0^2$ induktiv
\begin{align*}
 D^\mu f (k_1,k_2) &=i^{|d|} \sum_{\beta\leq d} C_{\beta,B} m^\beta \\
 &= \tilde f(k)
\end{align*}
mit von $k_1,k_2$ unabhängigen Konstanten $C_{\beta,B} \in \R$, d.h. einem Polynom $\tilde f$, dessen Koeffizienten von $B$ abhängen. 

Wir betrachten zunächst den Fall zur ersten Glättungsfunktion, d.h.
\[
K_{\delta,\alpha}^{(\ell)}(\xi,b)=[D^{\alpha+e_\ell}\widetilde{G}_\delta](\eta'(t) \xi - b n(t)) \fa \xi \in \R^2, b\in \R.
\]
Die Funktion $\psi$ hat nach Lemma \ref{LFKdelta} für $0<j\leq 4$ die Gestalt
\begin{align} \label{EqPsi}
\psi(k)= q_1(|m|)\erfc(a)+q_2(|m|)e^{-a^2} \fa k \in \R^2
\end{align}
mit von $\delta$ abhängigen Polynomen $q_1,q_2$. 
Für die Ableitungen von $\psi$ rechnen wir zunächst nach, dass für $k=(k_1,k_2)\in \R^2 \backslash \{0\}$ und $i=1,2$ wegen der Symmetrie von $B^{-1}$ und mit $B^{-2}=(g_{ij})^{-1}$
\begin{align*}
D^{e_i} |m| 
&= \frac{\partial}{\partial k_i} B^{-1} k \cdot B^{-1} k \\
&= \frac{\partial}{\partial k_i} B^{-2} k \cdot  k \\
&= \frac{1}{2}\frac{1}{|m|} \sum_{o,p=1}^2 g_{o,p}^{-1}\left( \frac{\partial k_o}{\partial k_i} k_p + k_o \frac{\partial k_p}{\partial k_i} \right) \\
&= \frac{1}{2}\frac{1}{|m|} \left(\sum_{p=1}^2 g_{i,p}^{-1}k_p + \sum_{o=1}^2 g_{o,i}^{-1}k_o \right)  \\
&=\frac{1}{|m|}\sum_{o=1}^2 g_{i,o}^{-1} k_o \\
&= \frac{g_i^{-1}\cdot k}{|m|}
\end{align*}
 gilt, wobei $g_i^{-1}$ die $i$-te Spalte von $(g_{ij})^{-1}$ ist. Daraus folgt für $d \in \Z\backslash \{0\}$
\begin{align*}
D^{e_i} |m|^d
&= d |m|^{d-1} D^{e_i} |m| \\ 
&= d |m|^{d-2} g_i^{-1} \cdot k
\end{align*} und induktiv
\begin{align*}
D^\mu \left[k \mapsto |m|^d \right] = \sum_{o=0}^{|\mu|}|m|^{d-2o}p_{d,\mu,o}(k)
\end{align*}
mit Polynomen $p_{d,\mu,o}$ mit Grad kleiner oder gleich $2o$. Daraus folgt insbesondere für Polynome $q$ vom Grad kleiner oder gleich $d \in \N_0$, dass die Ableitung die Form 
\[
D^\mu [k \mapsto q(|m|)] = \sum_{o=-2|\mu|}^{\ell} |m|^o q_{\mu,o,q}(k) 
\]
mit Polynomen $q_{\mu,o,q}$ hat. Analog erhalten wir für $|\mu|\neq 0$
\begin{align}\label{EqDerfc}
D^\mu [k \mapsto \erfc(a)](k)= \sum_{o=0}^{|\mu|-1} \frac{r_{\mu,o}(k)}{|m|^o} e^{-a^2}
\end{align} mit von $\delta$ abhängigen Polynomen $r_{\mu,o}$ und
\begin{align}\label{EqDexp}
D^\mu [k \mapsto e^{-a^2}](k)= r_{\mu}(k) e^{-a^2}
\end{align}
für alle $\R^2 \ni k \neq 0$, mit einem von $\delta$ abhängigem Polynom $r_{\mu}$.
Für $0<j\leq 4$ erhalten wir also für $\mu\in \N_0^2$
\begin{align*}
D^\mu \psi(k) 
&=D^\mu [ k \mapsto q_1(|m|)\erfc(a)+q_2(|m|)e^{-a^2}](k) \\
&=\sum_{0 \leq \zeta \leq \mu} 
\begin{pmatrix}
\mu \\ \zeta 
\end{pmatrix} \left[k \mapsto
D^{\zeta} q_1(|m|) D^{\mu-\zeta} \erfc(a) + D^{\zeta} q_2(|m|) D^{\mu-\zeta} e^{-a^2} \right](k) \\
&= \sum_{0 \leq \zeta < \mu} 
\begin{pmatrix}
\mu \\ \zeta 
\end{pmatrix} \bigg[\left(\sum_{o=-2|\zeta|}^{\ell_1} |m|^o q_{\zeta,o,q_1}(k)\right) \left(\sum_{o=0}^{|\mu-\zeta|-1} \frac{r_{\mu-\zeta,o}(k)}{|m|^o} e^{-a^2}\right) \\
& \qquad \qquad \qquad  +
\sum_{o=-2|\zeta|}^{\ell_2} |m|^o q_{\zeta,o,q_2}(k) r_{\mu-\zeta}(k) e^{-a^2}
 \bigg] \\
 & \quad + \left(\sum_{o=-2|\mu|}^{\ell_1} |m|^o q_{\mu,o,q_1}(k)\right) \erfc(a) \\
& \quad  +
\sum_{o=-2|\mu|}^{\ell_2} |m|^o q_{\mu,o,q_2}(k) e^{-a^2}
\end{align*}
wobei $\ell_1$ und $\ell_2$ die Grade von $q_1$ und $q_2$ sind. Zur Vereinfachung setzen wir $\tilde \ell=\max(\ell_1,\ell_2)$.
%
%Wir betrachten zunächst das zu $s_1$ gehörige  $K_{\delta,\alpha}^{(\ell)}$, also 
%\[
%K_{\delta,\alpha}^{(\ell)}(\xi,b)=[D^{\alpha+e_\ell}\widetilde{G}_\delta](\eta'(t) \xi - b n(t)), \quad \xi \in \R^2, \quad b \in \R.
%\]

Zur Berechnung von $D^\nu \left[ \F[ K_{\delta,\alpha}^{(l)}(\cdot,0)] \right](k)$ schauen wir uns nun zunächst die Fälle mit $\alpha_3 \neq 0$ oder $\ell=3$ an. In diesen Fällen haben wir
\[
\F[ K_{\delta,\alpha}^{(l)}(\cdot,0)](k)= \frac{1}{8 \pi |\det B|} f(k) \psi(k)
\] 
mit entsprechendem $d=d(\alpha)$ und Polynomen $q_1$ und $q_2$ (deren Koeffizienten von $\delta$ abhängen dürfen) in der Darstellung von $\psi$ (vgl. Gleichung \eqref{EqPsi}). Damit ist für $\ell=1,2$,

\begin{align*}
8 \pi |\det B| D^\nu \F[ K_{\delta,\alpha}^{(l)}(\cdot,0)](k)
&= \sum_{0 \leq \mu \leq \nu} \begin{pmatrix}
\nu \\ \mu
\end{pmatrix} D^\mu f(k) \cdot D^{\nu-\mu}\psi(k) \\
&= i^{\alpha_1+\alpha_2+1}\left(\sum_{o=-2|\nu|}^{\tilde \ell} \frac{r_o(k)}{|m|^o}e^{-a^2} + \sum_{o=-2|\nu|}^{l_1} \frac{\tilde r_o(k)}{|m|^o}\erfc(a) \right)
\end{align*}
bzw. 
\[
8 \pi |\det B| D^\nu \F[ K_{\delta,\alpha}^{(l)}(\cdot,0)](k) = i^{\alpha_1+\alpha_2}\left(\sum_{o=-2|\nu|}^{\tilde \ell} \frac{r_o(k)}{|m|^o}e^{-a^2} + \sum_{o=-2|\nu|}^{l_1} \frac{\tilde r_o(k)}{|m|^o}\erfc(a) \right)
\]
für $l=3$ mit Polynomen $r_0$, $\tilde r_0$, deren Koeffizienten von $\delta$ abhängen dürfen.

Für die Fälle mit $\ell=1,2$ und $\alpha_3=0$ hat $\F [K_{\delta,\alpha}^{(\ell)}(\cdot,0)](k_1,k_2)$ nach Lemma \ref{LFKdelta} die Darstellung
\begin{align*}
\F [K_{\delta,\alpha}^{(\ell)}(\cdot,0)](k_1,k_2)
&=\frac{i^{\alpha_1+\alpha_2+1} m^{\alpha_1+\alpha_2+ \tilde e_\ell}}{8 \pi|\det B| |m|} \cdot
2 \erfc(a) \\
&=\frac{2}{8\pi|\det B|}\frac{f(k)}{|m|}\erfc(a)
\end{align*} 
mit $\tilde e_\ell \in \R^2$ dem $\ell$-ten Einheitsvektor und $f$ wie oben mit $d=\alpha_1+\alpha_2+e_\ell$. Für die Abbildung $k\ \mapsto \frac{f(k)}{|m|}$ und $\nu \in \N_0^2$ berechnen wir 
\begin{align*}
D^\nu \left[ k\ \mapsto \frac{f(k)}{|m|} \right] 
&= \sum_{0 \leq \mu \leq \nu} \begin{pmatrix}
\nu \\ \mu 
\end{pmatrix} D^\mu f(k) D^{\nu-\mu} |m|^{-1}(k) \\
&= \sum_{0 \leq \mu \leq \nu} \begin{pmatrix}
\nu \\ \mu 
\end{pmatrix} \tilde f_\mu(k) \sum_{o=0}^{|\nu-\mu|}|m|^{-1-2o}p_{-1,\nu-\mu,o}(k) \\
&= \sum_{o=0}^{|\nu|} |m|^{-1-2o} \tilde p_{o,\nu}(k)
\end{align*}
mit Polynomen $\tilde p_{o,\nu}$. Damit haben wir
\begin{align*}
& \quad \frac{4\pi |\det B|^2 }{i^{\alpha_1+\alpha_2+1}}D^\nu \F[ K_{\delta,\alpha}^{(l)}(\cdot,0)](k) \\
&= \sum_{0 \leq \mu \leq \nu} \begin{pmatrix}
\nu \\ \mu
\end{pmatrix} D^\mu \frac{f(k)}{|m|} \cdot D^{\nu-\mu}\erfc(a) \\
&= \left[\sum_{0 \leq \mu < \nu} \begin{pmatrix}
\nu \\ \mu
\end{pmatrix} \sum_{o=0}^{|\mu|}\left( |m|^{-1-2o}  \tilde p_{o,\mu}(k) \right) \left( \sum_{o=0}^{|\nu-\mu|-1} \frac{r_{\nu-\mu,o}(k)}{|m|^o} e^{-a^2}\right)\right] \\
& \quad + \sum_{o=0}^{|\nu|}\left( |m|^{-1-2o}  \tilde p_{o,\nu}(k) \right) \erfc(a),
\end{align*}
d.h auch wieder eine Darstellung als Summe von Produkten von $\erfc(a)$ oder $e^{-a^2}$ mit Polynomen in $k$ und Monomen in $|m|$ oder $|m|^{-1}$.
 
Bei der Glättung mit $s_2$ haben wir nach \eqref{EqGdeltas2} und den Bemerkungen danach
\[
\F F_\rho(k) = \F \widetilde K_{\delta,\alpha}^{(\ell)}(\cdot,0)(\cdot)^\nu (k)
\]
mit
\[
\widetilde K_{\delta,\alpha}^{(\ell)}(\xi,b)=D^{\alpha+e_\ell}\left[z \mapsto \widetilde G_\delta(z) + \frac{2}{3 \sqrt{\pi} \delta } e^{-\frac{|z|^2}{\delta^2}}\right](\eta'(t) \xi - b n(t)).
\]
Wegen der Linearität von Fouriertransformation und Ableitung können wir die gesuchte Fouriertransformation zerlegen in
\begin{align*}
\F F_\rho(k) &= \F \widetilde K_{\delta,\alpha}^{(\ell)}(\cdot,0)(\cdot)^\nu (k) \\
&= \F \left(K_{\delta,\alpha}^{(\ell)}(\cdot,0)(\cdot)^\nu + E_{\delta,\alpha}^{(\ell)}(\cdot,0)(\cdot)^\nu\right)(k) \\
&= \F \left(K_{\delta,\alpha}^{(\ell)}(\cdot,0)(\cdot)^\nu\right)(k) +\F \left( E_{\delta,\alpha}^{(\ell)}(\cdot,0)(\cdot)^\nu \right)(k).
\end{align*}
Nach Lemma \ref{LFEdelta} ist $\F \left( E_{\delta,\alpha}^{(\ell)}(\cdot,0) \right)$ ebenfalls ein Produkt aus Polynomen in $m$ mit $\erfc(a)$. Damit erhalten wir mit der gleichen Rechnung wie bei $s_1$ die gleiche Darstellung von $\F F_\rho(k)$. 
\\
Bisher sind wir nicht auf die Abhängigkeit der auftretenden Polynome $r_o$ und $\tilde r_o$ von $\delta$ eingegangen, um die Darstellung einfach zu halten. Wir müssen noch einsehen, dass bei den Ableitungen von $\F[ K_{\delta,\alpha}^{(l)}(\cdot,0)](k)$ nach $k$ lediglich Monome in $\delta$ und $\delta^{-1}$ als Faktoren auftreten. Die Darstellungen aus Lemma \ref{LFKdelta} und Lemma \ref{LFEdelta} erfüllen diese Eigenschaft bereits, d.h. für $\nu=0$ haben wir die im Lemma formulierte Abhängigkeit der Polynome von $\delta$ bereits gezeigt. Induktiv sieht man leicht ein, dass dies für alle $\nu\in \N_0^2$ gilt, da nur die Ableitungen der Funktionen $k \mapsto e^{-a^2}$ und $k \mapsto \erfc(a)$ (Gleichungen \eqref{EqDexp} und\eqref{EqDerfc}) zusätzliche von $\delta$ abhängige Faktoren liefern. Da dabei jeweils nur Monome in $\delta$ hinzukommen, folgt die Behauptung.   
%Für die Abhängigkeit von $\delta$ stellt man fest, dass sich in allen Fällen 
\end{proof}
Wir können nun Lemma \ref{LFourierconst} anwenden und die Konstanten $c_k(\rho)$ aus Satz \ref{Sdiskeps} genauer bestimmen.
\begin{Lemma} \label{Lck}
Seien $\varphi \in C_{\per}^\infty(\R^2)$ und $\eta \in C_{\per}^\infty(\R^2, \partial D)$ und $\rho=\delta/h \geq \rho_0 >0$. Für die Konstanten $c_{k+2}^{(j)}(\rho)$,  $k=0,1,2$ aus Satz \ref{Sdiskeps} gilt für $j=1,2$
\begin{align*}
c_{k+2}^{(j)}(\rho) &\leq C ||\varphi||_{2,\infty} e^{-c \rho^2}
\end{align*}
mit von $\rho$ unabhängigen Konstanten $C,c>0$.
\end{Lemma}
\begin{proof}
Wir wenden Lemma \ref{LFourierconst} mit den Funktionen 
\[
F_\rho(\xi)= D^\alpha G_{\delta,l}(\eta'(t)\xi) \xi^{\nu} \quad \text{ für alle } \xi \in \R^2
\] an, wobei $0 \leq |\alpha|\leq 3$, $\ell=1,2,3$, $t\in Q$ und $|\nu| \geq 2|\alpha|+ 1$.
Für $|\alpha|=0$ hatten wir im Beweis von Satz \ref{Sdiskeps} gesehen, dass wir durch Entwicklung der Normalen und der Funktion $\varphi$ jeweils einen Term $\xi^{e_i}$, $i=1$ oder $i=2$ ausklammern können, also $|\nu|\geq 2$ für $\alpha=0$. Insgesamt haben wir also $|\nu| \geq \max\{2|\alpha|+ 1,2\}$.
Da uns Satz \ref{Sdiskeps} die Fehlerdarstellung aus der Voraussetzung von Lemma \ref{LFourierconst} mit $p=|\nu|-|\alpha|-2$ liefert und wir die Glattheit von $\varphi$ und $\eta$ vorausgesetzt haben, müssen wir nur noch die Homogenität
\[
h^{-p} F_{\rho h}(\xi) = F_\rho(\xi/h), \qquad \xi \in \R^2
\]
nachrechnen. Nach Lemma \ref{LDa} haben wir
\[
D^\alpha G_{\delta,l}(z) = \sum_{|\beta|\leq |\alpha|+1}\frac{1}{\delta^{|\beta|+|\alpha|+2}}z^\beta P_{\alpha,\beta,\ell} \left(\frac{|z|^2}{\delta^2}\right), \qquad z\in\R^3
\]
mit entsprechenden glatten Funktionen $P_{\alpha,\beta,\ell}$. Damit erhalten wir
\begin{align*}
F_\rho(\xi/h) &=  D^\alpha G_{\delta,l}\left(\eta'(t)\left(\frac{\xi}{h}\right)\right) \left(\frac{\xi}{h}\right)^{\nu} \\
&= \sum_{|\beta|\leq |\alpha|+1}\frac{1}{\delta^{|\beta|+|\alpha|+2}}\left(\eta'(t)\left(\frac{\xi}{h}\right)\right)^\beta P_{\alpha,\beta,\ell} \left(\frac{|\left(\eta'(t)\left(\frac{\xi}{h}\right)\right)|^2}{\delta^2}\right)\left(\frac{\xi}{h}\right)^{\nu} \\
&= \sum_{|\beta|\leq |\alpha|+1}\frac{1}{h^{|\beta|+|\nu|}} \frac{1}{\delta^{|\beta|+|\alpha|+2}}\left(\eta'(t)\xi \right)^\beta P_{\alpha,\beta,\ell} \left(\frac{|\left(\eta'(t)\xi\right)|^2}{h^2 \delta^2}\right)\xi^{\nu} \\
&= \frac{1}{h^{|\nu|-|\alpha|-2}}\sum_{|\beta|\leq |\alpha|+1}\frac{1}{h^{|\beta|+|\alpha|+2}} \frac{1}{\delta^{|\beta|+|\alpha|+2}}\left(\eta'(t)\xi \right)^\beta P_{\alpha,\beta,\ell} \left(\frac{|\left(\eta'(t)\xi\right)|^2}{h^2 \delta^2}\right)\xi^{\nu}\\
&= h^{-p} F_{\rho h}(\xi)
\end{align*}
für alle $\xi \in \R^2$. Lemma \ref{LFourierconst} liefert uns also für den Fehlerbeitrag $ch^{p+2} + O(h^{p+3})$ von den Summanden der Form (i) aus dem Beweis von Satz \ref{Sdiskeps}, dass
\[
c= (2 \pi)^2 f(0) \sum_{\mu \in \Z^2 \backslash \{0\} } \F F_\rho(2 \pi \mu).
\] 
Wegen \[F_{\rho h}(\xi)f(\xi)= S_{\alpha,\beta_,\nu,\gamma,\ell}(t,t-\xi)\left((\partial_1 \eta (t-\xi) \times \partial_2 \eta(t-\xi))_\ell\chi(\xi)\left[\xi \cdot \grad\varphi(t) + R_\varphi(t,t-\xi) \right] \right)\] 
mit $(k,\alpha,\nu,\beta,\gamma) \in J$, können wir $(2\pi)^2 |f(0)| \leq C ||\varphi||_{2,\infty}$ abschätzen, mit einer von $\eta$ abhängigen Konstante $C>0$. 

Den Reihenwert schätzen wir mit der Darstellung der Fouriertransformation aus Lemma \ref{LFFrho} ab, d.h. 
\begin{align*}
\F F_{\rho h}(k) = i^{d} \left(\sum_{o=-2|\nu|}^{\tilde \ell} \frac{r_o(k,\delta)}{|m|^o}e^{-a^2} + \sum_{o=-2|\nu|}^{\tilde \ell} \frac{\tilde r_o(k,\delta)}{|m|^o}\erfc(a) \right) \fa k \in \R^2 \backslash\{0\}
\end{align*}
mit \[
d= 
\begin{cases} 
|\nu|+\alpha_1+\alpha_2+1 &\text{ für } \ell=1,2\\
|\nu|+\alpha_1+\alpha_2 &\text{ für } \ell=3
\end{cases} 
\]
und $m$ und $a$ aus Lemma \ref{LFFrho}. Wir haben also $m= B^{-1} k $ und $a= \frac{|m| \delta}{2}= \frac{|B^{-1}k|\delta}{2}$. 
Damit ist für $k \neq 0$ und $\rho h = \delta$
\begin{align*}
i^{-d}\F F_{\rho}(k) 
&= \left(\sum_{o=-2|\nu|}^{\tilde \ell} \frac{r_o(k,\rho)}{|m|^o}e^{-\left(\frac{|B^{-1}k|\rho}{2}\right)^2} + \sum_{o=-2|\nu|}^{\tilde \ell} \frac{\tilde r_o(k,\rho)}{|m|^o}\erfc \left(\frac{|B^{-1}k| \rho}{2} \right) \right) \\
&= \left(\sum_{o=-2|\nu|}^{\tilde \ell} \frac{r_o(k,\rho)}{|m|^o}e^{-\left(\frac{|B^{-1}k|\rho}{2}\right)^2} + \sum_{o=-2|\nu|}^{\tilde \ell} \frac{\tilde r_o(k,\rho)}{|m|^o}\erfc \left(\frac{|B^{-1}k| \rho}{2} \right) \right) \\
&= \sum_{o=-2|\nu|}^{\tilde \ell} a_{o,\rho}(k) + b_{o,\rho}(k) 
\end{align*}
mit $a_{o,\rho}(k)= \frac{r_o(k,\rho)}{|m|^o}e^{-\left(\frac{|B^{-1}k|\rho}{2}\right)^2}$ und $b_{o,\rho}(k)=\frac{\tilde r_o(k,\rho)}{|m|^o}\erfc \left(\frac{|B^{-1}k| \rho}{2} \right)$. 
Um die Reihe $\sum_{\mu \in \Z^2 \backslash \{0\} } \F F_\rho(2 \pi \mu)$ abzuschätzen stellen wir zunächst fest, dass 
\[
|k| = |B B^{-1} k| \leq |B| |B^{-1} k| \quad \text{ für alle }k\in \R^2,
\] und sich die Summanden $a_{o,\rho}(k,\delta)$ für $|k| \geq 2\pi$ schreiben lassen als
\begin{align*}
|a_{o,\rho}(k) |
&= \left|\frac{r_o(k,\rho)}{|m|^o}e^{-\left(\frac{|B^{-1}k|\rho}{2}\right)^2} \right| \\
&= \left(\frac{|r_o(k,\rho)|}{|B^{-1}k|^o}e^{-\frac{1}{2}\left(\frac{|B^{-1}k|\rho}{2}\right)^2}\right)e^{-\frac{1}{2}\left(\frac{|B^{-1}k|\rho}{2}\right)^2}  \\
&\leq |B|^o \left(\frac{|r_o(k,\rho)|}{|k|^o}e^{-\frac{1}{2}\left(\frac{|B^{-1}k|\rho}{2}\right)^2}\right)e^{-\frac{1}{2}\left(\frac{|B^{-1}k|\rho}{2}\right)^2} \\
&\leq (2\pi)^o |B|^o \left(|r_o(k,\rho)|e^{-\frac{1}{2}\left(\frac{|B^{-1}k|\rho}{2}\right)^2}\right)e^{-\frac{1}{2}\left(\frac{|B^{-1}k|\rho}{2}\right)^2} \\
&\leq  C\left(|r_o(k,\rho)|e^{-\frac{1}{2}\left(\frac{|B^{-1}k|\rho}{2}\right)^2}\right)e^{-\frac{1}{2}\left(\frac{|B^{-1}k|\rho}{2}\right)^2}.
\end{align*} Wir benutzen die Höldersche Ungleichung auf dem Maßraum $\Z^2 \backslash \{0 \}$, versehen mit der Potenzmenge und dem Zählmaß und erhalten 
\begin{align*}
\sum_{k \in \Z^2 \backslash \{0\} }|a_{o,\rho}(2 \pi k)| 
\leq C \sup_{k \in \Z^2\backslash \{0\} }
\left(|r_o(2\pi k,\rho)|e^{-\frac{1}{2}\left(\frac{2 \pi |B^{-1}k|\rho}{2}\right)^2} \right)
\sum_{k \in \Z^2 \backslash \{0\} }e^{-\frac{1}{2}\left(\frac{2 \pi|B^{-1}k|\rho}{2}\right)^2},
\end{align*}
wobei für $k \in \Z^2 \backslash \{0\}$
\begin{align*}
|r_o(2\pi k,\rho)|e^{-\frac{1}{2}\left(\frac{2 \pi |B^{-1}k|\rho}{2}\right)^2}
&\leq |r_o(2\pi k,\rho)|e^{-\frac{\pi^2 \rho^2}{2|B|^2} |k|^2} \\
&= |r_o(2\pi k,\rho)|e^{-u\rho^2|k|^2} \\
&=: p(k,\rho)
\end{align*}
mit $u=\frac{\pi^2}{2|B|^2}$. Wir wollen $p(k,\rho)$ gleichmäßig abschätzen für alle $k \in \Z^2 \backslash \{0\}$ und $\rho>\rho_0$. Dazu verwenden wir die Darstellung von $r_o$ aus Lemma \ref{LFFrho}, d.h.
\[
r_o(2\pi k,\rho) = \sum_{|\beta| \leq \beta_{\max}} C_{\beta} p_{\beta}(\rho)  (2 \pi k)^{\beta}, \fa \rho>0, k \in \R^2
\]
mit Laurent-Polynomen $p_{\beta}$.
Damit lässt sich $p(k,\rho)$ mit der Dreiecksungleichung gegen eine endliche Summe von Funktionen der Form
\[
\tilde p(k,\rho) =\widetilde C \rho^j |k|^\beta e^{-u \rho^2 |k|^2} \fa k \in \R^2, \rho \in \R_{\geq \rho_0}
\]
mit $\N_0 \ni \beta\leq \beta_{\max},j \in \Z$ und Konstanten $\widetilde C \in \R$ abschätzen. 
Für $j\leq 0$ können wir $\rho \geq \rho_0$ verwenden und erhalten, da $k \mapsto e^{-u \rho_0^2 |k|^2}$ eine Schwartz-Funktion ist
\begin{align*}
\tilde p(k,\rho) &\leq \widetilde C \rho_0^j |k|^\beta e^{-u \rho_0^2 |k|^2}  \\
& \leq C
\end{align*}
für alle $k \in \R$ und $\rho\geq \rho_0$ mit einer von $k$ und $\rho$ unabhängigen Konstante $C>0$.
Für $j>0$ verwenden wir ebenfalls, dass $k \mapsto e^{-c k^2}$ für alle $c>0$ eine Schwartz-Funktion ist und erhalten
\begin{align*}
\tilde p(k,\rho) &=\widetilde C \rho^j |k|^\beta e^{-u \rho^2 |k|^2}\\
&= \widetilde C \left(\rho^j e^{-\frac{1}{2}u \rho^2 |k|^2} \right) \left(|k|^\beta e^{-\frac{1}{2}u \rho^2 |k|^2} \right) \\
& \leq  \widetilde C \left(\rho^j e^{-\frac{1}{2}u \rho^2} \right) \left(|k|^\beta e^{-\frac{1}{2}u \rho_0^2 |k|^2} \right) \\
&\leq C
\end{align*}
für alle $k \in \R$ und $\rho \geq \rho_0$ mit einer von $\rho$ unabhängigen Konstante $C>0$.
Insgesamt haben wir damit gezeigt, dass
\[
\sup_{k \in \Z^2\backslash \{0\} }
\left(|r_o(2\pi k,\rho)|e^{-\frac{1}{2}\left(\frac{2 \pi |B^{-1}k|\rho}{2}\right)^2} \right) \leq C
\]
für alle $\rho \geq \rho_0$, mit einer von $\rho$ unabhängigen Konstante $C>0$.
%\todo{Quelle: $k \mapsto e^{-c|k|^2}$ ist Schwartz-Funktion} \\

Somit erhalten wir mit $u= \frac{\pi^2}{2|B|^2}>0 $ 
\begin{align*}
\sum_{k \in \Z^2 \backslash \{0\} }|a_{o,\rho}(2 \pi k)| 
&\leq C
\sum_{k \in \Z^2 \backslash \{0\} }e^{-\frac{1}{2}\left(\frac{2 \pi|B^{-1}k|\delta}{2h}\right)^2} \\
& \leq C \sum_{\mu \in \Z^2 \backslash \{0\} }e^{-u |k|^2\rho^2} \\
& = C \left(\sum_{k \in (\Z \backslash \{0\})^2 }e^{-u |k|^2\rho^2} + \sum_{k_1 \in (\Z \backslash \{0\}) }e^{-u k_1^2\rho^2} + \sum_{k_2 \in (\Z \backslash \{0\}) }e^{-u k_2^2\rho^2} \right) \\
&= C (s_1 + 2 s_2),
\end{align*}
mit $s_1=\sum_{k \in (\Z \backslash \{0\})^2 }e^{-u |k|^2\rho^2}$ und $s_2=\sum_{k_1 \in (\Z \backslash \{0\}) }e^{-u k_1^2\rho^2}$, wobei wir die Menge über die summiert wird in $\Z^2 \backslash \{0\} = (\Z \backslash \{0\})^2 + (\Z \backslash \{0\})\times \{0\} + \{0\} \times (\Z \backslash \{0\})$ zerlegt haben. Die Reihe dürfen wir umordnen, da die Reihenglieder positiv sind. Diese Zerlegung können wir (mit dem Satz von Fubini-Tonelli für das Zählmaß) weiter vereinfachen, da 
\begin{align*}
s_1 
&= \sum_{k \in (\Z \backslash \{0\})^2 }e^{-u |k|^2\rho^2} \\
&= \sum_{k_1 \in (\Z \backslash \{0\})} \sum_{k_2 \in (\Z \backslash \{0\})} e^{-u k_1^2\rho^2}e^{-u k_2^2\rho^2} \\
&=\sum_{k_1 \in (\Z \backslash \{0\})} e^{-u k_1^2\rho^2} \sum_{k_2 \in (\Z \backslash \{0\})}e^{-u k_2^2\rho^2} \\
&= s_2^2.
\end{align*} 
Wegen der Symmetrie von $k_1 \mapsto e^{-u k_1^2\rho^2}$ haben wir zusätzlich
\begin{align*}
s_2 
&= \sum_{k_1 \in (\Z \backslash \{0\})} e^{-u k_1^2\rho^2} \\
&= 2 \sum_{k_1 \in \N_{>0}} e^{-u k_1^2\rho^2} \\
&= 2 e^{-u \rho^2} \sum_{k_1 \in \N_{>0}} e^{-u (k_1^2-1)\rho^2} \\
& \leq 2 e^{-u \rho^2} \sum_{k_1 \in \N_{>0}} e^{-u (k_1-1)^2\rho^2} \\
&\leq 2 e^{-u \rho^2} \sum_{k_1 \in \N_0} e^{-u k_1^2\rho_0^2} \\
& \leq C  e^{-u \rho^2},
\end{align*}
wobei man die Konvergenz der Reihe $\sum_{k_1 \in \N_0} e^{-u k_1^2\rho_0^2}$ wegen der Monotonie von $k_1 \mapsto e^{-u k_1\rho^2}$ sofort mit dem Wurzelkriterium einsieht. Insgesamt haben wir also gezeigt, dass alle verwendeten Reihen konvergieren und
\begin{align*}
\left|\sum_{k \in \Z^2 \backslash \{0\} }a_{o,\rho}(2 \pi k)\right| &\leq 
\sum_{k \in \Z^2 \backslash \{0\} }|a_{o,\rho}(2 \pi k)| \\
&\leq C (s_1 + 2s_2) \\
& =C (s_2^2+2s_2) \\
& \leq  C \left(e^{- 2c \rho^2} + 2e^{- c \rho^2} \right) \\
& \leq C e^{-u \rho^2}
\end{align*}
mit von $\rho$ unabhängigen Konstanten $C,c>0$. Für die Reihen zu $b_{o,\rho}(k)$ stellen wir fest, dass für $|k| \geq 2\pi$ die Ungleichung
\[
0 \leq \erfc \left( \frac{|B^{-1}k| \rho}{2} \right) \leq  e^{-\left( \frac{|B^{-1}k| \rho}{2} \right)^2} \leq e^{- c \rho^2}
\]
mit einer von $\rho$ unabhängigen Konstanten $c>0$ gilt, denn für $a \in \R_{\geq 0}$ ist
\[
\erfc(a)= \frac{2}{\sqrt{\pi}}\int_a^\infty e^{-r^2} \dd r = \frac{2}{\sqrt{\pi}}\int_0^\infty e^{-(t+a)^2} \dd t \leq \frac{2}{\sqrt{\pi}}e^{-a^2} \int_0^\infty e^{-t^2} \dd t = e^{-a^2}.
\] Damit können wir für die Reihen zu $b_{o,\rho}(k)$ die gleichen Abschätzungen wie für $a_{o,\rho}(k)$ verwenden und erhalten wie dort
\[
\left|\sum_{k \in \Z^2 \backslash \{0\} }b_{o,\rho}(2 \pi k)\right| \leq C e^{-u \rho^2}.
\]
Summieren wir nun die Abschätzungen über alle $o$ mit $-2|\nu| \leq o \leq \tilde \ell$ auf, so erhalten wir die Behauptung.
\end{proof}
\section{Zusammenfassung der Fehler}
Mit den Abschätzungen aus Kapitel \ref{chaRegErr} und Kapitel \ref{chaDiskErr} können wir nun Abschätzungen für den Gesamtfehler zwischen den im Kapitel \ref{chaKernapprox} definierten Operatoren angeben. 
\begin{Satz}
Sei $\eta \in C_{\per}^{\infty}(\R^2,\partial D)$ und $\rho_0>0$. Sei zudem $1 \geq h,\delta \geq 0$ mit $\delta/h \geq \rho_0$ und $N \in \N$. Dann gibt es Konstanten $C,c>0$, so dass für die Operatoren aus Kapitel \ref{chaKernapprox} gilt
\begin{align*}
||(A-A_{j,\delta})\varphi||_{\infty}  &\leq C ||\varphi||_{3,\infty} \delta^{2j+1} \\
||(A_{j,\delta} - A_{j,\delta,N})\varphi||_{\infty} &\leq C||\varphi||_{15,\infty} [h^2 \exp(-c \delta^2/h^2)+h^5].
\end{align*}
für alle $\varphi \in C_{\per}^\infty(\R^2)$ und $j=1,2$. Gilt zudem $\delta = h^q$ mit $q<1$, so folgt
\begin{align*}
||(A-A_{2,\delta,N}) \varphi||_\infty &\leq C ||\varphi||_{15,\infty} \delta^5 \\
 &= C ||\varphi||_{15,\infty} h^{5q} 
\end{align*}
\end{Satz}
\begin{proof}
Die Aussage zu
\[
||(A-A_{j,\delta})\varphi||_{\infty} 
\]
für $j=1,2$ folgt aus Lemma \ref{LepsohneSing}, Satz \ref{Seps1} und \ref{Seps2}.
Die Abschätzung zu
\[
||(A_{j,\delta} - A_{j,\delta,N})\varphi||_{\infty}
\]
für $j=1,2$ ist eine Zusammenfassung der Sätze \ref{Sdiskeps} und \ref{SDiskeps2} und Lemma \ref{Lck}.
Für die letzte Abschätzung nutzen wir, dass $x \mapsto \exp(-c x^{2})$ eine Schwartz-Funktion ist und berechnen für $\delta=h^q$, $1-q=:d>0$ und $x=h^{-d}$
\begin{align*}
h^{-3} \exp(-c \delta^2/h^2) 
&= h^{-3} \exp(-c h^{2q}/h^2) \\
&= h^{-3}\exp(-c h^{-2d}) \\
&= x^{3/d}\exp(-c x^{2}) \\
& \leq x^{\lceil 3/d \rceil}\exp(-c x^{2}) \\
& \leq C
\end{align*}
für alle $x \in [1, \infty)$, d.h. $1 \geq h>0$, und eine Konstante $C>0$.
Daraus folgt insbesondere 
\begin{align*}
h^{2} \exp(-c \delta^2/h^2) 
&= h^5 h^{-3} \exp(-c \delta^2/h^2) \\
& \leq C h^5
\end{align*}
für alle $1 \geq h>0$, und mit
\[
||(A-A_{2,\delta,N}) \varphi||_\infty \leq ||(A-A_{2,\delta})\varphi||_{\infty} + ||(A_{2,\delta} - A_{2,\delta,N})\varphi||_{\infty} \fa \varphi \in C_{\per}^\infty(\R^2)
\]
folgt die Behauptung.

\end{proof}
\newpage
%\section{Numerische Tests}
%In diesem Abschnitt wollen wir das bisher vorgestellte Verfahren zur Lösung einer Integralgleichung mit schwach singulärer Kernfunktion verwenden. Auch hierzu widmen wir uns wieder der zum Dirichletproblem \eqref{LGl} äquivalenten Integralgleichung \eqref{IntGl} (Siehe Abschnitt \ref{Intmethode}). Das zu lösende Problem besteht wegen Satz \ref{MinSing} also darin für ein gegebenes beschränktes Gebiet $D \subset \R^3$ mit Randparametrisierung $\eta :Q \to \partial D$ und einer Funktion $g \in C^2(\partial D)$ ein Potential $\varphi \in C(\partial D)$ zu finden, das die Gleichung
%\begin{equation} \label{NIntgl}
%\varphi(x) - \int_{\partial D} \frac{\partial \Phi(x,y)}{\partial n (y)}(\varphi(y)-\varphi(x)) \dd s(y) = - g(x) \fa x \in \partial D
%\end{equation} 
%mit 
% \[
%    \Phi(x,y) = \frac{1}{4 \pi} \frac{1}{|x-y|} \fa x,y \in \R^3, x\neq y
%\] erfüllt.
%Wir wählen $D$ als Torus der durch die Parametrisierung $\eta \in C^\infty(Q,D)$ 
%\[
%\eta(t) = \begin{pmatrix} 
%(R_1 + R_2 \cos(t_2)) \cos (t_1) \\
% (R_1 + R_2 \cos(t_2)) \sin(t_1) \\
% R_2 \sin(t_2)
%\end{pmatrix}, \qquad t= \begin{pmatrix}
%t_1 \\ t_2
%\end{pmatrix} \in Q=[-\pi,\pi)^2
%\]
%mit $R_1 =1.0$ und $R_2=0.25$. Für die rechte Seite $g$ wählen wir die nach Definition und Satz \ref{DFund} harmonische Funktion 
%\begin{align*}
%g(x) &:= \Phi(x,(2,1,1)^T)\\
%&=\frac{1}{4 \pi} \frac{1}{|x-(2,1,1)^T|} \fa x \in \partial D
%\end{align*}
%Die Parametrisierung $\eta$ lässt sich bezüglich beider Variablen periodisch auf $\R^2$ fortsetzen. Nach dem Transformationssatz und der Regularisierung des Kerns wird aus
%\eqref{NIntgl} damit
%\begin{align*}
%\varphi(\eta(t)) - \int_{Q} K_{j,\delta}(\eta(t),\eta(\tau))(\varphi(\eta(\tau))&-\varphi(\eta(t))) |\partial_1 \eta (\tau) \times \partial_2 \eta(\tau)| \dd \tau \\&= - g(\eta(t)) \fa t \in Q
%\end{align*}
%\todo[inline]{Auswertung numerische Experimente EOC}
%%\section{Ausblick: Helmholtzgleichung}
%%%%%%%%%%%%%%%%%%%%%%%%%%%%%%%%%%%
%%\todo[inline]{Ausblick Helmholtzgleichung}

  % Literaturverzeichnis (beginnt auf einer ungeraden Seite)
  \newpage

\begin{thebibliography}{Lam00}
  \bibitem{kress} Rainer Kress, \emph{Linear Integral Equations}. Springer Verlag, 3. Auflage 2014.
  \bibitem{beale} J. Thomas Beale, \emph{A Grid-Based Boundary Integral Method In 3D}, DOI: 10.1137/S0036142903420959, 2004.
  \bibitem{Borceux} Francis Borceux, \emph{A Differential Approach to Geometry}, Springer 2014.
  \bibitem{Anderson} C. Anderson and C. Greengard, \emph{On vortex methods}, SIAM J. Numer. Anal., 22, 413-440
(1985).
  \bibitem{Grafakos} Loukas Grafakos, \emph{Classical Fourier Analysis}, Third Edition, Springer 2010
(1985).
  \bibitem{Forster} Otto Forster, \emph{Analysis 2}, 10. Auflage, Springer 2013
  \bibitem{Stein} Elias M. Stein, Guido Weiss, \emph{Introduction to Fourier analysis on Euclidean
spaces}, Princeton, New Jersey: Princeton University Press, ISBN: 0-691-08078-X,
1971.
  \bibitem{Stegun} Milton Abramowitz, Irene A. Stegun (Hrsg.): \emph{Handbook of Mathematical Functions with Formulas, Graphs, and Mathematical Tables}. Dover, New York 1972.
  \bibitem{Dym}(Probability and Mathematical Statistics) H. Dym, H. P. McKean, David Aldous, Y. L. Tong, \emph{Fourier series and integrals}, Academic Press (1985).
  \bibitem{Collet} S.P. Collet, \emph{Numerische Lösung von Randintegralgleichungen durch Glättung der Kernfunktionen}, Master thesis, KIT, Karlsruhe (2015).
   \bibitem{Arens} Tilo Arens, \emph{Scattering by Biperiodic Layered Media: The Integral Equation Approach}, Habilitationsarbeit, Karlsruhe (2010).
   \bibitem{schutte}Christof Schütte, "Was ein CT mit Algorithmen zu tun", in: welt.de (10.03.2008) unter: https://www.welt.de/wissenschaft/article1781601/Was-ein-CT-mit-Algorithmen-zu-tun.html (aufgerufen am 18.01.2017)
 \end{thebibliography}

      
  % ggf. hier Tabelle mit Symbolen 
  % (kann auch auf das Inhaltsverzeichnis folgen)

%\newpage
%  
% \thispagestyle{empty}
%
%
%\vspace*{8cm}
%
%
%\section*{Erklärung}
%
%Hiermit versichere ich, dass ich diese Arbeit selbständig verfasst und keine anderen, als die angegebenen Quellen und Hilfsmittel benutzt, die wörtlich oder inhaltlich übernommenen Stellen als solche kenntlich gemacht und die Satzung des Karlsruher Instituts für Technologie zur Sicherung guter wissenschaftlicher Praxis in der jeweils gültigen Fassung beachtet habe. \\[2ex] 
%
%\noindent
%Karlsruhe, den 20.01.2017\\[5ex]
%
%% Unterschrift (handgeschrieben)
%


\end{document}

